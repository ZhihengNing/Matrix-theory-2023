\ifx\allfiles\undefined
\documentclass[12pt, a4paper, oneside, UTF8]{ctexbook}
\def\configPath{../config}
\def\basicPath{\configPath/basic}


% 在这里定义需要的包
\usepackage{amsmath}
\usepackage{amsthm}
\usepackage{amssymb}
\usepackage{graphicx}
\usepackage{mathrsfs}
\usepackage{enumitem}
\usepackage{geometry}
\usepackage[colorlinks, linkcolor=black]{hyperref}
\usepackage{stackengine}
\usepackage{yhmath}
\usepackage{extarrows}
\usepackage{arydshln}
% \usepackage{unicode-math}
\usepackage{tasks}
\usepackage{fancyhdr}
\usepackage[dvipsnames, svgnames]{xcolor}
\usepackage{listings}

\definecolor{mygreen}{rgb}{0,0.6,0}
\definecolor{mygray}{rgb}{0.5,0.5,0.5}
\definecolor{mymauve}{rgb}{0.58,0,0.82}

\graphicspath{ {figure/},{../figure/}, {config/}, {../config/} }

\linespread{1.6}

\geometry{
    top=25.4mm, 
    bottom=25.4mm, 
    left=20mm, 
    right=20mm, 
    headheight=2.17cm, 
    headsep=4mm, 
    footskip=12mm
}

\setenumerate[1]{itemsep=5pt,partopsep=0pt,parsep=\parskip,topsep=5pt}
\setitemize[1]{itemsep=5pt,partopsep=0pt,parsep=\parskip,topsep=5pt}
\setdescription{itemsep=5pt,partopsep=0pt,parsep=\parskip,topsep=5pt}

\lstset{
    language=Mathematica,
    basicstyle=\tt,
    breaklines=true,
    keywordstyle=\bfseries\color{NavyBlue}, 
    emphstyle=\bfseries\color{Rhodamine},
    commentstyle=\itshape\color{black!50!white}, 
    stringstyle=\bfseries\color{PineGreen!90!black},
    columns=flexible,
    numbers=left,
    numberstyle=\footnotesize,
    frame=tb,
    breakatwhitespace=false,
} 
% 在这里定义自己顺手的环境
\def\d{\mathrm{d}}
\def\i{\mathrm{i}}
\def\R{\mathbb{R}}
\newcommand{\bs}[1]{\boldsymbol{#1}}
\newcommand{\mc}[1]{\mathcal{#1}}
\newcommand{\ora}[1]{\overrightarrow{#1}}
\newcommand{\myspace}[1]{\par\vspace{#1\baselineskip}}
\newcommand{\xrowht}[2][0]{\addstackgap[.5\dimexpr#2\relax]{\vphantom{#1}}}
\newenvironment{ca}[1][1]{\linespread{#1} \selectfont \begin{cases}}{\end{cases}}
\newenvironment{vx}[1][1]{\linespread{#1} \selectfont \begin{vmatrix}}{\end{vmatrix}}
\newcommand{\tabincell}[2]{\begin{tabular}{@{}#1@{}}#2\end{tabular}}
\newcommand{\pll}{\kern 0.56em/\kern -0.8em /\kern 0.56em}
\newcommand{\dive}[1][F]{\mathrm{div}\;\bs{#1}}
\newcommand{\rotn}[1][A]{\mathrm{rot}\;\bs{#1}} 
\usepackage[strict]{changepage} 
\usepackage{framed}

\definecolor{greenshade}{rgb}{0.90,1,0.92}
\definecolor{redshade}{rgb}{1.00,0.88,0.88}
\definecolor{brownshade}{rgb}{0.99,0.95,0.9}
\definecolor{lilacshade}{rgb}{0.95,0.93,0.98}
\definecolor{orangeshade}{rgb}{1.00,0.88,0.82}
\definecolor{lightblueshade}{rgb}{0.8,0.92,1}
\definecolor{purple}{rgb}{0.81,0.85,1}
\theoremstyle{definition}
\newtheorem{myDefn}{\indent 定义}[section]
% \newtheorem{myLemma}{\indent 引理}[section]
\newtheorem{myLemma}{\indent 引理}[chapter]
\newtheorem{myThm}[myLemma]{\indent 定理}
\newtheorem{myCorollary}[myLemma]{\indent 推论}
\newtheorem{myCriterion}[myLemma]{\indent 准则}
\newtheorem*{myRemark}{\indent 注}
\newtheorem{myProposition}{\indent 命题}[section]


\newenvironment{formal}[2][]{%
    \def\FrameCommand{%
        \hspace{1pt}%
        {\color{#1}\vrule width 2pt}%
        {\color{#2}\vrule width 4pt}%
        \colorbox{#2}%
    }%
    \MakeFramed{\advance\hsize-\width\FrameRestore}%
    \noindent\hspace{-4.55pt}%
    \begin{adjustwidth}{}{7pt}\vspace{2pt}\vspace{2pt}}{%
        \vspace{2pt}\end{adjustwidth}\endMakeFramed%
}

\newenvironment{defn}{\begin{formal}[Green]{greenshade}\vspace{-\baselineskip / 2}\begin{myDefn}}{\end{myDefn}\end{formal}}
\newenvironment{thm}{\begin{formal}[LightSkyBlue]{lightblueshade}\vspace{-\baselineskip / 2}\begin{myThm}}{\end{myThm}\end{formal}}
\newenvironment{lemma}{\begin{formal}[Plum]{lilacshade}\vspace{-\baselineskip / 2}\begin{myLemma}}{\end{myLemma}\end{formal}}
\newenvironment{corollary}{\begin{formal}[BurlyWood]{brownshade}\vspace{-\baselineskip / 2}\begin{myCorollary}}{\end{myCorollary}\end{formal}}
\newenvironment{criterion}{\begin{formal}[DarkOrange]{orangeshade}\vspace{-\baselineskip / 2}\begin{myCriterion}}{\end{myCriterion}\end{formal}}
\newenvironment{rmk}{\begin{formal}[LightCoral]{redshade}\vspace{-\baselineskip / 2}\begin{myRemark}}{\end{myRemark}\end{formal}}
\newenvironment{proposition}{\begin{formal}[RoyalPurple]{purple}\vspace{-\baselineskip / 2}\begin{myProposition}}{\end{myProposition}\end{formal}}

\newtheorem{example}{\indent \color{SeaGreen}{例}}[section]
\newtheorem{question}{\color{SeaGreen}{题}}[chapter]
% \renewenvironment{proof}{\indent\textcolor{SkyBlue}{\textbf{证明.}}\;}{\qed\par}
% \newenvironment{solution}{\indent\textcolor{SkyBlue}{\textbf{解.}}\;}{\qed\par}

\renewcommand{\proofname}{\textbf{\textcolor{TealBlue}{证明}}}
\newenvironment{solution}{\begin{proof}[\textbf{\textcolor{TealBlue}{解}}]}{\end{proof}}

\definecolor{mygreen}{rgb}{0,0.6,0}
\definecolor{mygray}{rgb}{0.5,0.5,0.5}
\definecolor{mymauve}{rgb}{0.58,0,0.82}

\graphicspath{ {figure/},{../figure/}, {config/}, {../config/},{cover/graph} }

\linespread{1.6}

\geometry{
    top=25.4mm, 
    bottom=25.4mm, 
    left=20mm, 
    right=20mm, 
    headheight=2.17cm, 
    headsep=4mm, 
    footskip=12mm
}

\setenumerate[1]{itemsep=5pt,partopsep=0pt,parsep=\parskip,topsep=5pt}
\setitemize[1]{itemsep=5pt,partopsep=0pt,parsep=\parskip,topsep=5pt}
\setdescription{itemsep=5pt,partopsep=0pt,parsep=\parskip,topsep=5pt}

\lstset{
    language=Mathematica,
    basicstyle=\tt,
    breaklines=true,
    keywordstyle=\bfseries\color{NavyBlue}, 
    emphstyle=\bfseries\color{Rhodamine},
    commentstyle=\itshape\color{black!50!white}, 
    stringstyle=\bfseries\color{PineGreen!90!black},
    columns=flexible,
    numbers=left,
    numberstyle=\footnotesize,
    frame=tb,
    breakatwhitespace=false,
} 

\begin{document}
\else
\fi

\chapter{矩阵函数}
\begin{question}(p79.1)
设函数矩阵$\bs{A}(t)=\begin{pmatrix}
    \sin t &-e^t & t \\
    \cos t & e^t & t^2 \\
    1 & 0 & 0
\end{pmatrix}$,
试求$\frac{\d }{\d t} \bs{A}(t),|\frac{\d }{\d t} \bs{A}(t)|,\lim\limits_{t \to 0} \bs{A}(t)$。
\end{question}

\begin{solution}
    $\frac{\d }{\d t} \bs{A}(t)=\begin{pmatrix}
    \cos t &-e^t & 1 \\
    -\sin t & e^t & 2t \\
    0 & 0 & 0
    \end{pmatrix}$,$|\frac{\d }{\d t} \bs{A}(t)|=0$,
    $\lim\limits_{t \to 0} \bs{A}(t)=\begin{pmatrix}
        0 &-1 & 0 \\
        1 & 1 & 0 \\
        1 & 0 & 0
    \end{pmatrix}$

\end{solution}

\begin{question}(p79.2)
    设函数矩阵$\bs{A}(t)=\begin{pmatrix}
        e^{2t}& te^t & 1\\
        e^{-t} & 2e^{2t} & 0\\
        3t & 0 & 0
    \end{pmatrix}$,试求$\int \bs{A}(t) dt,\int_{0}^u \bs{A}(t)dt$。
\end{question}

\begin{solution}
    $\int \bs{A}(t) dt=\begin{pmatrix}
        \frac{1}{2} e^{2t}& (t-1)e^t & t\\
        -e^{-t} & e^{2t} & 0\\
        \frac{3}{2}t^2 & 0 & 0
    \end{pmatrix}$
    
    $\int_{0}^u \bs{A}(t) dt=\begin{pmatrix}
        \frac{1}{2} e^{2u}& (u-1)e^u & u\\
        -e^{-u} & e^{2u} & 0\\
        \frac{3}{2}u^2 & 0 & 0
    \end{pmatrix}-\begin{pmatrix}
        \frac{1}{2} & -1 & 0\\
        -1 & 1 & 0\\
        0 &0& 0
    \end{pmatrix}=\begin{pmatrix}
        \frac{1}{2} e^{2u} -\frac{1}{2}& (u-1)e^u+1 & u\\
        -e^{-u}+1 & e^{2u}-1 & 0\\
        \frac{3}{2}u^2 & 0 & 0
    \end{pmatrix}$
\end{solution}

\begin{question}(p79.3)
    判断级数$\sum\limits_{n=0}^\infty \frac{1}{10^n}\begin{pmatrix}
        1 & 2 \\
        8& 1
    \end{pmatrix}^n$是否收敛,如果收敛,计算出结果。
\end{question}


\begin{solution}
    令$\bs{A}=\begin{pmatrix}
         1 & 2\\
         8 &1
    \end{pmatrix}$,特征值$\lambda_1=-3,\lambda_2=5$。

    则$\bs{A}=\bs{P}\bs{\Lambda}\bs{P}^{-1}=\bs{P}\begin{pmatrix}
        -3 & 0\\
        0 & 5
    \end{pmatrix} \bs{P}^{-1}$,其中$\bs{P}=\begin{pmatrix}
        1 & 1\\
        -2& 2
    \end{pmatrix},\bs{P}^{-1}=\begin{pmatrix}
        \frac{1}{2} & -\frac{1}{4} \\
        \frac{1}{2} & \frac{1}{4}
    \end{pmatrix}$,于是:
    \begin{align*}
        \sum\limits_{n=0}^\infty \frac{1}{10^n}\begin{pmatrix}
            1 & 2 \\
            8& 1
        \end{pmatrix}^n
        &=\begin{pmatrix}
            1 & 1\\
            -2& 2
        \end{pmatrix}  \sum\limits_{n=0}^\infty \frac{1}{10^n}\begin{pmatrix}
            -3 & 0 \\
            0& 5
        \end{pmatrix}^n \begin{pmatrix}
            \frac{1}{2} & -\frac{1}{4} \\
            \frac{1}{2} & \frac{1}{4}
        \end{pmatrix}\\
    &=\begin{pmatrix}
        1 & 1\\
        -2& 2
    \end{pmatrix}  \sum\limits_{n=0}^\infty \frac{1}{10^n}\begin{pmatrix}
        \frac{10}{13} & 0 \\
        0& 2
    \end{pmatrix} \begin{pmatrix}
        \frac{1}{2} & -\frac{1}{4} \\
        \frac{1}{2} & \frac{1}{4}
    \end{pmatrix}\\
    &=\begin{pmatrix}
        \frac{18}{13}&\frac{4}{13}\\
        \frac{16}{13} & \frac{18}{13}
    \end{pmatrix}
    \end{align*}
计算出级数的值恰恰说明了其收敛。

\end{solution}


\begin{question}(p79.4)
    已知矩阵$\bs{A}=\begin{pmatrix}
        0 & 1 \\
        -2 & 3
    \end{pmatrix},\bs{B}=\begin{pmatrix}
        0 & -1 \\
        4 & 4
    \end{pmatrix}$,试求$e^{\bs{A}},e^{\bs{B}}$。
\end{question}

\begin{solution}
    \begin{enumerate}[label=(\arabic*)]
        \item $\lambda_1=1,\lambda_2=2,m_{\bs{A}}(\lambda)=(\lambda-1)(\lambda-2)$,
        令$P(\lambda)=a_0+a_1\lambda$,则:
        \begin{align*}
            \left\{
                \begin{array}{ll}
                    P(\lambda_1)=P(1)=a_0+a_1=e\\
                    P(\lambda_1)=P(2)=a_0+2a_1=e^2
                \end{array}
                \right.
                \Rightarrow
                \left\{
                    \begin{array}{ll}
                        a_0=a_0=2e-e^2\\
                        a_1=e^2-e
                    \end{array}
                    \right.    
        \end{align*}
    \begin{align*}
        e^{\bs{A}}=P(\bs{A})
    =(2e-e^2)\bs{E}+(e^2-e)\bs{A}
    =\begin{pmatrix}
        2e-e^2 &e^2-e \\
        -2e^2+2e & 2e^2-e
    \end{pmatrix}
    \end{align*}
    \item $\lambda_1=\lambda_2=2,m_{\bs{A}}(\lambda)=(\lambda-2)^2$,
    令$P(\lambda)=a_0+a_1\lambda$,则:
    \begin{align*}
        \left\{
            \begin{array}{ll}
                P(\lambda)=P(2)=a_0+2a_1=e^2\\
                P'(\lambda)=P'(2)=a_1=e^2
            \end{array}
            \right.
            \Rightarrow
        \left\{
            \begin{array}{ll}
                a_0=-e^2\\
                a_1=e^2
            \end{array}
            \right.
    \end{align*}
    \begin{align*}
        e^{\bs{B}}=P(\bs{B})
    &=e^2\bs{E}+e^2\bs{B}\\
    &=\begin{pmatrix}
        -e^2 &-e^2 \\
        4e^2 & 3e^2
    \end{pmatrix}
    \end{align*} 
    \end{enumerate}
\end{solution}


\begin{question}(p79.5)
    \label{上题}
    已知矩阵$\bs{A}=\begin{pmatrix}
        0 & -\theta \\
        \theta & 0
    \end{pmatrix}$,试证$e^{\bs{A}t}=\begin{pmatrix}
        \cos \theta t & -\sin \theta t\\
        \sin \theta t & \cos \theta t
    \end{pmatrix}$。
\end{question}


\begin{proof} 
    $\lambda_1=\theta \i,\lambda_2=-\theta \i,m_{\bs{A}}(\lambda)=(\lambda-\theta \i)(\lambda+\theta \i)$,
    令$P(\lambda)=a_0+a_1\lambda$,则:
    \begin{align*}
        \left\{
            \begin{array}{ll}
                P(\lambda_1)=P(\theta \i)=a_0+a_1 \theta \i=e^{\theta t \i}\\
                P(\lambda_2)=P(-\theta \i)=a_0-a_1 \theta \i=e^{-\theta t \i}
            \end{array}
            \right.
            \Rightarrow
            \left\{
                \begin{array}{ll}
                    a_0=\cos \theta t\\
                    a_1=\frac{\sin \theta t}{\theta}
                \end{array}
                \right.
    \end{align*}
    \begin{align*}
        e^{\bs{A}t}=P(\bs{A})
    =\cos \theta t\bs{E}+\frac{\sin \theta t}{\theta} \bs{A}
    =\begin{pmatrix}
        \cos \theta t &-\sin \theta t \\
        -\sin \theta t & \cos \theta t
    \end{pmatrix}
    \end{align*} 
\end{proof}


\begin{question}(p79.6)
    设矩阵$\bs{A}=\begin{pmatrix}
        0 & -\theta \\
        \theta & 0
    \end{pmatrix}$,利用上题\ref{上题}结果求$e^{\bs{A}}$。
\end{question}

\begin{solution}
    令$\bs{A}=\sigma \bs{E}+\bs{B}$,其中$\bs{B}=\begin{pmatrix}
         0 & -\theta \\
         \theta & 0
    \end{pmatrix}$
    由于$\bs{E}$与$\bs{B}$可交换,并且根据题\ref{上题}可得:
    \begin{align*}
        e^{\bs{A}}=e^{\sigma \bs{E}+\bs{B}}=e^{\sigma \bs{E}} e^{\bs{B}}
        =\begin{pmatrix}
            e^{\sigma} & 0\\
            0 & e^{\sigma}
        \end{pmatrix} \begin{pmatrix}
            \cos \theta  &-\sin \theta  \\
            -\sin \theta  & \cos \theta 
        \end{pmatrix}=\begin{pmatrix}
            e^{\sigma} \cos \theta  &-e^{\sigma}\sin \theta  \\
            -e^{\sigma} \sin \theta  & e^{\sigma}\cos \theta 
        \end{pmatrix}
    \end{align*}
\end{solution}


\begin{question}(p80.7)
    设$\bs{A}=\begin{pmatrix}
        9 &-6 &  -7 \\
        -1 & -1 & 1 \\
        10 & -6 & -8
    \end{pmatrix}$,求$e^{2\bs{A}t}$。
\end{question}

\begin{solution}
    $\lambda_1=2,\lambda_2=\lambda_3=-1$,
    显然$\bs{A}$的若当标准形$\bs{J}=\begin{pmatrix}
        2 & & \\
         & -1 & 1\\
         & & -1
    \end{pmatrix}$,其中空白位置全是$0$。
    并且有$\bs{J}=\bs{P}^{-1}\bs{A}\bs{P} \Rightarrow \bs{P}\bs{J}=\bs{A}\bs{P}$。
    不妨令$\bs{P}=(\bs{p}_1,\bs{p}_2,\bs{p}_3)$,其中$\bs{P}$为非奇异矩阵,则:
    \begin{align*}
        \left\{
            \begin{array}{ll}
                \bs{A}\bs{p}_1=2\bs{p}_1 \\
                \bs{A}\bs{p}_2=-\bs{p}_2 \\
                \bs{A}\bs{p}_3=\bs{p}_2-\bs{p}_3
            \end{array} 
            \right.
            \Rightarrow
            \left\{
            \begin{array}{ll}
                \bs{p}_1=(1,0,1)^T \\
                \bs{p}_2=(2,1,2)^T \\
                \bs{p}_3=(5,1,2)^T
            \end{array} 
            \right.
    \end{align*}
    \begin{align*}
        \bs{P}=\begin{pmatrix}
            1 & 2 &5 \\
            0 & 1 &1 \\
            1 & 2 & 6
        \end{pmatrix} \quad
        \bs{P}^{-1}=\begin{pmatrix}
            4 & -2 &-3 \\
            1 & 1 &-1 \\
            -1 & 0 & 1        
        \end{pmatrix}
    \end{align*}
    于是
    \begin{align*}
    \bs{e}^{2\bs{A}t}=\bs{P}e^{2\bs{J}t}\bs{P}^{-1}
    =&\begin{pmatrix}
        1 & 2 &5 \\
        0 & 1 &1 \\
        1 & 2 & 6
    \end{pmatrix}\begin{pmatrix}
        e^{4t}& 0 & 0 \\
        0 & e^{-2t} & 2te^{-2t} \\
        0& 0 &e^{-2t}
    \end{pmatrix}
    \begin{pmatrix}
        4 & -2 &-3 \\
        1 & 1 &-1 \\
        -1 & 0 & 1        
    \end{pmatrix}\\
    =&\begin{pmatrix}
        4e^{4t}+(-4t-3)e^{-2t}& -2e^{4t}+2e^{-2t} & -3e^{4t}+(4t+3)e^{-2t} \\
        -2te^{-2t} & e^{-2t} & 2te^{-2t} \\
        4e^{4t}+(-4t-4)e^{-2t}& -2e^{4t}+2e^{-2t} &-3e^{4t}+(4t+4)e^{-2t} 
    \end{pmatrix}
    \end{align*}
\end{solution}

\begin{question}(p80.8)
    设$\bs{A}=\begin{pmatrix}
        3 &1 &  -3 \\
        -7 & -2 & 8 \\
        -2 & -1 & 4
    \end{pmatrix}$,求$e^{\bs{A}t}$。
\end{question}


\begin{solution}
    $\lambda_1=\lambda_2=2,\lambda_3=1$,
    显然$\bs{A}$的若当标准形$\bs{J}=\begin{pmatrix}
        2 &1 & \\
         & 2 & \\
         & & 1
    \end{pmatrix}$,其中空白位置全是$0$。
    并且有$\bs{J}=\bs{P}^{-1}\bs{A}\bs{P} \Rightarrow \bs{P}\bs{J}=\bs{A}\bs{P}$。
    不妨令$\bs{P}=(\bs{p}_1,\bs{p}_2,\bs{p}_3)$,其中$\bs{P}$为非奇异矩阵,则:
    \begin{align*}
        \left\{
            \begin{array}{ll}
                \bs{A}\bs{p}_1=2\bs{p}_1 \\
                \bs{A}\bs{p}_2=\bs{p}_1+2\bs{p}_2 \\
                \bs{A}\bs{p}_3=\bs{p}_3
            \end{array} 
            \right.
            \Rightarrow
            \left\{
            \begin{array}{ll}
                \bs{p}_1=(-1,4,1)^T \\
                \bs{p}_2=(-1,3,1)^T \\
                \bs{p}_3=(0,3,1)^T
            \end{array} 
            \right.
    \end{align*}
    \begin{align*}
        \bs{P}=\begin{pmatrix}
            -1 & -1 &0 \\
            4 & 3 &3 \\
            1 & 1 & 1
        \end{pmatrix} \quad
        \bs{P}^{-1}=\begin{pmatrix}
            0 & 1 &-3 \\
            -1 & -1 &3 \\
            1 & 0 & 1        
        \end{pmatrix}
    \end{align*}
    于是
    \begin{align*}
    \bs{e}^{\bs{A}t}=\bs{P}e^{\bs{J}t}\bs{P}^{-1}
    =&\begin{pmatrix}
        -1 & -1 &0 \\
        4 & 3 &3 \\
        1 & 1 & 1
    \end{pmatrix}\begin{pmatrix}
        e^{2t}& te^{2t} & 0 \\
        0 & e^{2t} & 0 \\
        0& 0 &e^{t}
    \end{pmatrix}
    \begin{pmatrix}
        0 & 1 &-3 \\
        -1 & -1 &3 \\
        1 & 0 & 1         
    \end{pmatrix}\\
    =&\begin{pmatrix}
        (t+1)e^{2t}& te^{2t} & -3te^{2t} \\
        (-4t-3)e^{2t}+3e^t & (-4t+1)e^{2t} & (12t-3)e^{2t}+3e^{t} \\
       (-t-1)e^{2t}+e^{t}& -te^{2t} &3te^{2t}+e^{t}
    \end{pmatrix}
    \end{align*}
\end{solution}


\begin{question}(p80.9)
    已知$\bs{A}=\begin{pmatrix}
        2 & 2 & 1\\
        1 &3 &1 \\
        1 &2 &2
    \end{pmatrix},\bs{B}=\begin{pmatrix}
        3 & 0 & 0 &0 \\
        0 & -2 & 1 &0 \\
        0 & 0 & -2 &1 \\
        0 & 0 & 0 &-2 \\
    \end{pmatrix}$,试求$\cos \bs{A},\sin \bs{B},e^{\bs{B}t}$。
\end{question}


\begin{solution}
    $\lambda_1=\lambda_2=1,\lambda_3=5,m_{\bs{A}}(\lambda)=(\lambda-1)(\lambda-5)$,
    令$P(\lambda)=a_0+a_1\lambda$,则:
    \begin{align*}
        \left\{
            \begin{array}{ll}
                P(\lambda_1)=P(1)=a_0+a_1=\cos 1\\
                P(\lambda_2)=P(5)=a_0+5a_1=\cos 5
            \end{array}
            \right.
            \Rightarrow
        \left\{
            \begin{array}{ll}
                a_0=\frac{-\cos 5+5\cos 1}{4}\\
                a_1=\frac{\cos 5-\cos1}{4}
            \end{array}
            \right.
    \end{align*}
    \begin{align*}
        \cos \bs{A}=P(\bs{A})
    &=\frac{-\cos 5+5\cos 1}{4}\bs{E}+\frac{\cos 5-cos1}{4}\bs{A}\\
    &=
    \frac{1}{4}\begin{pmatrix}
        \cos 5+3\cos 1 &2\cos5-2\cos 1 &\cos 5-\cos 1\\
        \cos 5- \cos 1& 2\cos 5+2\cos 1& \cos 5-\cos 1 \\
        \cos 5-\cos 1 & 2\cos 5-2 \cos 1 & \cos 5 +3\cos 1
    \end{pmatrix}
    \end{align*}

    由于$\bs{B}=
    \left ( \begin{array}{c:ccc}
        3 & 0 & 0 &0 \\
        \hdashline
        0 & -2 & 1 &0 \\
        0 & 0 & -2 &1 \\
        0 & 0 & 0 &-2 \\
    \end{array} \right )
    =\begin{pmatrix}
        \bs{J}_1 & \\
         & \bs{J}_2 
    \end{pmatrix}$,其中$\bs{J}_1,\bs{J}_2$为若当块,空白位置全是$0$。
    
    则$\sin \bs{B}=\begin{pmatrix}
        \sin \bs{J}_ 1 & \\
         & \sin \bs{J}_2
    \end{pmatrix}=\begin{pmatrix}
        \sin 3&0 &0 &0 \\
         0&-\sin 2&\cos 2&\frac{\sin 2}{2}\\
         0&0 &-\sin 2& \cos 2\\
         0& 0& 0& -\sin 2 
    \end{pmatrix}$

    则$e^{\bs{B}t}=\begin{pmatrix}
        e^{\bs{J}_1 t} & \\
         & e^{\bs{J}_2 t}
    \end{pmatrix}=\begin{pmatrix}
        e^{3t}&0 &0 &0 \\
         0&e^{-2t}&te^{-2t} &\frac{t^2e^{-2t} }{2}\\
         0&0 & e^{-2t} & te^{-2t} \\
         0& 0& 0& e^{-2t} 
    \end{pmatrix}$

\end{solution}


\begin{question}(p80.10)
    已知$\bs{A}=\begin{pmatrix}
        0 & 1\\
        0 & -2
    \end{pmatrix},\bs{B}=\begin{pmatrix}
        -2 & 1 & 1\\
        0& 2 & 0\\
        -4 & 1& 3
    \end{pmatrix}$,试求$e^{\bs{A}t},e^{\bs{B}t},\sin \bs{B}t$。
\end{question}

\begin{solution}
    $\bs{A}$的特征值为$\lambda_1=0,\lambda_2=-2,m_{\bs{A}}(\lambda)=\lambda(\lambda+2)$,
    令$P(\lambda)=a_0+a_1\lambda$,则:
    \begin{align*}
        \left\{
            \begin{array}{ll}
                P(\lambda_1)=P(0)=a_0=1\\
                P(\lambda_2)=P(-2)=a_0-2a_1=e^{-2t}
            \end{array}
            \right.
        \Rightarrow
        \left\{
        \begin{array}{ll}
            a_0=1\\
            a_1=\frac{1-e^{-2t}}{2}
        \end{array}
        \right.
    \end{align*}
    \begin{align*}
        e^{\bs{A}t}=P(\bs{A})
    &=\bs{E}+\frac{1-e^{-2t}}{2}\bs{A}\\
    &=\begin{pmatrix}
        1 &\frac{1-e^{-2t}}{2} \\
        0 & e^{-2t}
    \end{pmatrix}
    \end{align*}

    $\bs{B}$的特征值$\lambda_1=-1,\lambda_2=\lambda_3=2,m_{\bs{B}}(\lambda)=(\lambda+1)(\lambda-2)$,
    \begin{enumerate}[label=(\arabic*)]
        \item 令$P(\lambda)=a_0+a_1\lambda$,则:
        \begin{align*}
        \left\{
            \begin{array}{ll}
                P(\lambda_1)=P(-1)=a_0-a_1=e^{-t}\\
                P(\lambda_2)=P(2)=a_0+2a_1=e^{2t}
            \end{array}
            \right.
        \Rightarrow
        \left\{
            \begin{array}{ll}
                a_0=\frac{2e^{-t}+e^{2t}}{3}\\
                a_1=\frac{-e^{-t}+e^{2t}}{3}
            \end{array}
            \right.
        \end{align*}
        \begin{align*}
            e^{\bs{B}t}=P(\bs{B})
        &=\frac{2e^{-t}+e^{2t}}{3}\bs{E}+\frac{-e^{-t}+e^{2t}}{3}\bs{B}\\
        &=\frac{1}{3}\begin{pmatrix}
            -e^{2t}+4e^{-t} &e^{2t}-e^{-t} &e^{2t}-e^{-t} \\
            0 & 3e^{2t}& 0\\
            -4e^{2t}+4e^{-t}&e^{2t}-e^{-t}&4e^{2t}-e^{-t}
        \end{pmatrix}
        \end{align*}
        \item 令$P(\lambda)=a_0+a_1\lambda$,则:
        \begin{align*}
        \left\{
            \begin{array}{ll}
                P(\lambda_1)=P(-1)=a_0-a_1=\sin t\\
                P(\lambda_2)=P(2)=a_0+2a_1=\sin 2t
            \end{array}
            \right.
            \Rightarrow
            \left\{
                \begin{array}{ll}
                    a_0=\frac{\sin 2t+2\sin t}{3}\\
                    a_1=\frac{\sin 2t-\sin t}{3}
                \end{array}
                \right.
        \end{align*}
        \begin{align*}
            \sin \bs{B}t=P(\bs{B})
        &=\frac{\sin 2t+2\sin t}{3}\bs{E}+\frac{\sin 2t-\sin t}{3}\bs{B}\\
        &=\frac{1}{3}\begin{pmatrix}
            -\sin 2t +4\sin t &\sin 2t-\sin t &\sin 2t-\sin t\\
            0 & 3\sin 2t& 0\\
            -4\sin 2t+4\sin t&\sin 2t-\sin t&4\sin 2t -\sin t
        \end{pmatrix}
    \end{align*}
    \end{enumerate}
\end{solution}


\begin{question}(p80.11)
    求常系数线性齐次微分方程组
    \begin{align*}
        \left\{
        \begin{array}{ll}
            x_1'(t)=-7x_1-7x_2+5x_3\\
            x_2'(t)=-8x_1-8x_2-5x_3\\
            x_3'(t)=-5x_2
        \end{array}
        \right.
    \end{align*}
    满足初始条件$x_1(0)=3,x_2(0)=-2,x_3(0)=1$的解。
\end{question}

\begin{solution}
    由题意得:$\bs{x}'(t)=\bs{A}\bs{x}(t)$,其中$\bs{A}=\begin{pmatrix}
        -7 & -7 & 5\\
        -8 &-8& -5 \\
        0 & -5 & 0
    \end{pmatrix},\bs{x}=(x_1,x_2,x_3)^T$,矩阵$\bs{A}$的特征值为$\lambda_1=-15,\lambda_2=-5,\lambda_3=5$。

    则$\bs{A}=\bs{P}\bs{\Lambda}\bs{P}^{-1}=\begin{pmatrix}
        2 & 1& 1\\
        3 &-1 &-1 \\
        1&-1&1
    \end{pmatrix} \begin{pmatrix}
        -15 & &\\
        & -5& \\
        & & 5
    \end{pmatrix} \begin{pmatrix}
        \frac{1}{5}& \frac{1}{5}&0\\
        \frac{2}{5}&-\frac{1}{10}&-\frac{1}{2}\\
        \frac{1}{5}&-\frac{3}{10}&\frac{1}{2}
    \end{pmatrix}$

    $e^{\bs{A}t}=\bs{P}e^{\bs{\Lambda}t}\bs{P}^{-1}
    =\begin{pmatrix}
        2 & 1& 1\\
        3 &-1 &-1 \\
        1&-1&1
    \end{pmatrix} \begin{pmatrix}
        e^{-15t} & &\\
        & e^{-5t}& \\
        & & e^{5t}
    \end{pmatrix} \begin{pmatrix}
        \frac{1}{5}& \frac{1}{5}&0\\
        \frac{2}{5}&-\frac{1}{10}&-\frac{1}{2}\\
        \frac{1}{5}&-\frac{3}{10}&\frac{1}{2}
    \end{pmatrix}
    =\begin{pmatrix}
        2e^{-15t} & e^{-5t} & e^{5t}\\
        3e^{-15t} & -e^{-5t} & -e^{5t}\\
        e^{-15t} & -e^{-5t} & e^{5t}
    \end{pmatrix}$

    $\bs{x}(t)=e^{\bs{A}t}\bs{x}(0)=\begin{pmatrix}
        \frac{2}{5}e^{-15t}+\frac{9}{10}e^{-5t}+\frac{17}{10}e^{5t}\\
        \frac{3}{5}e^{-15t}-\frac{9}{10}e^{-5t}-\frac{17}{10}e^{5t}\\
        \frac{1}{5}e^{-15t}-\frac{9}{10}e^{-5t}+\frac{17}{10}e^{5t}

    \end{pmatrix}$

\end{solution}

\begin{question}(p80.12)
    求常系数线性齐次微分方程组
    \begin{align*}
        \left\{
        \begin{array}{ll}
            x_1'(t)=x_1(t)-x_2(t)\\
            x_2'(t)=4x_1(t)-3x_2(t)+1
        \end{array}
        \right.
    \end{align*}
    满足初始条件$x_1(0)=1,x_2(0)=2$的解。
\end{question}

\begin{solution}
    由题意得:$\bs{x}'(t)=\bs{A}\bs{x}(t)+\bs{F}(t)$,其中$\bs{A}=\begin{pmatrix}
        1 & -1\\
        4 &-3
    \end{pmatrix},\bs{x}=(x_1,x_2,x_3)^T,\bs{F}(t)=(0,1)^T$,
    $\bs{A}$的特征值为$\lambda_1=\lambda_2=-1,m_{\bs{A}}(\lambda)=(\lambda+1)^2$。
    令$P(\lambda)=a_0+a_1\lambda$
    ,则 \begin{align*}
        \left\{
            \begin{array}{ll}
                P(\lambda)=P(-1)=a_0-a_1=e^{-t}\\
                P'(\lambda)=P'(-1)=a_1=te^{-t}
            \end{array}
            \right.
        \Rightarrow
        \left\{
            \begin{array}{ll}
                a_0=(1+t)e^{-t}\\
                a_1=te^{-t}
            \end{array}
            \right.
    \end{align*}
    \begin{align*}
        e^{\bs{A}t}=P(\bs{A})
    &=(1+t)e^{-t}\bs{E}+te^{-t}\bs{A}\\
    &=\begin{pmatrix}
        (2t+1)e^{-t} &-te^{-t} \\
        4te^{-t}  & (-2t+1)e^{-t}
    \end{pmatrix}
    \end{align*}
    \begin{align*}
    \bs{x}(t)&=e^{\bs{A}t}\bs{x}(0)+e^{\bs{A}t}\int_{0}^t e^{-\bs{A}u}\bs{b} du\\
    &=\begin{pmatrix}
        (2t+1)e^{-t} &-te^{-t} \\
        4te^{-t}  & (-2t+1)e^{-t}
    \end{pmatrix}\begin{pmatrix}
        1 \\
        2
    \end{pmatrix}\\
    &+\begin{pmatrix}
        (2t+1)e^{-t} &-te^{-t} \\
        4te^{-t}  & (-2t+1)e^{-t}
    \end{pmatrix}\int_{0}^t 
    \begin{pmatrix}
        (-2u+1)e^{u} &ue^{u} \\
        -4ue^{u}  & (2u+1)e^{u}
    \end{pmatrix} \begin{pmatrix}
        0\\
        1
    \end{pmatrix} du \\
    &=\begin{pmatrix}
        (t+2)e^{-t}-1 \\
        (2t+3)e^{-t}-1
    \end{pmatrix}
    \end{align*}
\end{solution}

\begin{question}(p80.13)
    求微分方程组$\bs{X}'(t)=\bs{A}\bs{X}(t)$的通解,其中$\bs{A}=\begin{pmatrix}
        2 &1 &1 \\
        0 & 3 & 1\\
        0 & -1 & 1
    \end{pmatrix}$。
\end{question}


\begin{solution}
    由题意得:$\bs{X}'(t)=\bs{A}\bs{X}(t)$,其中$\bs{A}=\begin{pmatrix}
        2 & 1 & 1\\
        0 &3& 1 \\
        0 & -1 & 1
    \end{pmatrix},\bs{x}=(x_1,x_2,x_3)^T$,
    矩阵$\bs{A}$的特征值为$\lambda_1=\lambda_2=\lambda_3=2$,显然$\bs{A}$的若当标准型$\bs{J}=\begin{pmatrix}
        2 & &\\
        & 2 & 1\\
        & & 2
    \end{pmatrix}$,
    并且有$\bs{J}=\bs{P}^{-1}\bs{A}\bs{P} \Rightarrow \bs{P}\bs{J}=\bs{A}\bs{P}$。
    不妨令$\bs{P}=(\bs{p}_1,\bs{p}_2,\bs{p}_3)$,其中$\bs{P}$为非奇异矩阵,则:

    \begin{align*}
        \left\{
            \begin{array}{ll}
                \bs{A}\bs{p}_1=2\bs{p}_1 \\
                \bs{A}\bs{p}_2=2\bs{p}_2 \\
                \bs{A}\bs{p}_3=\bs{p}_2+\bs{p}_3
            \end{array} 
            \right.
            \Rightarrow
            \left\{
            \begin{array}{ll}
                \bs{p}_1=(1,0,0)^T \\
                \bs{p}_2=(2,2,-2)^T \\
                \bs{p}_3=(0,1,1)^T
            \end{array} 
            \right.
    \end{align*}
    \begin{align*}
        \bs{P}=\begin{pmatrix}
            1 & 2 &0 \\
            0 & 2 &1 \\
            0 & -2 & 1
        \end{pmatrix} \quad
        \bs{P}^{-1}=\begin{pmatrix}
            1 & -\frac{1}{2} &\frac{1}{2} \\
            0 & \frac{1}{4} &-\frac{1}{4} \\
            0 & \frac{1}{2}& \frac{1}{2}       
        \end{pmatrix}
    \end{align*}
    于是
    \begin{align*}
    e^{\bs{A}t}=\bs{P}e^{\bs{\Lambda}t}\bs{P}^{-1}
    =\begin{pmatrix}
        1 & 2 &0 \\
        0 & 2 &1 \\
        0 & -2 & 1
    \end{pmatrix} \begin{pmatrix}
        e^{2t} & &\\
        & e^{2t}&te^{2t} \\
        & & e^{2t}
    \end{pmatrix} \begin{pmatrix}
        1 & -\frac{1}{2} &\frac{1}{2} \\
        0 & \frac{1}{4} &-\frac{1}{4} \\
        0 & \frac{1}{2}& \frac{1}{2} 
    \end{pmatrix}
    =\begin{pmatrix}
        e^{2t} &te^{2t}&te^{2t} \\
        0 &  (t+1)e^{2t} &te^{2t}\\
        0 & -te^{2t} &(1-t)e^{2t}
    \end{pmatrix}
    \end{align*}
    不妨设$\bs{x}(0)=(k_1,k_2,k_3)^T$,则
    $\bs{x}(t)=e^{\bs{A}t}\bs{x}(0)=\begin{pmatrix}
        k_1e^{2t}+k_2te^{2t}+k_3te^{2t} \\
        k_2(t+1)e^{2t}+k_3te^{2t}\\
        -k_2te^{2t}+k_3(1-t)e^{2t}
    \end{pmatrix}$

\end{solution}



\begin{question}(p80.14)
    求微分方程组$\bs{X}'(t)=\bs{A}\bs{X}(t)+\bs{F}(t)$的通解,其中$\bs{A}=\begin{pmatrix}
        3 &1  \\
        1 & 3  \\
    \end{pmatrix},\bs{F}(t)=\begin{pmatrix}
        1 \\
        -1
    \end{pmatrix}$。
\end{question}

\begin{solution}
    由题意得:$\bs{x}'(t)=\bs{A}\bs{x}(t)+\bs{F}(t)$,其中$\bs{A}=\begin{pmatrix}
        3 & 1\\
        1 &3 \\
    \end{pmatrix},\bs{x}=(x_1,x_2,x_3)^T$,
    矩阵$\bs{A}$的特征值为$\lambda_1=2,\lambda_2=4,\lambda_3=5$。

    则$\bs{A}=\bs{P}\bs{\Lambda}\bs{P}^{-1}=\begin{pmatrix}
        -1 & 1\\
        1 &1  \\
    \end{pmatrix} \begin{pmatrix}
        2 & \\
        & 4
    \end{pmatrix} \begin{pmatrix}
        -\frac{1}{2}& \frac{1}{2}\\
        \frac{1}{2}&\frac{1}{2}\\
    \end{pmatrix}$

    $e^{\bs{A}t}=\bs{P}e^{\bs{\Lambda}t}\bs{P}^{-1}
    =\begin{pmatrix}
        -1 & 1\\
        1 &1  \\
    \end{pmatrix} \begin{pmatrix}
        e^{2t} & \\
        & e^{4t}
    \end{pmatrix} \begin{pmatrix}
        -\frac{1}{2}& \frac{1}{2}\\
        \frac{1}{2}&\frac{1}{2}\\
    \end{pmatrix}
    =\frac{1}{2} \begin{pmatrix}
        e^{2t}+e^{4t} & -e^{2t}+e^{4t}\\
        -e^{2t}+e^{4t}& e^{2t}+e^{4t}
    \end{pmatrix}$
    
    不妨设$\bs{x}(0)=(k_1,k_2)^T$,则:
    \begin{align*}
        \bs{x}(t)&=e^{\bs{A}t}\bs{x}(0)+e^{\bs{A}t}\int_{0}^t e^{-\bs{A}u}\bs{F}(t) du\\
        &=\frac{1}{2}\begin{pmatrix}
            e^{2t}+e^{4t} & -e^{2t}+e^{4t}\\
        -e^{2t}+e^{4t}& e^{2t}+e^{4t}
        \end{pmatrix}\begin{pmatrix}
            k_1 \\
            k_2
        \end{pmatrix}\\
        &+\frac{1}{2}\begin{pmatrix}
            e^{2t}+e^{4t} & -e^{2t}+e^{4t}\\
            -e^{2t}+e^{4t}& e^{2t}+e^{4t}
        \end{pmatrix}\int_{0}^t
        \frac{1}{2}\begin{pmatrix}
            e^{-2u}+e^{-4u} & -e^{-2u}+e^{-4u}\\
            -e^{-2u}+e^{-4u}& e^{-2u}+e^{-4u}
        \end{pmatrix} \begin{pmatrix}
            1\\
            -1
        \end{pmatrix} du \\
        &=\frac{1}{2} \begin{pmatrix}
            k_1(e^{2t}+e^{4t})+k_2(-e^{2t}+e^{4t})+e^{2t}-1 \\
            k_1(-e^{2t}+e^{4t})+k_2(e^{2t}+e^{4t})+1-e^{2t}
        \end{pmatrix}\\
        &=\frac{1}{2} \begin{pmatrix}
           (k_1+k_2)e^{4t}+(k_1-k_2+1)e^{2t}-1 \\
            (k_1+k_2)e^{4t}+(k_2-k_1-1)e^{2t}+1
        \end{pmatrix}
        \end{align*}
\end{solution}


\begin{question}(p81.16)
    设$\bs{A}$为方阵,$\bs{B}(t)=e^{\bs{A}t}$。若$\mathrm{tr}\bs{A}=0$,证明对一切$t \in \R$,$\mathrm{det}\bs{B}(t)=1$。
\end{question}

\begin{proof}
    由题意得,不妨设$\bs{A}\in \R^{n\times n}$,$\lambda_i$为矩阵$\bs{A}$的特征值,
    则$\bs{B}(t)e^{\bs{A}t}$的特征值为$e^{\lambda_i t}$,
    则
    \begin{align*}
    \mathrm{det}\bs{B}(t)=\prod\limits_{i}^ne^{\lambda_i  t}=
    e^{\sum\limits_{i}^n \lambda_i t}=e^{t \times \mathrm{tr}(\bs{A})}=e^0=1
    \end{align*}
    
\end{proof}


\ifx\allfiles\undefined
\end{document}
\fi