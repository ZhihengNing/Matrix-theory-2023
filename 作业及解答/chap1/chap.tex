\ifx\allfiles\undefined
\documentclass[12pt, a4paper, oneside, UTF8]{ctexbook}
\def\configPath{../config}
\def\basicPath{\configPath/basic}


% 在这里定义需要的包
\usepackage{amsmath}
\usepackage{amsthm}
\usepackage{amssymb}
\usepackage{graphicx}
\usepackage{mathrsfs}
\usepackage{enumitem}
\usepackage{geometry}
\usepackage[colorlinks, linkcolor=black]{hyperref}
\usepackage{stackengine}
\usepackage{yhmath}
\usepackage{extarrows}
\usepackage{arydshln}
% \usepackage{unicode-math}
\usepackage{tasks}
\usepackage{fancyhdr}
\usepackage[dvipsnames, svgnames]{xcolor}
\usepackage{listings}

\definecolor{mygreen}{rgb}{0,0.6,0}
\definecolor{mygray}{rgb}{0.5,0.5,0.5}
\definecolor{mymauve}{rgb}{0.58,0,0.82}

\graphicspath{ {figure/},{../figure/}, {config/}, {../config/} }

\linespread{1.6}

\geometry{
    top=25.4mm, 
    bottom=25.4mm, 
    left=20mm, 
    right=20mm, 
    headheight=2.17cm, 
    headsep=4mm, 
    footskip=12mm
}

\setenumerate[1]{itemsep=5pt,partopsep=0pt,parsep=\parskip,topsep=5pt}
\setitemize[1]{itemsep=5pt,partopsep=0pt,parsep=\parskip,topsep=5pt}
\setdescription{itemsep=5pt,partopsep=0pt,parsep=\parskip,topsep=5pt}

\lstset{
    language=Mathematica,
    basicstyle=\tt,
    breaklines=true,
    keywordstyle=\bfseries\color{NavyBlue}, 
    emphstyle=\bfseries\color{Rhodamine},
    commentstyle=\itshape\color{black!50!white}, 
    stringstyle=\bfseries\color{PineGreen!90!black},
    columns=flexible,
    numbers=left,
    numberstyle=\footnotesize,
    frame=tb,
    breakatwhitespace=false,
} 
% 在这里定义自己顺手的环境
\def\d{\mathrm{d}}
\def\i{\mathrm{i}}
\def\R{\mathbb{R}}
\newcommand{\bs}[1]{\boldsymbol{#1}}
\newcommand{\mc}[1]{\mathcal{#1}}
\newcommand{\ora}[1]{\overrightarrow{#1}}
\newcommand{\myspace}[1]{\par\vspace{#1\baselineskip}}
\newcommand{\xrowht}[2][0]{\addstackgap[.5\dimexpr#2\relax]{\vphantom{#1}}}
\newenvironment{ca}[1][1]{\linespread{#1} \selectfont \begin{cases}}{\end{cases}}
\newenvironment{vx}[1][1]{\linespread{#1} \selectfont \begin{vmatrix}}{\end{vmatrix}}
\newcommand{\tabincell}[2]{\begin{tabular}{@{}#1@{}}#2\end{tabular}}
\newcommand{\pll}{\kern 0.56em/\kern -0.8em /\kern 0.56em}
\newcommand{\dive}[1][F]{\mathrm{div}\;\bs{#1}}
\newcommand{\rotn}[1][A]{\mathrm{rot}\;\bs{#1}} 
\usepackage[strict]{changepage} 
\usepackage{framed}

\definecolor{greenshade}{rgb}{0.90,1,0.92}
\definecolor{redshade}{rgb}{1.00,0.88,0.88}
\definecolor{brownshade}{rgb}{0.99,0.95,0.9}
\definecolor{lilacshade}{rgb}{0.95,0.93,0.98}
\definecolor{orangeshade}{rgb}{1.00,0.88,0.82}
\definecolor{lightblueshade}{rgb}{0.8,0.92,1}
\definecolor{purple}{rgb}{0.81,0.85,1}
\theoremstyle{definition}
\newtheorem{myDefn}{\indent 定义}[section]
% \newtheorem{myLemma}{\indent 引理}[section]
\newtheorem{myLemma}{\indent 引理}[chapter]
\newtheorem{myThm}[myLemma]{\indent 定理}
\newtheorem{myCorollary}[myLemma]{\indent 推论}
\newtheorem{myCriterion}[myLemma]{\indent 准则}
\newtheorem*{myRemark}{\indent 注}
\newtheorem{myProposition}{\indent 命题}[section]


\newenvironment{formal}[2][]{%
    \def\FrameCommand{%
        \hspace{1pt}%
        {\color{#1}\vrule width 2pt}%
        {\color{#2}\vrule width 4pt}%
        \colorbox{#2}%
    }%
    \MakeFramed{\advance\hsize-\width\FrameRestore}%
    \noindent\hspace{-4.55pt}%
    \begin{adjustwidth}{}{7pt}\vspace{2pt}\vspace{2pt}}{%
        \vspace{2pt}\end{adjustwidth}\endMakeFramed%
}

\newenvironment{defn}{\begin{formal}[Green]{greenshade}\vspace{-\baselineskip / 2}\begin{myDefn}}{\end{myDefn}\end{formal}}
\newenvironment{thm}{\begin{formal}[LightSkyBlue]{lightblueshade}\vspace{-\baselineskip / 2}\begin{myThm}}{\end{myThm}\end{formal}}
\newenvironment{lemma}{\begin{formal}[Plum]{lilacshade}\vspace{-\baselineskip / 2}\begin{myLemma}}{\end{myLemma}\end{formal}}
\newenvironment{corollary}{\begin{formal}[BurlyWood]{brownshade}\vspace{-\baselineskip / 2}\begin{myCorollary}}{\end{myCorollary}\end{formal}}
\newenvironment{criterion}{\begin{formal}[DarkOrange]{orangeshade}\vspace{-\baselineskip / 2}\begin{myCriterion}}{\end{myCriterion}\end{formal}}
\newenvironment{rmk}{\begin{formal}[LightCoral]{redshade}\vspace{-\baselineskip / 2}\begin{myRemark}}{\end{myRemark}\end{formal}}
\newenvironment{proposition}{\begin{formal}[RoyalPurple]{purple}\vspace{-\baselineskip / 2}\begin{myProposition}}{\end{myProposition}\end{formal}}

\newtheorem{example}{\indent \color{SeaGreen}{例}}[section]
\newtheorem{question}{\color{SeaGreen}{题}}[chapter]
% \renewenvironment{proof}{\indent\textcolor{SkyBlue}{\textbf{证明.}}\;}{\qed\par}
% \newenvironment{solution}{\indent\textcolor{SkyBlue}{\textbf{解.}}\;}{\qed\par}

\renewcommand{\proofname}{\textbf{\textcolor{TealBlue}{证明}}}
\newenvironment{solution}{\begin{proof}[\textbf{\textcolor{TealBlue}{解}}]}{\end{proof}}

\definecolor{mygreen}{rgb}{0,0.6,0}
\definecolor{mygray}{rgb}{0.5,0.5,0.5}
\definecolor{mymauve}{rgb}{0.58,0,0.82}

\graphicspath{ {figure/},{../figure/}, {config/}, {../config/},{cover/graph} }

\linespread{1.6}

\geometry{
    top=25.4mm, 
    bottom=25.4mm, 
    left=20mm, 
    right=20mm, 
    headheight=2.17cm, 
    headsep=4mm, 
    footskip=12mm
}

\setenumerate[1]{itemsep=5pt,partopsep=0pt,parsep=\parskip,topsep=5pt}
\setitemize[1]{itemsep=5pt,partopsep=0pt,parsep=\parskip,topsep=5pt}
\setdescription{itemsep=5pt,partopsep=0pt,parsep=\parskip,topsep=5pt}

\lstset{
    language=Mathematica,
    basicstyle=\tt,
    breaklines=true,
    keywordstyle=\bfseries\color{NavyBlue}, 
    emphstyle=\bfseries\color{Rhodamine},
    commentstyle=\itshape\color{black!50!white}, 
    stringstyle=\bfseries\color{PineGreen!90!black},
    columns=flexible,
    numbers=left,
    numberstyle=\footnotesize,
    frame=tb,
    breakatwhitespace=false,
} 

\begin{document}
\else
\fi

\chapter{线性空间与线性变换}
\begin{question} 
    (P98.3)在$\mathbb{R}^3$中,取
    \begin{align*}
    \bs{F}_1=\begin{pmatrix}
        1\\
        0\\
        0
    \end{pmatrix},
    \bs{F}_2=\begin{pmatrix}
        1\\
        1\\
        0
    \end{pmatrix},
    \bs{F}_3=\begin{pmatrix}
        1\\
        1\\
        1
    \end{pmatrix}
\end{align*}
    \begin{enumerate}[label=(\arabic*)]
        \item 证明:$\bs{F}_1,\bs{F}_2,\bs{F}_3$构成${\R}^3$的一组基。
        \item 已知$\R^3$中元素$\bs{A}$在基$\bs{F}_1,\bs{F}_2,\bs{F}_3$下的坐标为$(1,2,3)^T$,求$\bs{A}$。
        \item 求$\bs{B}=(1,2,3)^T$在基$\bs{F}_1,\bs{F}_2,\bs{F}_3$下的坐标。
    \end{enumerate}
\end{question}

\begin{solution}
    \begin{enumerate}[label=(\arabic*)]
        \item 令$\bs{F}=(\bs{F}_1,\bs{F}_2,\bs{F}_3)$,此时$|\bs{F}|=\begin{vx}
        1& 1& 1 \\
        0& 1 & 1 \\
        0 & 0 & 1
        \end{vx}=1\neq 0$,这构成了$\R^3$下的一组基。
        \item $\bs{A}=\bs{F}\bs{x}=(6,5,3)^T$,其中$\bs{x}=(1,2,3)^T$。
        \item $\bs{B}=\bs{F}\bs{x}$,则$\bs{x}=\bs{F}^{-1}\bs{B}=(-1,-1,3)^T$。
    \end{enumerate}
\end{solution}

\begin{question}
    (p98.4)验证
    \begin{align*}
        \bs{\nu}_1=\begin{pmatrix}
            1 \\
            2\\
            1
        \end{pmatrix},\bs{\nu}_2=\begin{pmatrix}
            2\\
            3\\
            3
        \end{pmatrix},\bs{\nu}_3=\begin{pmatrix}
            3\\
            7\\
            10
        \end{pmatrix}
    \end{align*}
    与
    \begin{align*}
        \bs{\omega}_1=\begin{pmatrix}
            3 \\
            1\\
            4
        \end{pmatrix},\bs{\omega}_2=\begin{pmatrix}
            5\\
            2\\
            1
        \end{pmatrix},\bs{\omega}_3=\begin{pmatrix}
            1\\
            1\\
            -6
        \end{pmatrix}
    \end{align*}
    都可作为$\R^3$的基,并求$\bs{\nu}_1,\bs{\nu}_2,\bs{\nu}_3$到$\bs{\omega}_1,\bs{\omega}_2,\bs{\omega}_3$的过渡矩阵。
\end{question}

\begin{solution}
令$\bs{V}=(\bs{\nu}_1,\bs{\nu}_2,\bs{\nu}_3)$,$\bs{W}=(\bs{\omega}_1,\bs{\omega}_2,\bs{\omega}_3)$,
此时有$|\bs{V}|=\begin{vx}
    1&2&3\\
    2&3&7\\
    1&3&10
\end{vx} \neq 0$,$|\bs{W}|=\begin{vx}
    3&5&1\\
    1&2&1\\
    4&1&-6
\end{vx} \neq 0$,这说明了$\bs{\nu}_1,\bs{\nu}_2,\bs{\nu}_3$与$\bs{\omega}_1,\bs{\omega}_2,\bs{\omega}_3$都可作为$\R^3$的一组基。

记过渡矩阵为$\bs{A}$,则$\bs{W}=\bs{V}\bs{A}$,则$\bs{A}=\bs{V}^{-1}\bs{W}$,$\bs{A}=\begin{pmatrix}
    -\frac{9}{2}& -\frac{7}{2}&4 \\
    \frac{9}{2}& \frac{13}{2}& 0 \\
    -\frac{1}{2} &-\frac{3}{2} &-1
\end{pmatrix}$。

\end{solution}

\begin{question}
    \label{no zhihe}
    (P99.10)设$\bs{V}_1,\bs{V}_2,\bs{V}_3$为线性空间$\bs{V}$的子空间,
    且$\bs{V}_1 \cap \bs{V}_2 \cap \bs{V}_3=\{\bs{0}\}$,试问$\bs{V}_1+\bs{V}_2+\bs{V}_3$是否为直和?    
\end{question}

\begin{proof}
    \textbf{结论}:不构成直和。下面通过举反例给出证明:

    取线性空间$\bs{V}=\R^3$,并令$\bs{e}_1,\bs{e}_2,\bs{e}_3$分别为$\bs{V}$
    子空间$\bs{V}_1,\bs{V}_2,\bs{V}_3$上的一组基,其中$\bs{V}_i=\{k\bs{e}_i|k\in\R,i=1,2,3\}$
    ,$(\bs{e}_1,\bs{e}_2,\bs{e}_3)=\begin{pmatrix}
        1 & -1 &1 \\
        1 & 1 &2 \\
        0 & 0 & 0
    \end{pmatrix}$。考察$\bs{V}_i$的几何意义,为同一二维平面的一条直线,
    则$\mathrm{dim}(\bs{V}_i)=1$。
    
    容易验证子空间满足$\bs{V}_1 \cap \bs{V}_2 \cap \bs{V}_3=\{\bs{0}\}$,
    并且$\bs{V}_1+\bs{V}_2+\bs{V}_3=\{\sum\limits_{i=1}^n\bs{\alpha}_i|\bs{\alpha}_i \in \bs{V}_i\}
    =\{k_i\bs{e}_i|k_i \in \R \}=(k_1-k_2+k_3,k_1+k_2+2k_3,0)^T=\R^2$,
    所以$\mathrm{dim}(\bs{V}_1+\bs{V}_2+\bs{V}_3)=\mathrm{r}(\bs{e}_1,\bs{e}_2,\bs{e}_3)=2$。
    此时$\mathrm{dim}(\bs{V}_1)+\mathrm{dim}(\bs{V}_2)+\mathrm{dim}(\bs{V}_3)=1+1+1=3 \neq \mathrm{dim}(\bs{V}_1+\bs{V}_2+\bs{V}_3)$,这便说明了
    $\bs{V}_1+\bs{V}_2+\bs{V}_3$不构成直和。

\end{proof}

\begin{question}
    (P99.12)设$\bs{V}_1,\bs{V}_2,\ldots,\bs{V}_n$为线性空间$\bs{V}$的子空间,举例说明,
    即使$\bs{V}_1,\bs{V}_2,\ldots,\bs{V}_n$两两的交空间均为零空间,其和$\bs{V}_1+\bs{V}_2+\cdots+\bs{V}_n$
    也未必是直和。
\end{question}

\begin{proof}
    与例\ref{no zhihe}有类似的证明过程,不妨令$\bs{V}=\R^n$,则$\mathrm{dim}(\bs{V})=n$
    ,取$n$个向量$\bs{e}_1,\cdots,\bs{e}_n$,令$\bs{V}_i=\{k_i\bs{e}_i|k_i\in\R,i=1,\ldots,n\}$,
    其中
    $\bs{e}_1=(1,0,\ldots,0)^T$,
    $\bs{e}_2=(\cos \frac{\pi}{n},\sin \frac{\pi}{n},0,\ldots,0)^T$,
    $\bs{e}_k=(\cos \frac{k\pi}{n},\sin \frac{\pi}{n},0,\ldots,0)^T$,
    $\bs{e}_n=(\cos \frac{(n-1)\pi}{n},\sin \frac{(n-1)\pi}{n},0,\ldots,0)^T$,
    并且$||\bs{e}_i||_2^2=1$。考察$\bs{V}_i$的几何意义,为同一二维平面上的一条直线,则$\mathrm{dim}(\bs{V}_i)=1$。

    于是$\sum\limits_{i=1}^n \bs{V}_i=\{\sum\limits_{i=1}^n \bs{\alpha}_i|\bs{\alpha}_i \in \bs{V}_i\}=
    \{\sum\limits_{i=1}^n k_i\bs{e}_i|k_i \in \R\}=\R^2$,
    所以$\mathrm{dim}(\sum\limits_{i=1}^n \bs{V}_i)=2$。而$\sum\limits_{i=1}^n \mathrm{dim}(\bs{V}_i)=n \neq \mathrm{dim}(\sum\limits_{i=1}^n \bs{V}_i)$,
    这便说明了不构成直和。
    
    下面验证$\bs{V}_1$,$\bs{V}_2$,$\ldots$,$\bs{V}_n$两两的交空间均为零空间。
    任取$\bs{V}_i=\{k_i\bs{e}_i\}$,$\bs{V}_j=\{k_j\bs{e}_j\}$,其中$i < j$。要验证其交空间为零空间,
    只需验证前两个维度的交为0。即满足如下等式:
    \begin{align}
    k_i\cos \frac{(i-1)\pi}{n}=k_j \cos \frac{(j-1)\pi}{n} \label{(1)} \\
    k_i\sin \frac{(i-1)\pi}{n}=k_j \sin \frac{(j-1)\pi}{n} \label{(2)}
    \end{align}
\begin{itemize}
    \item 若$i=1$,根据式\eqref{(2)}得$k_j=0$,根据式\eqref{(1)}得$k_i=0$。
    \item 若$i>1$,$k_i=0$,
    由于$\frac{(i-1)\pi}{n} \in (0,\pi)$,
    则$\sin x \in (0,\pi)$,
    根据式\eqref{(2)}得$k_j=0$。同理:若$k_j=0$,则$k_i=0$。
    \item 若$i>1$,$i=1+\frac{n}{2}$,
    根据式\eqref{(1)}得$k_j=0$,根据式\eqref{(2)}得$k_i=0$。
    同理:若$j=1+\frac{n}{2}$,则$k_i=k_j=0$。
    \item 若$i>1$,且$k_i\neq 0$,$k_j \neq 0$,
    $i\neq 1+\frac{n}{2}$,$j \neq 1+\frac{n}{2}$,
    此时$k_i\cos \frac{(i-1)\pi}{n} \neq 0$,$k_j\cos \frac{(i-1)\pi}{n} \neq 0$,
    用式\eqref{(2)}除以式\eqref{(1)},
    得$\tan \frac{(i-1)\pi}{n}=\tan \frac{(i-1)\pi}{n}$,此时$i= j$,这与$i \neq j$矛盾。
\end{itemize}
综上$k_1=k_2=0$,$\bs{V}_i \cap \bs{V}_j =\{\bs{0}\}$,这便完成了证明。

\end{proof}


\begin{question}
    (P100.14)考虑关于函数的集合$\bs{V}=\{(a_2x^2+a_1x+a_0)e^x:a_0,a_1,a_2 \in \R\}$。
   \begin{enumerate}[label=(\arabic*)]
    \item 证明该集合关于函数的线性运算构成3维实线性空间。
    \item 证明求导算子$\mathcal{D}: f \rightarrow f'$为$\bs{V}$上的线性变换,并给出$\mathcal{D}$
    在基$\alpha_1=x^2e^x$,$\alpha_2=xe^x$,$\alpha_3=e^x$下的矩阵。
\end{enumerate}
\end{question}

\begin{solution}
    \begin{enumerate}[label=(\arabic*)]
        \item 由于函数的本质是$\R^3 \to \bs{V}$的映射,其中$\bs{V} \subset \R$。
        所以$\bs{V}$显然满足加法和乘法的八条运算法则。
        接下来一一验证八条法则:

        任取$a,b,c \in \bs{V}$,$k,l \in \R$,令$a=(a_2x^2+a_1x+a_0)e^x$,$a=(b_2x^2+b_1x+b_0)e^x$,$c=(c_2x^2+c_1x+c_0)e^x$。
        
        加法:

        $a+b=b+a$;$(a+b)+c=a+(b+c)$;
        $a+0=a \Leftrightarrow 0=(0x^2+0x+0)e^x$;
        $a+b=0 \Leftrightarrow b=(-a_2x^2-a_1x-a_0)e^x$。
        
        乘法:

        $k(a+b)=ka+kb$;
        $(k+l)a=ka+la$;
        $(kl)a=k(la)$;
        $1a=a$。
        
        \item 只需验证其对加法和乘法封闭,
        任取$a,b \in \bs{V}$,$k \in \R$,
        $\mc{D}(a)=(a_2x^2+(2a_2+a_1)x+a_1+a_0)e^x$,$\mc{D}(b)=(b_2x^2+(2b_2+b_1)x+b_1+b_0)e^x$。
        则$\mc{D}(a)+\mc{D}(b)=((a_2+b_2)x^2+(2a_2+a_1+2b_2+b_1)x+a_1+a_0+b_1+b_0)e^x=\mc{D}(a+b)$,
        $\mc{D}(ka)=k(a_2x^2+(2a_2+a_1)x+a_1+a_0)e^x=k\mc{D}(a)$。

        接下来计算基经过线性变换后的结果,即$\mc{D}(\alpha_1)=(x^2+2x)e^x$,$\mc{D}(\alpha_2)=(x+1)e^x$,$\mc{D}(\alpha_3)=e^x$。
        所以$\mc{D}(\alpha_1,\alpha_2,\alpha_3)=(\alpha_1,\alpha_2,\alpha_3)\begin{pmatrix}
            1 & 0 & 0\\
            2 & 1 & 0 \\
            0 & 1 & 1
        \end{pmatrix}$。
    \end{enumerate}
\end{solution}

\begin{question}
(p100.17)设$\mc{A}$是线性空间$\bs{V}$上的线性变换,且$\mathrm{Im} \mc{A}^2=\mathrm{Im} \mc{A}$,则$\mc{A}^2=\mc{A}$是否成立?说明理由。
\end{question}

\begin{solution}
\textbf{结论}:不一定成立。下面通过举反例说明,即若$\mc{A}^2=n\mc{A}$,$\mathrm{Im}\mc{A}^2=\mathrm{Im}\mc{A}$仍然成立。

先证$\mathrm{Im}\mc{A}^2 \subset \mathrm{Im}\mc{A}$,任取$\bs{\alpha} \in \bs{V}$,有$\mc{A}\bs{\alpha} \in \bs{V}$,
即$\mathrm{Im}\mc{A} \subset \bs{V}$,于是有$\mc{A}(\mathrm{Im}\mc{A}) \subset \mc{A}(\bs{V})$
即$\mathrm{Im}\mc{A}^2 \subset \mathrm{Im}\mc{A}$;再证$\mathrm{Im}\mc{A} \subset \mathrm{Im}\mc{A}^2$,
任取$\bs{\alpha} \in \bs{V}$,若存在$\bs{\beta}$,使$\mc{A}^2(\bs{\beta})=\mc{A}(\bs{\alpha})$,则$\mathrm{Im}\mc{A} \subset \mathrm{Im}\mc{A}^2$。因为$\mc{A}(\bs{\alpha})=\frac{1}{n}\mc{A}^2(\bs{\alpha})=\mc{A}^2(\frac{1}{n}\bs{\alpha})$,
令$\frac{1}{n}\bs{\alpha}=\bs{\beta}$便说明了$\mathrm{Im}\mc{A} \subset \mathrm{Im}\mc{A}^2$。综上,$\mathrm{Im}\mc{A}^2=\mathrm{Im}\mc{A}$。

\end{solution}

\begin{question}(p100.18)
    设$\mc{A}$是线性空间$\bs{V}$上的线性变换,且$\bs{V}=\mathrm{ker}\mc{A} \oplus  \mathrm{Im} \mc{A}$,
    证明$\mathrm{Im}\mc{A}^2=\mathrm{Im}\mc{A}$。举例说明一般情况下$\mathrm{ker}\mc{A}$和$\mathrm{Im}\mc{A}$不构成直和关系?
    
\end{question}

\begin{proof}
    给出如下两种解法:
    \begin{enumerate}[label=(\arabic*)]
        \item 直接利用题目条件证明
    \begin{align*}
        \mathrm{Im}(\mc{A})=\mathrm{Im}(\bs{V})=\mathrm{Im}(\mathrm{ker}\mc{A} \oplus  \mathrm{Im} \mc{A})=\mathrm{Im}(\mathrm{ker}\mc{A})+\mathrm{Im}(\mathrm{Im}\mc{A})
        =\bs{0}+\mathrm{Im}^2\mc{A}=\mathrm{Im}^2\mc{A}
    \end{align*}
    \item 先证$\mathrm{Im}\mc{A}^2 \subset \mathrm{Im}\mc{A}$,任取$\bs{\alpha} \in \bs{V}$,有$\mc{A}\bs{\alpha} \in \bs{V}$,
    即$\mathrm{Im}\mc{A} \subset \bs{V}$,于是有$\mc{A}(\mathrm{Im}\mc{A}) \subset \mc{A}(\bs{V})$
    即$\mathrm{Im}\mc{A}^2 \subset \mathrm{Im}\mc{A}$;
    再证$\mathrm{Im}\mc{A} \subset \mathrm{Im}\mc{A}^2$,
    任取$\bs{\alpha} \in \bs{V}$,一定存在$\bs{\beta},\bs{\gamma}$,
    使$\bs{\alpha}=\bs{\beta}+\bs{\gamma}$,其中$\mc{A}(\bs{\beta})=\bs{0},\bs{\gamma}=\mc{A}(\eta) \in \mathrm{Im}(\mc{A})$,
    \begin{align*}
        \mc{A}(\bs{\alpha})=\mc{A}(\bs{\beta}+\bs{\gamma})=
        \mc{A}(\bs{\beta})+\mc{A}(\bs{\gamma})=\mc{A}(\bs{\gamma})=\mc{A}(\mc{A}(\bs{\eta}))
        =\mc{A}^2(\eta)
    \end{align*}
    则$\mathrm{Im}\mc{A} \subset \mathrm{Im}\mc{A}^2$。综上,$\mathrm{Im}(\mc{A}^2)=\mathrm{Im}(\mc{A})$。
    
\end{enumerate}
\end{proof}

\begin{question}(p100.19)
定义映射$\mc{T}:\R^{2 \times 2} \rightarrow \R^{2\times 2}$为
\begin{align*}
    \mc{T}(\bs{A})=\begin{pmatrix}
        1 & 2\\
        0 & 0
    \end{pmatrix} \bs{A}
\end{align*}
\begin{enumerate}[label=(\arabic*)]
    \item 证明:$\mc{T}$是$\R^{2 \times 2}$上的线性变换。
    \item 求$\mc{T}$在基
    \begin{align*}
        \bs{E}_1=\begin{pmatrix}
            1  & 0\\
            0 & 0
        \end{pmatrix},\bs{E}_2=\begin{pmatrix}
            1  & 0\\
            0 & 1
        \end{pmatrix},\bs{E}_3=\begin{pmatrix}
            0  & 1\\
            1 & 0
        \end{pmatrix},\bs{E}_4=\begin{pmatrix}
            0  & 1\\
            -1 & 0
        \end{pmatrix}
    \end{align*}下的矩阵。
    \item 已知$\R^{2 \times 2}$中元素$\bs{A}$在基$\bs{E}_1,\bs{E}_2,\bs{E}_3,\bs{E}_4$下的坐标为$(1,2,3,4)^T$,求$\mc{T}(\bs{A})$。
    \item 求$\mathrm{ker}\mc{T}$和$\mathrm{Im}\mc{T}$。
    \item 求$\mc{T}$的不变因子和最小多项式。
    \item 是否存在一组基,使得$\mc{T}$在这组基下的矩阵为对角矩阵?如存在,求出这组基和相应的对角阵。
\end{enumerate}
\end{question}

\begin{solution}
    \begin{enumerate}[label=(\arabic*)]
        \item 任取$\bs{A},\bs{B} \in \R^{2 \times 2}$,$k \in \R$,
        显然$\mc{T}(\bs{A}+\bs{B})=\mc{T}(\bs{A})+\mc{T}(\bs{A})$,$k\mc{T}(\bs{A})=\mc{T}(k\bs{A})$,这便说明了$\mc{T}$是$\R^{2 \times 2}$上的线性变换。
        \item 计算基经过线性变换后的结果,即
        \begin{align*}
            \mc{T}(\bs{E}_1)=\begin{pmatrix}
                1  & 0\\
                0 & 0
            \end{pmatrix},\mc{T}(\bs{E}_2)=\begin{pmatrix}
                1  & 2\\
                0 & 0
            \end{pmatrix},\mc{T}(\bs{E}_3)=\begin{pmatrix}
                2  & 1\\
                0 & 0
            \end{pmatrix},\mc{T}(\bs{E}_4)=\begin{pmatrix}
                -2  & 1\\
                0 & 0
            \end{pmatrix}
        \end{align*}
        所以$\mc{T}(\bs{E}_1)=\bs{E}_1$,
        $\mc{T}(\bs{E}_2)=\bs{E}_1+\bs{E}_3+\bs{E}_4$,
        $\mc{T}(\bs{E}_3)=2\bs{E}_1+\frac{1}{2}\bs{E}_3+\frac{1}{2}\bs{E}_4$,
        $\mc{T}(\bs{E}_4)=-2\bs{E}_1+\frac{1}{2}\bs{E}_3+\frac{1}{2}\bs{E}_4$,
        于是
        \begin{align*}
            \mc{T}(\bs{E}_1,\bs{E}_2,\bs{E}_3,\bs{E}_4)=(\bs{E}_1,\bs{E}_2,\bs{E}_3,\bs{E}_4) 
            \begin{pmatrix}
                1 & 1 & 2& -2 \\
                0 & 0& 0& 0 \\
                0& 1 & \frac{1}{2} & \frac{1}{2} \\
                0& 1 & \frac{1}{2} & \frac{1}{2}
            \end{pmatrix}
        \end{align*}
        记$\bs{C}=\begin{pmatrix}
            1 & 1 & 2& -2 \\
            0 & 0& 0& 0 \\
            0& 1 & \frac{1}{2} & \frac{1}{2} \\
            0& 1 & \frac{1}{2} & \frac{1}{2}
        \end{pmatrix}$,则$\bs{C}$为所求。
        \item \begin{align*}
        \mc{T}(\bs{A})=&\mc{T}[(\bs{E}_1,\bs{E}_2,\bs{E}_3,\bs{E}_4)\bs{x}] \\
        =&\mc{T}(\bs{E}_1,\bs{E}_2,\bs{E}_3,\bs{E}_4)\bs{x} \\
        =&\bs{E}_1+\frac{11}{2}\bs{E}_3+\frac{11}{2}\bs{E}_4 \\
        =& \begin{pmatrix}
            1 &11 \\
            0 &0 
        \end{pmatrix}
        \end{align*}
        其中$\bs{x}=(1,2,3,4)^T$。
        \item 将$\bs{C}$化为行阶梯形式矩阵$\begin{pmatrix}
            1 & 1 & 2& -2 \\
            0 & 0& 0& 0 \\
            0& 1 & \frac{1}{2} & \frac{1}{2} \\
            0& 0 & 0 & 0
        \end{pmatrix}$,可知$\mathrm{r}(\bs{C})=2$,则选取第一和第二列作为极大无关组,
        则$\mathrm{Im}\mc{T}=(\bs{E_1},\bs{E_1}+\bs{E}_3+\bs{E}_4)$;由于$\mathrm{ker}(\mc{T})$为矩阵$\bs{C}$的化零空间,
        则$\mathrm{ker}(\mc{T})=(-3\bs{E_1}-\bs{E}_2+2\bs{E}_3,5\bs{E_1}-\bs{E}_2+2\bs{E}_4)$。
        \item
        矩阵$\bs{C}$的特征值为$\lambda_1=\lambda_2=0,\lambda_3=\lambda_4=1$;
        
        行列式因子为$D_1=1,D_2=1,D_3=\lambda(\lambda-1),D_4=\lambda^2(\lambda-1)^2$;
        
        则不变因子为$d_1=1,d_2=1,d_3=\lambda(\lambda-1),d_3=\lambda(\lambda-1)$;
        
        最小多项式为$m_{\bs{A}}(\lambda)=\lambda(\lambda-1)$。
        \item 一定存在,这是由于最小多项式不同项的最高系数为1。
        
        则其若当标准型$\bs{J}=\begin{pmatrix}
            0 & 0 & 0& 0 \\
            0 & 0& 0& 0 \\
            0& 0 & 1 & 0 \\
            0& 0 & 0 & 1
        \end{pmatrix}$。

        求出属于$\lambda=0$的特征向量$\bs{p}_1=(-3,-1,2,0)^T$,$\bs{p}_2=(5,-1,0,2)^T$。
        
        求出属于$\lambda=1$的特征向量$\bs{p}_3=(1,0,0,0)^T$,$\bs{p}_4=(0,0,1,1)^T$。
        
        接着设新基底为$(\bs{Y}_1,\bs{Y}_2,\bs{Y}_3,\bs{Y}_4)$,
        令$\bs{P}=(\bs{p}_1,\bs{p}_2,\bs{p}_3,\bs{p}_4)$,
        则$(\bs{Y}_1,\bs{Y}_2,\bs{Y}_3,\bs{Y}_4)=(\bs{E}_1,\bs{E}_2,\bs{E}_3,\bs{E}_4)\bs{P}
        =(-3\bs{E_1}-\bs{E}_2+2\bs{E}_3,5\bs{E_1}-\bs{E}_2+2\bs{E}_4,\bs{E}_1,\bs{E}_3+\bs{E}_4)$。
    \end{enumerate}
\end{solution}

\begin{question}(p100.21)
复数集$\mathbb{C}$上的共轭变换$\bs{z} \rightarrow \bar{\bs{z}}$是否是$\mathbb{C}$作为复线性空间上的线性变换?
是否是$\mathbb{C}$作为实线性空间上的
线性变换?
\end{question}

\begin{solution}
    定义变换$\mc{T}(z)=\bar{z}$。
    任取$z_1=a+bi,z_2=c+di \in \mathbb{C}$,
    $k=k_1+k_2i \in \mathbb{C}$,其中$a,b,c,d,k_1,k_2 \in \R$。

    若选取的是复线性空间。$\mc{T}(z_1+z_2)=
    \overline{z_1+z_2}=\overline{(a+c)+(b+d)i}=
    (a-bi)+(c-di)=\bar{z}_1+\bar{z}_2=\mc{T}(z_1)+\mc{T}(z_2)$,对加法封闭;
    $\mc{T}(kz_1)=(k_1a-k_2b)+(-k_1b-k_2a)i $,
    $k\mc{T}(z_1)=(k_1+k_2i)(a-bi)=(k_1a+k_2b)+(-k_1b+k_2a)i$,
    $\mc{T}(kz_1) \neq k\mc{T}(z_1)$,
    对数乘不封闭。于是在复线性空间上不构成线性变换。
    
    若选取的是实线性空间。加法封闭同上,下面验证数乘封闭,此时$k_2=0$,
    $\mc{T}(kz_1)=ka-k_1b=k\mc{T}(z_1)$。于是在实线性空间上构成线性变换。
\end{solution}

\begin{question}(p100.22)
设矩阵$\bs{A}$可以相似对角化,证明:$\bs{A}$可以表示为矩阵$\bs{P}_1,\ldots,\bs{P}_n$的线性组合,
其中$\bs{P}_1,\ldots,\bs{P}_n$满足
\begin{enumerate}[label=(\arabic*)]
    \item 对一切$i$,有$\bs{P}_i^2=\bs{P}_i$
    \item 对一切$i \neq j$,有$\bs{P}_i\bs{P}_j=\bs{O}$
    \item $\bs{E}=\bs{P_1}+\cdots+\bs{P}_n$
\end{enumerate}
给出具体的构造方法,并讨论该分解的唯一性。
\end{question}

\begin{solution}
    不妨令$\bs{A}\in \R^{n\times n}$,
    由于矩阵$\bs{A}$可相似对角化,则存在$\bs{Q}$使得
    $\bs{Q}^{-1}\bs{A}\bs{Q}=\bs{\Lambda}=\\
    \mathrm{diag}\{\lambda_1,\ldots,\lambda_n\}$。
    设$\bs{E}_{ii}=\begin{pmatrix}
        0& \cdots &0 \\
        \vdots & 1 & \vdots \\
        0 & \cdots & 0
    \end{pmatrix}$,这表明了矩阵第$i$行第$i$列为$1$,其他元素全为$0$。
    则$\bs{\Lambda}=\sum\limits_{i=1}^n \lambda_i \bs{E}_{ii}$,
    $\bs{A}=\sum\limits_{i=1}^n\lambda_i \bs{Q}\bs{E}_{ii}\bs{Q}^{-1}$,
    令$\bs{P}_i=\bs{Q}\bs{E}_{ii}\bs{Q}^{-1}$,
    则$\bs{A}$可以表示为矩阵$\bs{P}_1,\ldots,\bs{P}_n$的线性组合,
    并且容易验证$\bs{P}_{i},\ldots,\bs{P}_{n}$满足三个约束条件。

    唯一性的证明不会。
    % 下面验证该分解的唯一性,假设存在其他符合题意的分解,记$\bs{A}=\sum\limits_{i=1}^n \nu_i\bs{H}_i$。
    % 则$\bs{A}\bs{H}_j=\sum\limits_{i=1}^n \nu_i \bs{H}_i\bs{H}_j= \nu_i \bs{H}_j^2=\nu_i \bs{H}_j$。

    % 若$\bs{H}_j=\bs{O}$,说明了矩阵$\bs{H}_j$在线性组合中是不作贡献的,

    % 若$\bs{H}_j\neq\bs{O}$,对其进行列分块$(\bs{h}_1,\ldots,\bs{h}_n)$,则有$\bs{H}_j\bs{h}_k=\nu_i \bs{h}_k$。
    % 若$\bs{h}_k \neq \bs{0}$,则
    % $\nu_i$是$\bs{H}_j$的特征值,$\bs{h}_k$是对应的特征向量。于是$\bs{H}_j$可表示为$\begin{pmatrix}
    %     \bs{h}
    % \end{pmatrix}$

    % 由于$\bs{H}_j^2=\bs{H}_j$,则$\bs{H}_j$必可相似对角化,
    % 并且其特征值为$0,1$,于是$\bs{H}_j=\bs{P}_j\begin{pmatrix}
    %     1 & & & & \\
    %     & \ddots& & & \\
    %     & & 0 & &\\
    %     & & & \ddots &\\
    %     & & & & 0
    % \end{pmatrix}$
\end{solution}

\begin{question}(p100.23)
已知$\mc{A}$为线性空间$\bs{V}$上的线性变换,$\bs{\nu}\in \bs{V}$,$k \geq 1$ 为正整数,满足$\mc{A}^k\bs{\nu}=\bs{0}$,且$\mc{A}^{k-1} \bs{\nu} \neq \bs{0}$。
\begin{enumerate}[label=(\arabic*)]
    \item 证明:$\bs{\nu},\mc{A}\bs{\nu},\ldots,\mc{A}^{k-1}\bs{\nu}$线性无关,特别$k \leq \mathrm{dim}\bs{V}$。
    \item 证明:$\bs{W}=\mathrm{span}\{\bs{\nu},\mc{A}\bs{\nu},\ldots,\mc{A}^{k-1}\bs{\nu}\}$为$\mc{A}$的不变子空间。
    \item 求$\mc{A}$在$\bs{W}$上的限制$\mc{A}|_{\bs{W}}$在基$\bs{\nu},\mc{A}\bs{\nu},\ldots,\mc{A}^{k-1}\bs{\nu}$下的矩阵。
\end{enumerate}

\end{question}

\begin{solution}
    \begin{enumerate}[label=(\arabic*)]
        \item 只需验证对于实数$l_1,\ldots,l_k$,当$l_1\bs{\nu}+l_2 \mc{A}\bs{\nu}+\cdots+l_k\mc{A}^{k-1}\bs{\nu} = \bs{0}$时,有$l_1=\cdots=l_k=0$。
        对等式两边进行$\mc{A}^m \ (m=k-1,\ldots,1)$的线性变换, 由于$\mc{A}^k\bs{\nu}=\bs{0}$,则$\mc{A}^{k+d}\bs{\nu}=\bs{0}$,其中$d \geq 0$。
        \begin{align*}
            &l_1\mc{A}^{k-1}\bs{\nu}+l_2\mc{A}^{k}\bs{\nu}+\cdots+l_k\mc{A}^{2k-2}\bs{\nu}=\bs{0}  \tag{1} \label{式1}\\
            &l_1\mc{A}^{k-2}\bs{\nu}+l_2\mc{A}^{k-1}\bs{\nu}+\cdots+l_k\mc{A}^{2k-1}\bs{\nu}=\bs{0}  \tag{2} \label{式2}\\
            &\vdots \notag \\
            &l_1\mc{A}\bs{\nu}+l_2\mc{A}^{2}\bs{\nu}+\cdots+l_k\mc{A}^{k}\bs{\nu}=\bs{0}  \tag{n} \label{式n}
        \end{align*}
        由式\eqref{式1}可知,$l_1\mc{A}^{k-1}\bs{\nu}=\bs{0}$,而$\mc{A}^{k-1}\bs{\nu} \neq \bs{0}$,所以$l_1=0$。
        同理,观察式\eqref{式2} 到式\eqref{式n},
        可得$l_2=0,\ldots,l_n=0$,于是$l_1=\cdots=l_n=0$。
        \item 验证$\bs{W}$的基$(\bs{e}_1,\ldots,\bs{e}_k)$经过线性变换后仍在$\bs{W}$中即可。$\mc{A}(\bs{e}_1)=\mc{A}\bs{\nu},\ldots,\mc{A}(\bs{e}_k)=\mc{A}^{k}\bs{\nu}=\bs{0}$,
        这表明了$\mc{A}(\bs{e}_i)=\bs{e}_{i+1} \in \bs{W}\ (i=1,\ldots,k-1)$,$\mc{A}(\bs{e}_k)=\bs{0} \in \bs{W}$。\label{100.23(2)}
        \item 根据第\ref{100.23(2)}问,$\mc{A}(\bs{e}_1,\ldots,\bs{e}_k)=(\bs{e}_1,\ldots,\bs{e}_k)\begin{pmatrix}
            0 & 0 &\cdots& 0 & 0\\
            1 & 0 & \cdots& 0 & 0 \\
            0 & 1 &\cdots  & 0 & 0  \\
            \vdots& \vdots& \ddots &\vdots & \vdots \\
            0 & 0 & \cdots & 1 & 0
        \end{pmatrix}$。
    \end{enumerate}
\end{solution}

\begin{question}(p100.24)
已知$\mc{A}$为线性空间$\bs{V}$上的线性变换,$\bs{\alpha}_i \in \bs{V}$,$k_i \geq 1$为正整数,其中$i=1,2,\ldots,s$,
满足$(\mc{A}-\lambda_i \mathrm{id})^{k_i}\bs{\nu}_i = \bs{0}$且$(\mc{A}-\lambda_i \mathrm{id})^{k_i-1}\bs{\nu}_i \neq \bs{0}$,并记
\begin{align*}
    \bs{W}_i=\mathrm{span}\{\bs{\nu}_i,(\mc{A}-\lambda_i \mathrm{id})\bs{\nu}_i,\ldots,(\mc{A}-\lambda_i \mathrm{id})^{k_i-1}\bs{\nu}_i\}
\end{align*}
证明:若$\lambda_i \neq \lambda_j$,则$\bs{W} \cap \bs{W}_j =\{\bs{0}\}$。
\end{question}

\begin{proof}
    由$(x-\lambda_i)^{k_i}$与$(x-\lambda_j)^{k_j}$互素,则存在$f(x),g(x)$,使$f(x)(x-\lambda_i)^{k_i}+g(x)(x-\lambda_j)^{k_j}=1$,
    对应到线性变换$\mc{A}$上即为$f(\mc{A})(\mc{A}-\lambda_i\mathrm{id})^{k_i}+g(\mc{A})(\mc{A}-\lambda_j\mathrm{id})^{k_j}=\mathrm{id}$。
    
    令$\bs{\alpha} \in \bs{W}_i \cap \bs{W}_j$,则$\bs{\alpha} \in \bs{W}_i$,于是$\bs{\alpha}=l_1\bs{\nu}_i+\cdots+l_{k_i}(\mc{A}-\lambda_i\mathrm{id})^{k_i-1}\bs{\nu}_i$,
    $(\mc{A}-\lambda_i\mathrm{id})^{k_i}\bs{\alpha}=l_1(\mc{A}-\lambda_i\mathrm{id})^{k_i} \bs{\nu}_i+\cdots+l_{k_i}(\mc{A}-\lambda_i\mathrm{id})^{2k_i-1}=\bs{0}$;
    同理,$\bs{\alpha} \in \bs{W}_j$,$(\mc{A}-\lambda_j\mathrm{id})^{k_j}\bs{\alpha}=\bs{0}$。
    又因为$\bs{\alpha}=\mathrm{id}(\bs{\alpha})=f(\mc{A})(\mc{A}-\lambda_i\mathrm{id})^{k_i}\bs{\alpha}+g(\mc{A})(\mc{A}-\lambda_j\mathrm{id})^{k_j}\bs{\alpha}$。
    所以$\bs{\alpha}=\mathrm{id}(\bs{\alpha})=\bs{0}+\bs{0}=\bs{0}$,
    即$\bs{W} \cap \bs{W}_j =\{\bs{0}\}$。

\end{proof}
\ifx\allfiles\undefined
\end{document}
\fi