\ifx\allfiles\undefined
\documentclass[12pt, a4paper, oneside, UTF8]{ctexbook}
\def\configPath{../config}
\def\basicPath{\configPath/basic}


% 在这里定义需要的包
\usepackage{amsmath}
\usepackage{amsthm}
\usepackage{amssymb}
\usepackage{graphicx}
\usepackage{mathrsfs}
\usepackage{enumitem}
\usepackage{geometry}
\usepackage[colorlinks, linkcolor=black]{hyperref}
\usepackage{stackengine}
\usepackage{yhmath}
\usepackage{extarrows}
\usepackage{arydshln}
% \usepackage{unicode-math}
\usepackage{tasks}
\usepackage{fancyhdr}
\usepackage[dvipsnames, svgnames]{xcolor}
\usepackage{listings}

\definecolor{mygreen}{rgb}{0,0.6,0}
\definecolor{mygray}{rgb}{0.5,0.5,0.5}
\definecolor{mymauve}{rgb}{0.58,0,0.82}

\graphicspath{ {figure/},{../figure/}, {config/}, {../config/} }

\linespread{1.6}

\geometry{
    top=25.4mm, 
    bottom=25.4mm, 
    left=20mm, 
    right=20mm, 
    headheight=2.17cm, 
    headsep=4mm, 
    footskip=12mm
}

\setenumerate[1]{itemsep=5pt,partopsep=0pt,parsep=\parskip,topsep=5pt}
\setitemize[1]{itemsep=5pt,partopsep=0pt,parsep=\parskip,topsep=5pt}
\setdescription{itemsep=5pt,partopsep=0pt,parsep=\parskip,topsep=5pt}

\lstset{
    language=Mathematica,
    basicstyle=\tt,
    breaklines=true,
    keywordstyle=\bfseries\color{NavyBlue}, 
    emphstyle=\bfseries\color{Rhodamine},
    commentstyle=\itshape\color{black!50!white}, 
    stringstyle=\bfseries\color{PineGreen!90!black},
    columns=flexible,
    numbers=left,
    numberstyle=\footnotesize,
    frame=tb,
    breakatwhitespace=false,
} 
% 在这里定义自己顺手的环境
\def\d{\mathrm{d}}
\def\i{\mathrm{i}}
\def\R{\mathbb{R}}
\newcommand{\bs}[1]{\boldsymbol{#1}}
\newcommand{\mc}[1]{\mathcal{#1}}
\newcommand{\ora}[1]{\overrightarrow{#1}}
\newcommand{\myspace}[1]{\par\vspace{#1\baselineskip}}
\newcommand{\xrowht}[2][0]{\addstackgap[.5\dimexpr#2\relax]{\vphantom{#1}}}
\newenvironment{ca}[1][1]{\linespread{#1} \selectfont \begin{cases}}{\end{cases}}
\newenvironment{vx}[1][1]{\linespread{#1} \selectfont \begin{vmatrix}}{\end{vmatrix}}
\newcommand{\tabincell}[2]{\begin{tabular}{@{}#1@{}}#2\end{tabular}}
\newcommand{\pll}{\kern 0.56em/\kern -0.8em /\kern 0.56em}
\newcommand{\dive}[1][F]{\mathrm{div}\;\bs{#1}}
\newcommand{\rotn}[1][A]{\mathrm{rot}\;\bs{#1}} 
\usepackage[strict]{changepage} 
\usepackage{framed}

\definecolor{greenshade}{rgb}{0.90,1,0.92}
\definecolor{redshade}{rgb}{1.00,0.88,0.88}
\definecolor{brownshade}{rgb}{0.99,0.95,0.9}
\definecolor{lilacshade}{rgb}{0.95,0.93,0.98}
\definecolor{orangeshade}{rgb}{1.00,0.88,0.82}
\definecolor{lightblueshade}{rgb}{0.8,0.92,1}
\definecolor{purple}{rgb}{0.81,0.85,1}
\theoremstyle{definition}
\newtheorem{myDefn}{\indent 定义}[section]
% \newtheorem{myLemma}{\indent 引理}[section]
\newtheorem{myLemma}{\indent 引理}[chapter]
\newtheorem{myThm}[myLemma]{\indent 定理}
\newtheorem{myCorollary}[myLemma]{\indent 推论}
\newtheorem{myCriterion}[myLemma]{\indent 准则}
\newtheorem*{myRemark}{\indent 注}
\newtheorem{myProposition}{\indent 命题}[section]


\newenvironment{formal}[2][]{%
    \def\FrameCommand{%
        \hspace{1pt}%
        {\color{#1}\vrule width 2pt}%
        {\color{#2}\vrule width 4pt}%
        \colorbox{#2}%
    }%
    \MakeFramed{\advance\hsize-\width\FrameRestore}%
    \noindent\hspace{-4.55pt}%
    \begin{adjustwidth}{}{7pt}\vspace{2pt}\vspace{2pt}}{%
        \vspace{2pt}\end{adjustwidth}\endMakeFramed%
}

\newenvironment{defn}{\begin{formal}[Green]{greenshade}\vspace{-\baselineskip / 2}\begin{myDefn}}{\end{myDefn}\end{formal}}
\newenvironment{thm}{\begin{formal}[LightSkyBlue]{lightblueshade}\vspace{-\baselineskip / 2}\begin{myThm}}{\end{myThm}\end{formal}}
\newenvironment{lemma}{\begin{formal}[Plum]{lilacshade}\vspace{-\baselineskip / 2}\begin{myLemma}}{\end{myLemma}\end{formal}}
\newenvironment{corollary}{\begin{formal}[BurlyWood]{brownshade}\vspace{-\baselineskip / 2}\begin{myCorollary}}{\end{myCorollary}\end{formal}}
\newenvironment{criterion}{\begin{formal}[DarkOrange]{orangeshade}\vspace{-\baselineskip / 2}\begin{myCriterion}}{\end{myCriterion}\end{formal}}
\newenvironment{rmk}{\begin{formal}[LightCoral]{redshade}\vspace{-\baselineskip / 2}\begin{myRemark}}{\end{myRemark}\end{formal}}
\newenvironment{proposition}{\begin{formal}[RoyalPurple]{purple}\vspace{-\baselineskip / 2}\begin{myProposition}}{\end{myProposition}\end{formal}}

\newtheorem{example}{\indent \color{SeaGreen}{例}}[section]
\newtheorem{question}{\color{SeaGreen}{题}}[chapter]
% \renewenvironment{proof}{\indent\textcolor{SkyBlue}{\textbf{证明.}}\;}{\qed\par}
% \newenvironment{solution}{\indent\textcolor{SkyBlue}{\textbf{解.}}\;}{\qed\par}

\renewcommand{\proofname}{\textbf{\textcolor{TealBlue}{证明}}}
\newenvironment{solution}{\begin{proof}[\textbf{\textcolor{TealBlue}{解}}]}{\end{proof}}

\definecolor{mygreen}{rgb}{0,0.6,0}
\definecolor{mygray}{rgb}{0.5,0.5,0.5}
\definecolor{mymauve}{rgb}{0.58,0,0.82}

\graphicspath{ {figure/},{../figure/}, {config/}, {../config/},{cover/graph} }

\linespread{1.6}

\geometry{
    top=25.4mm, 
    bottom=25.4mm, 
    left=20mm, 
    right=20mm, 
    headheight=2.17cm, 
    headsep=4mm, 
    footskip=12mm
}

\setenumerate[1]{itemsep=5pt,partopsep=0pt,parsep=\parskip,topsep=5pt}
\setitemize[1]{itemsep=5pt,partopsep=0pt,parsep=\parskip,topsep=5pt}
\setdescription{itemsep=5pt,partopsep=0pt,parsep=\parskip,topsep=5pt}

\lstset{
    language=Mathematica,
    basicstyle=\tt,
    breaklines=true,
    keywordstyle=\bfseries\color{NavyBlue}, 
    emphstyle=\bfseries\color{Rhodamine},
    commentstyle=\itshape\color{black!50!white}, 
    stringstyle=\bfseries\color{PineGreen!90!black},
    columns=flexible,
    numbers=left,
    numberstyle=\footnotesize,
    frame=tb,
    breakatwhitespace=false,
} 

\begin{document}
\else
\fi

\chapter{内积空间}
\begin{question}(p119.1)
    证明内积的平行四边形恒等式和极化恒等式(定理6.3)。

    设$(\cdot,\cdot)$为实线性空间$\bs{V}$上的内积,$||\cdot||$为由内积定义的范数,则
    \begin{enumerate}[label=(\arabic*)]
        \item 平行四边形恒等式:对任意$\bs{x},\bs{y} \in \bs{V}$,$2(||\bs{x}||^2+||\bs{y}||^2)=||\bs{x}+\bs{y}||^2+||\bs{x}-\bs{y}||^2$
        \item 极化恒等式:对任意$\bs{x},\bs{y} \in \bs{V}$,$4(\bs{x},\bs{y})=||\bs{x}+\bs{y}||^2-||\bs{x}-\bs{y}||^2$
    \end{enumerate}
\end{question}

\begin{proof}
    \begin{align*}
        ||\bs{x}+\bs{y}||^2+||\bs{x}-\bs{y}||^2=&||\bs{x}||^2+2(\bs{x},\bs{y})+||\bs{y}||^2
        +||\bs{x}||^2-2(\bs{x},\bs{y})+||\bs{y}||^2=2(||\bs{x}||^2+||\bs{y}||^2) \\
        ||\bs{x}+\bs{y}||^2-||\bs{x}-\bs{y}||^2=&(||\bs{x}||^2+2(\bs{x},\bs{y})+||\bs{y}||^2)
        -(||\bs{x}||^2-2(\bs{x},\bs{y})+||\bs{y}||^2)=4(\bs{x},\bs{y})
    \end{align*}
\end{proof}

\begin{question}(p119.2)
    求$\R^3$上的一组标准正交基$\bs{\nu}_1,\bs{\nu}_2,\bs{\nu}_3$,其中$\bs{\nu}_1$与向量$(1,1,1)^T$线性相关。
\end{question}

\begin{solution}
    由题意得,设$\bs{\nu}_1=k(1,1,1)^T$,则$k=\frac{\sqrt{3}}{3}$,
    易知$\bs{\nu}_2=\frac{1}{\sqrt{2}}(-1,0,1)^T,\bs{\nu}_3=\frac{1}{\sqrt{6}}(1,-2,-1)^T$与$\bs{\nu}_1$正交,且$\bs{\nu}_2$与$\bs{\nu}_3$正交。
    则标准正交基为$(\bs{\nu}_1,\bs{\nu}_2,\bs{\nu}_3)=\begin{pmatrix}
        \frac{1}{\sqrt{3}} & -\frac{1}{\sqrt{2}} & \frac{1}{\sqrt{6}} \\
        \frac{1}{\sqrt{3}} & 0& -\frac{2}{\sqrt{6}} \\
        \frac{1}{\sqrt{3}} & \frac{1}{\sqrt{2}}& \frac{1}{\sqrt{6}}
    \end{pmatrix}$
\end{solution}

\begin{question}(p119.3)
    已知$\R^3$上的向量$\bs{\nu}=(1,2,-1)^T,\bs{\omega}=(1,1,0)^T$,求一个与$\bs{\nu},\bs{\omega}$都正交的单位向量。
\end{question}


\begin{solution}
    构造$\bs{A}=\begin{pmatrix}
        \bs{\nu}^T\\
        \bs{\omega}^T
    \end{pmatrix}=\begin{pmatrix}
        1 & 2 & -1\\
        1 & 1 & 0
    \end{pmatrix}$,即求$\bs{A}{\bs{x}}=\bs{0}$的一个单位长度解
    $\bs{x}=\pm \frac{1}{\sqrt{3}}\begin{pmatrix}
        -1 \\
        1 \\
        1
    \end{pmatrix}$
\end{solution}


\begin{question}(p119.6)
    在一元多项式函数构成的线性空间$\bs{R}[x]$上定义内积为
    \begin{align*}
        (f,g)=\int_{-1}^1 f(x)g(x) \d x
    \end{align*}
    试求$1,x,x^2,x^3$的Schmidt正交化。
\end{question}

\begin{solution}
    \begin{align*}
        e_1=&1 \\
        e_2=&x-\frac{(x,1)}{(1,1)} 1 =x \\
        e_3=&x^2-\frac{(x^2,x)}{(x,x)}x-\frac{(x^2,1)}{(1,1)} 1=x^2-\frac{1}{3} \\
        e_4=&x^3-\frac{(x^3,x^2-\frac{1}{3})}{(x^2-\frac{1}{3},x^2-\frac{1}{3})}(x^2-\frac{1}{3})-
        \frac{(x^3,x)}{(x,x)}x-\frac{(x^3,x)}{(1,1)}1=x^3-\frac{3}{5}x
    \end{align*}
    $e=(e_1,e_2,e_3,e_4)=(1,x,x^2-\frac{1}{3},x^2-\frac{3}{5}x)$
\end{solution}

\begin{question}(p120.11)
    设$\bs{\nu}\in \R^n$,$\bs{A}=\bs{\nu}\bs{\nu}^T$,求$\bs{A}$的正交对角化。
\end{question}

\begin{proof}
    令$\bs{\nu}=(a_1,\ldots,a_n)^T$,显然$(\bs{\nu}\bs{\nu}^T)^T=\bs{\nu}\bs{\nu}^T$,$\bs{A} \in \mc{S}^n$必可相似对角化。
    $\mathrm{r}(\bs{A})=\mathrm{r}(\bs{\nu}\bs{\nu}^T) \leq \min(\mathrm{\bs{\nu},\mathrm{\bs{\nu}^T}})=1$,所以$\mathrm{r}(\bs{A})=0,1$,
    
    若$\mathrm{r}(\bs{A})=0$,则$\bs{A}=\bs{O} \Rightarrow \bs{\nu}=\bs{0}$,此时$\bs{A}$相似于$\mathrm{diag}\{0,\ldots,0\}$。
    取$\R^n$上的一组标准正交基$\bs{e}_1,\ldots,\bs{e}_n$组成矩阵$\bs{Q}$,
    使得$\bs{Q}^T\bs{A}\bs{Q}=\bs{\Lambda}=\bs{O}$其中$\bs{e}_1=(1,0,\ldots,0),\ldots,\bs{e}_n=(0,\ldots,1)$。

    若$\mathrm{r}(\bs{A})=1$,则$a_1,\ldots,a_n$不全为0,不妨令$a_1 \neq 0$,所以
    $\bs{A}=\begin{pmatrix}
        a_1^2& a_1a_2 & \cdots & a_1a_n \\
        a_2a_1 & a_2^2 &\cdots & a_2a_n\\
        \vdots &\vdots & \ddots & \vdots \\
        a_na_1& a_na_2&\cdots  & a_n^2 
    \end{pmatrix}$,其特征值为$\lambda_1=\ldots=\lambda_{n-1}=0,\lambda_n=\mathrm{tr}(\bs{A})=\bs{\nu}^T\bs{\nu}$。
    
    当$\lambda=0$时,计算其特征向量,对$\bs{A}$进行行变换,得$\begin{pmatrix}
        a_1^2 & a_1a_2 &\cdots &a_1a_n \\
        0 &0 &\cdots & 0 \\
        \vdots &\vdots & \ddots& \vdots \\
        0 &0 &\cdots &0 
    \end{pmatrix}$,
    则特征向量为$\bs{e}_1=(-a_2,a_1,0,\ldots,0)^T,\bs{e}_2=(-a_3,0,a_1,0,\ldots,0)^T,\bs{e}_k=(-a_k,0,\ldots,0,a_1,0,\ldots,0)^T,
    \bs{e}_{n-1}=(-a_{n-1},0,\ldots,0,a_1)^T$,将$e_1,\ldots,e_n$进行施密特正交并单位化,得
    \begin{align*}
        \bs{\eta}_1=&\frac{1}{||\bs{\eta}_1||} \bs{e}_1 \\
        \bs{\eta}_2=&\frac{1}{||\bs{\eta}_2||}(\bs{e}_2-\frac{(\bs{e}_2,\bs{\eta}_1)}{\bs{\eta}_1,\bs{\eta}_1} \bs{\eta}_1) \\
        \vdots& \\
        \bs{\eta}_{n-1}=&\frac{1}{||\bs{\eta}_{n-1}||}(\bs{e}_{n-1}-\frac{(\bs{e}_{n-1},\bs{\eta}_{n-2})}{\bs{\eta}_{n-2},\bs{\eta}_{n-2}} \bs{\eta}_{n-2}-\cdots\frac{(\bs{e}_{n-1},\bs{\eta}_1)}{\bs{\eta}_1,\bs{\eta}_1} \bs{\eta}_1)
    \end{align*}
    则$\bs{\eta}_1,\ldots,\bs{\eta}_{n-1}$是$\lambda=0$的$n-1$个正交的单位特征向量。

    当$\lambda=1$时,其单位特征向量为$\bs{\eta}_n=\frac{1}{\bs{\nu}^T\bs{\nu}}(a_1,\ldots,a_n)^T$,且与$\bs{\eta}_1,\ldots,\bs{\eta}_{n-1}$正交。

    令$\bs{Q}=(\bs{\eta}_1,\ldots,\bs{\eta}_{n})$,则$\bs{Q}^T\bs{A}\bs{Q}=\bs{\Lambda}=\mathrm{diag}\{0,\ldots,0,\bs{\nu}^T\bs{\nu}\}$。
\end{proof}

\begin{question}(p120.12)
    设$\bs{A}=\begin{pmatrix}
        0 & a& b \\
        -a & 0 &c \\
        -b & -c & 0
    \end{pmatrix}$,已知$\bs{A}$正交相似于$\bs{W}=\begin{pmatrix}
        0 & 0& 0 \\
        0 & 0 &-\omega \\
        0 & \omega & 0
    \end{pmatrix}$型矩阵,求出$\omega$的值。
\end{question}

\begin{solution}
    由于$\bs{A}$与$\bs{W}$相似,所以两者的行列式因子相同。
    \begin{align*}
        \bs{\lambda \bs{E}-\bs{A}}=\begin{pmatrix}
            \lambda & -a& -b \\
            a & \lambda & -c \\
            b & c & \lambda 
        \end{pmatrix} \quad\bs{\lambda \bs{E}-\bs{W}}=\begin{pmatrix}
            \lambda & 0& 0 \\
            0 & \lambda & \omega \\
            0 & -\omega & \lambda 
        \end{pmatrix}
    \end{align*}
    矩阵$\bs{A}$的行列式因子为$D_1=D_1=1,D_3=\lambda(\lambda^2+a^2+b^2+c^2)$;
    矩阵$\bs{W}$的行列式因子为$D_1=D_1=1,D_3=\lambda(\lambda^2+\omega^2)$。
    所以$a^2+b^2+c^2=\omega^2 \Rightarrow \omega=\pm \sqrt{a^2+b^2+c^2}$。
\end{solution}

\begin{question}(p120.13)
    证明:实反对称矩阵($\bs{A}^T=-\bs{A}$)于反Hermite矩阵($\bs{A}^H=-\bs{A}$)的特征值为纯虚数或0。
\end{question}

\begin{proof}
    不妨设矩阵$\bs{A}$的特征值为$\lambda=a+bi$,特征向量为$\bs{x}=\bs{u}+\bs{v}i$,
        其中$a,b \in \R$,$\bs{\mu},\bs{\nu} \in \R^n$。
        下面提供两种方法来解决这个问题:
    \begin{enumerate}[label=(\arabic*)]
        \item     
        $\bs{A}\bs{x}=\lambda\bs{x} \Rightarrow \bs{A}(\bs{u}+\bs{v}i)=(a+bi)(\bs{u}+\bs{v}i) \Rightarrow \bs{A}\bs{\mu}=a\bs{\mu}-b\bs{\nu},\bs{A}\bs{\nu}=a\bs{\nu}+\bs{\mu}$,
        且$\bs{\mu}^H\bs{A}\bs{\mu}=(\bs{\mu}^H\bs{A}\bs{\mu})^H=\bs{\mu}^H\bs{A}^H\bs{\mu}=-\bs{\mu}^H\bs{A}\bs{\mu}$,所以$\bs{\mu}^H\bs{A}\bs{\mu}=0$;同理,$\bs{\nu}^H\bs{A}\bs{\nu}=0$。
        又因为$\bs{\mu}^H\bs{A}\bs{\mu}=a\bs{\mu}^H\bs{\mu}-b\bs{\mu}^H\bs{\nu},\bs{\nu}^H\bs{A}\bs{\nu}=a\bs{\nu}^H\bs{\nu}+b\bs{\nu}^H\bs{\mu}$,
        $\bs{\mu}^H\bs{A}\bs{\mu}+\bs{\nu}^H\bs{A}\bs{\nu}=a(||\bs{\mu}||^2+||\bs{\nu}||^2)=0$,而$\bs{\mu},\bs{\nu}$不能同时为$\bs{0}$,所以$a=0$,这说明了$\lambda=bi$为纯虚数或0。
        \item  $\bs{A}\bs{x}=\lambda\bs{x} \Rightarrow  (\bs{A}\bs{x})^H=(\lambda\bs{x})^H \Rightarrow \bs{x}^H\bs{A}^H=\bar{\lambda}\bs{x}^H $,
        两边同时与$\bs{x}$做内积,得$\bs{x}^H\bs{A}^H\bs{x}=\bar{\lambda}\bs{x}^H\bs{x} \Rightarrow -\bs{x}^H\bs{A}\bs{x}=\bar{\lambda}\bs{x}^H\bs{x}
        \Rightarrow -\lambda \bs{x}^H\bs{x}=\bar{\lambda}\bs{x}^H\bs{x}$,此时有$-(a+bi)=a-bi \Rightarrow a=0$,这说了$\lambda=bi$为纯虚数或0。
        
        ps:两种方法都不需要针对矩阵类型分类讨论,原因在于共轭转置$H$包含转置$T$。
    \end{enumerate}
\end{proof}

\begin{question}(p121.15)
    设$\bs{V}$为Euclid空间,$\mc{A}$为$\bs{V}$上的对称变换,若对一切非零向量$\bs{\nu} \in \bs{V}$,均有$(\mc{A}\bs{\nu},\bs{\nu}) > 0$,
    这样的对称变换称为\textbf{正定}的,求证:正定的对称变换在标准正交基下的矩阵为正定矩阵。
\end{question}

\begin{proof}
    取标准正交基$(\bs{e}_1,\ldots,\bs{e}_n)$,设$\bs{v}$在基下的坐标为$\bs{x}$,
    对称变换$\mc{A}$在基下的变换矩阵为$\bs{A}$
    则$\mc{A}\bs{v}=\bs{A}\bs{x}$。由$(\mc{A}\bs{\nu},\bs{\nu}) > 0$与$\bs{A}^T=\bs{A}$,得$\bs{x}^T\bs{A}^T> 0 \Rightarrow \bs{x}^T\bs{A}\bs{x}>0$,
    即$\bs{A}$是正定矩阵。
\end{proof}


\begin{question}(p121.16)
    试给出一个既不是对称变换,也不是正交变换的正规变换。
\end{question}

\begin{solution}
    不妨设矩阵$\bs{A}$为满足这些变换,
    在标准正交基$(\bs{e}_1,\ldots,\bs{e}_n)$下的变换矩阵。
    对称变换即$\bs{A}^T=\bs{A}$;正交变换即$\bs{A}\bs{A}^T=\bs{E}$;
    正规变换为$\bs{A}^H\bs{A}=\bs{A}\bs{A}^H$。由于对称变换与正交变换定义在欧式空间,矩阵需要满足的条件变为
    $\bs{A}\neq \bs{A}^T,\bs{A}^T\bs{A}=\bs{A}^T\bs{A} \neq \bs{E}$。
    显然,$\bs{A}=\begin{pmatrix}
        0 & 0 & 2 \\
        2 & 0 & 0 \\
        0 & 2 & 0
    \end{pmatrix}$为所求,其中$\bs{A}^T\neq \bs{A},\bs{A}^T\bs{A}=\bs{A}^T\bs{A}=4\bs{E}$。
\end{solution}


\begin{question}(p121.21)
设$\bs{V}$为Euclid空间,$\bs{T}$为$\bs{V}$上的线性变换,
且对任何$\bs{\nu}\in \bs{V}$,均有$\bs{\nu}-\mc{T}\bs{\nu} \in (\mathrm{Im}\mc{T})^{\perp}$,
这样的线性变换$\bs{V}$上的\textbf{投影变换}。
\begin{enumerate}[label=(\arabic*)]
    \item 证明:对任何$\bs{V}$上的投影变换$\mc{T}$,有$\mathrm{ker}\mc{T}=(\mathrm{Im}\mc{T})^{\perp}$,该命题的逆命题是否成立?
    \item 证明:线性变换$\mc{T}$为$\bs{V}$上的投影变换的充分必要条件是$\mc{T}$是对称变换且满足$\mc{T}^2=\mc{T}$。
    \item 设$\mc{T}_1,\mc{T}_2$均为$\bs{V}$上的投影变换,求证:$\mc{T}_1+\mc{T}_2$为投影变换当且仅当$\mc{T}_1\mc{T}_2=\mc{O}$,当且仅当$\mc{T}_1 \perp \mc{T}_2$。
    \item 设$\mc{T}_1,\mc{T}_2$均为$\bs{V}$上的投影变换,求证:$\mc{T}_1-\mc{T}_2$为投影变换当且仅当$\mathrm{Im}\mc{T}_1 \supset \mathrm{Im}\mc{T}_2$
    \item 设$\mc{T}_1,\mc{T}_2$均为$\bs{V}$上的投影变换,求证:$\mc{T}_1\mc{T}_2$为投影变换当且仅当$\mc{T}_1\mc{T}_2=\mc{T}_2\mc{T}_1$,且此时有$\mathrm{Im}\mc{T}_1\mc{T}_2=\mathrm{Im}\mc{T}_1 \cap \mathrm{Im} \mc{T}_2$。
    \item 设$\mc{A}$为$\bs{V}$上的对称变换,证明$\mc{A}$可表示为一组投影变换的线性组合,且该组投影变换的像空间的直和恰为$\bs{V}$。
\end{enumerate}
\end{question}


\begin{proof}
    由于线性变换和线性变换下对应的矩阵是基本等价的,在后续的证明过程中,不再对两者进行区分。
    \begin{thm} \label{反对称}
        若对于任意的$\bs{\alpha} \in \bs{V}$ ,有$(\bs{\alpha},\bs{A}\bs{\alpha})=\bs{\alpha}^T\bs{A}\bs{\alpha}=0$恒成立,则$\bs{A}$为反对称矩阵(即$\bs{A}^T=-\bs{A}$)。
    \end{thm}
    \begin{enumerate}[label=(\arabic*)]
        \item 先证$\mathrm{ker}\mc{T} \subset (\mathrm{Im}\mc{T})^{\perp}$,只需证明任取$\bs{\alpha} \in \mathrm{ker}\mc{T}$有$\bs{\alpha} \in (\mathrm{Im}\mc{T})^{\perp}$,
        即任取$\bs{\alpha} \in \bs{V},\bs{\beta} \in \mathrm{Im}\mc{T}$有$(\bs{\alpha},\mc{T}\bs{\gamma})=0$,
        其中$\mc{T}\bs{\alpha}=0,\mc{T}\bs{\gamma}=\bs{\beta}$。根据投影变换的定义,有$0=(\bs{\alpha}-\mc{T}\bs{\alpha},\mc{T}\bs{\gamma})=(\bs{\alpha},\mc{T}\bs{\gamma})$,
        于是$\mathrm{ker}\mc{T} \subset (\mathrm{Im}\mc{T})^{\perp}$;

        下证$\mathrm{ker}\mc{T} =(\mathrm{Im}\mc{T})^{\perp}$,由于$\mathrm{dim}(\mathrm{ker}\mc{T})+\mathrm{dim}(\mathrm{Im}\mc{T})=n$,
        则$\mathrm{dim}(\mathrm{ker}\mc{T})=\mathrm{dim}(\mathrm{Im}\mc{T})^{\perp}$,
        于是$\mathrm{ker}\mc{T} =(\mathrm{Im}\mc{T})^{\perp}$。

        \textbf{逆命题不一定成立}。令$\mc{T}$在标准正交基下的矩阵为$2\bs{E}$,此时$\mathrm{ker}(\mc{T})=\{\bs{0}\},\mathrm{Im}\mc{T}=\R^n,(\mathrm{Im}\mc{T})^{\perp}=\{\bs{0}\}$,
        这满足了$\mathrm{ker}\mc{T} =(\mathrm{Im}\mc{T})^{\perp}$,
        但$\bs{\nu}-\mc{T}\bs{\nu}=\bs{\nu}-2\bs{E}\bs{\nu}=-\bs{\nu}\notin (\mathrm{Im}\mc{T})^{\perp}$。

        \item 先证必要性:由(1)可得,任取$\bs{\alpha} \in \bs{V}$有$\bs{\alpha} -\mc{T}\bs{\alpha} \in \mathrm{ker}{\mc{T}}$,
        因此$\mc{T}(\bs{\alpha}-\mc{T}\bs{\alpha})=\bs{0} \Rightarrow \mc{T}\bs{\alpha}-\mc{T}^2 \bs{\alpha}=\bs{0}$,这说明了$\mc{T}^2=\mc{T}$。
        又由于$(\bs{\alpha}-\mc{T}\bs{\alpha},\mc{T}\bs{\alpha})=0$,
        而$(\mc{T}(\bs{\alpha}-\mc{T}\bs{\alpha}),\bs{\alpha})=(\mc{T}\bs{\alpha}-\mc{T}^2 \bs{\alpha},\bs{\alpha})=(\bs{0},\bs{\alpha})=0$,
        这说明了$(\bs{\alpha}-\mc{T}\bs{\alpha},\mc{T}\bs{\alpha})=(\mc{T}(\bs{\alpha}-\mc{T}\bs{\alpha}),\bs{\alpha})$,即$\mc{T}$是对称变换。
        
        再证充分性:即任取$\bs{\alpha},\bs{\beta} \in \bs{V},(\bs{\alpha}-\mc{T}\bs{\alpha},\mc{T}\bs{\beta})
        =(\mc{T}(\bs{\alpha}-\mc{T}\bs{\alpha}),\bs{\beta})
        =(\mc{T}\bs{\alpha}-\mc{T}^2\bs{\alpha},\bs{\beta})
        =((\mc{T}-\mc{T}^2)\bs{\alpha},\bs{\beta})
        =(\bs{0},\bs{\beta})
        =0$。
        
        \item \begin{enumerate}
            \item 
            先证充分性:显然$\mc{T}_1+\mc{T}_2$是对称变换,只需证明$(\mc{T}_1+\mc{T}_2)^2=\mc{T}_1+\mc{T}_2$,
            即证$\mc{T}_1\mc{T}_2+\mc{T}_2\mc{T}_1=\mc{O}$,而$\mc{T}_2\mc{T}_1=(\mc{T}_1\mc{T}_2)^T=\mc{O}$,
            所以$\mc{T}_1\mc{T}_2+\mc{T}_2\mc{T}_1=\mc{O}$。
            
            再证必要性:若$\mc{T}_1+\mc{T}_2$是投影变换,则$\mc{T}_1\mc{T}_2+\mc{T}_2\mc{T}_1=\mc{O}$。
            又因为$\mc{T}_1\mc{T}_2=\mc{T}_1\mc{T}^2_2=-\mc{T}_2(\mc{T}_1\mc{T}_2)=\mc{T}_2(\mc{T}_2\mc{T}_1)=\mc{T}^2_2\mc{T}_1=\mc{T}_2\mc{T}_1$,
            所以$\mc{T}_1\mc{T}_2=\mc{T}_2\mc{T}_1=\mc{O}$。

            \item 先证充分性,任取$\bs{\alpha},\bs{\beta} \in \bs{V}$,有$(\mc{T}_1\bs{\alpha},\mc{T}_2\bs{\beta})=0$
        ,不妨令$\bs{\alpha}=\bs{\beta}$,则
        $
            (\mc{T}_1\bs{\alpha},\mc{T}_2\bs{\alpha}) 
            =(\bs{\alpha},\mc{T}_1\mc{T}_2\bs{\alpha})=0
        $,
        根据定理\ref{反对称}有$(\mc{T}_1\mc{T}_2)^T=-(\mc{T}_1\mc{T}_2)$,即$\mc{T}_1\mc{T}_2+\mc{T}_2\mc{T}_1=\mc{O}$,所以$\mc{T}_1+\mc{T}_2$是投影变换。

        再证必要性,任取$\bs{\alpha},\bs{\beta} \in \bs{V}$,欲证$(\mc{T}_1\bs{\alpha},\mc{T}_2\bs{\beta})=0$,
        只需注意到$(\mc{T}_1\bs{\alpha},\mc{T}_2\bs{\beta})=(\bs{\alpha},\mc{T}_1\mc{T}_2\bs{\beta})=(\bs{\alpha},\bs{0})=0$。
        \end{enumerate}
        \item 先证充分性,显然$\mc{T}_1-\mc{T}_2$是对称变换,只需证明$(\mc{T}_1-\mc{T}_2)^2=\mc{T}_1-\mc{T}_2$,
        即证$2\mc{T}_2=\mc{T}_1\mc{T}_2+\mc{T}_2\mc{T}_1$。任取$\bs{\alpha} \in \bs{V}$,由于$\bs{\alpha}-\mc{T}_1\bs{\alpha} \in (\mathrm{Im}\mc{T}_1)^{\perp} \subset (\mathrm{Im}\mc{T}_2)^{\perp} $,
        则$(\bs{\alpha}-\mc{T}_1\bs{\alpha},\mc{T}_2\bs{\alpha})=\bs{0}$。
        \begin{align*}
            &(\bs{\alpha}-\mc{T}_1\bs{\alpha},\mc{T}_2\bs{\alpha})\\
            =&(\bs{\alpha},\mc{T}_2 \bs{\alpha})-(\mc{T}_1\bs{\alpha},\mc{T}_2 \bs{\alpha}) \\
            =&(\bs{\alpha},(\mc{T}_2-\mc{T}_1\mc{T}_2 )\bs{\alpha}) 
        \end{align*}
        根据定理\ref{反对称},$(\mc{T}_2-\mc{T}_1\mc{T}_2)^T=-(\mc{T}_2-\mc{T}_1\mc{T}_2)$,于是$2\mc{T}_2=\mc{T}_1\mc{T}_2+\mc{T}_2\mc{T}_1$。

        再证必要性,若$\mc{T}_1-\mc{T}_2$是投影变换,则$2\mc{T}_2=\mc{T}_1\mc{T}_2+\mc{T}_2\mc{T}_1$。
        只需证明任取$\bs{\alpha} \in \bs{V}$,存在$\bs{\beta} \in \bs{V}$,
        有$\mc{T}_2 \bs{\alpha}=\mc{T}_1 \bs{\beta}$。
        由于$\mc{T}_2 \bs{\alpha}=2\mc{T}_2 \bs{\alpha}_1
        =(\mc{T}_1\mc{T}_2+\mc{T}_2\mc{T}_1)\bs{\alpha}_1=
        \mc{T}_1\mc{T}_2\bs{\alpha}_1+\mc{T}_2\mc{T}_1 \bs{\alpha}_1
        =\mc{T}_1\bs{\beta}+\mc{T}_2\bs{\gamma} \Rightarrow
        \mc{T}_2 (\bs{\alpha}-\bs{\gamma})=\mc{T}_1 \bs{\beta}$
        
        
        只需证明任取$\bs{\alpha},\bs{\beta} \in \bs{V}$,
        有$(\bs{\alpha}-\mc{T}_1\bs{\alpha},\mc{T}_2\bs{\beta})=0$,不妨令$\bs{\alpha}=\bs{\beta}$
        \begin{align*}
            (\bs{\alpha}-\mc{T}_1\bs{\alpha},\mc{T}_2\bs{\alpha})
            =&(\bs{\alpha},\mc{T}_2 \bs{\alpha})-(\mc{T}_1\bs{\alpha},\mc{T}_2 \bs{\alpha})=(\bs{\alpha},(\mc{T}_2-\mc{T}_1\mc{T}_2 )\bs{\alpha}) \label{4.1} \tag{1}\\
            (\bs{\alpha}-\mc{T}_1\bs{\alpha},\mc{T}_2\bs{\alpha})
            =&(\bs{\alpha},\mc{T}_2 \bs{\alpha})-(\mc{T}_2\bs{\alpha},\mc{T}_1 \bs{\alpha})=(\bs{\alpha},(\mc{T}_2-\mc{T}_2\mc{T}_1 )\bs{\alpha})
            \label{4.2} \tag{2}
        \end{align*}
        \eqref{4.1}式加\eqref{4.2}式得:
        \begin{align*}
        2(\bs{\alpha}-\mc{T}_1\bs{\alpha},\mc{T}_2\bs{\alpha})=&(\bs{\alpha},(\mc{T}_2-\mc{T}_1\mc{T}_2 )\bs{\alpha})+(\bs{\alpha},(\mc{T}_2-\mc{T}_2\mc{T}_1) \bs{\alpha}) \\
        =&(\bs{\alpha},(2\mc{T}_2-\mc{T}_2\mc{T}_1-\mc{T}_1\mc{T}_2)\bs{\alpha})\\
        =&(\bs{\alpha},\bs{0})=0 
        \end{align*}
        即$(\bs{\alpha}-\mc{T}_1\bs{\alpha},\mc{T}_2\bs{\alpha})=0$恒成立。
        根据定理\ref{反对称}有$(\mc{T}_2-\mc{T}_1\mc{T}_2)^T=-(\mc{T}_2-\mc{T}_1\mc{T}_2)$,即
        即$(\bs{\alpha}-\mc{T}_1\bs{\alpha},\mc{T}_2\bs{\beta})=(\bs{\alpha},(\mc{T}_2-\mc{T}_1\mc{T}_2 )\bs{\beta})=\bs{0} $。
        \item 先证充分性,$(\mc{T}_1\mc{T}_2)^T=\mc{T}_2^T\mc{T}_1^T=\mc{T}_2\mc{T}_1=\mc{T}_1\mc{T}_2$,则其为对称变换;接着有$(\mc{T}_1\mc{T}_2)^2=\mc{T}_1(\mc{T}_2\mc{T}_1)\mc{T}_2
        =\mc{T}_1\mc{T}_1\mc{T}_2\mc{T}_2=\mc{T}_1^2\mc{T}^2_2=\mc{T}_1\mc{T}_2$,说明为投影变换。
        
        再证必要性,若$\mc{T}_1\mc{T}_2$是投影变换,则其也是对称变换,
        于是$\mc{T}_1\mc{T}_2=(\mc{T}_1\mc{T}_2)^T=\mc{T}_2^T\mc{T}_1^T=\mc{T}_2\mc{T}_1$。

        先证$\mathrm{Im}\mc{T}_1\mc{T}_2 \subset \mathrm{Im}\mc{T}_1 \cap \mathrm{Im}\mc{T}_2$,若$\bs{\alpha} \in \mathrm{Im}{\mc{T}_1\mc{T}_2}$,则$\alpha \in \mathrm{Im}\mc{T}_1$,
        又由于$\mc{T}_1\mc{T}_2=\mc{T}_2\mc{T}_1$,所以$\mathrm{Im}\mc{T}_1\mc{T}_2=\mathrm{Im}\mc{T}_2\mc{T}_1$,
        即$\bs{\alpha} \in \mathrm{Im}\mc{T}_2$,所以$\bs{\alpha} \in \mathrm{Im}\mc{T}_1 \cap \mathrm{Im}\mc{T}_2$。

        再证$\mathrm{Im}\mc{T}_1\mc{T}_2 \supset \mathrm{Im}\mc{T}_1 \cap \mathrm{Im}\mc{T}_2$,任取$\bs{\alpha} \in \mathrm{Im}\mc{T}_1 \cap \mathrm{Im}\mc{T}_2$,
        其中$\bs{\alpha}=\mc{T}_1\bs{\alpha}_1=\mc{T}_2\bs{\alpha}_2$,则
        $\bs{\alpha}=\mc{T}_1\bs{\alpha}=\mc{T}_1^2\bs{\alpha}=\mc{T}_1\mc{T}_2\bs{\alpha}_2$,说明了$\bs{\alpha} \in \mathrm{Im}\mc{T}_1\mc{T}_2$。
        
        综上$\mathrm{Im}\mc{T}_1\mc{T}_2 = \mathrm{Im}\mc{T}_1 \cap \mathrm{Im}\mc{T}_2$。
        

    \end{enumerate}
\end{proof}
\ifx\allfiles\undefined
\end{document}
\fi