\ifx\allfiles\undefined
\documentclass[12pt, a4paper, oneside, UTF8]{ctexbook}
\def\configPath{../config}
\def\coverPath{\configPath/cover}
\def\packagePath{\configPath/package}
\def\theormPath{\configPath/theorem}
\def\customPath{\configPath/custom}
\def\prefacePath{\configPath/preface}

% 在这里定义需要的包
\usepackage{amsmath}
\usepackage{amsthm}
\usepackage{amssymb}
\usepackage{graphicx}
\usepackage{mathrsfs}
\usepackage{enumitem}
\usepackage{geometry}
\usepackage[colorlinks, linkcolor=black]{hyperref}
\usepackage{stackengine}
\usepackage{yhmath}
\usepackage{extarrows}
\usepackage{arydshln}
% \usepackage{unicode-math}
\usepackage{tasks}
\usepackage{fancyhdr}
\usepackage[dvipsnames, svgnames]{xcolor}
\usepackage{listings}


\input{\theormPath/theorem1_zh}
\input{\customPath/custom}




\begin{document}
\else
\fi

\chapter{多项式}
\begin{question}(p44.1)
计算$g(x)$除$f(x)$的商式$q(x)$和余式$r(x)$。
\begin{enumerate}[label=(\arabic*)]
    \item $f(x)=x^4-4x+5$,$g(x)=x^2-x+2$
\end{enumerate}
\end{question}

\begin{solution}
    $f(x)=q(x)g(x)+r(x)$,其中$q(x)=x^2+x-1$,$r(x)=-7x+7$。
    
\end{solution}

\begin{question}(p44.2)
    求多项式$f(x)$和$g(x)$的最大公因式和最小公倍式。
    \begin{enumerate}[label=(\arabic*)]
        \item $f(x)=x^4+x^3+2x^2+x+1$,$g(x)=x^3+2x^2+2x+1$
    \end{enumerate}
\end{question}

\begin{solution}
    $f(x)=q_1(x)g(x)+r_1(x)$,其中$q_1(x)=x-1$,$r_1(x)=2x^2+2x+2$

    $g(x)=q_2(x)r_1(x)$,其中$q_2(x)=\frac{1}{2}x+\frac{1}{2}$

    于是
    \begin{align*}
        \mathrm{gcd}(f(x),g(x))=&\frac{1}{2} r_1(x)=x^2+x+1 \\
        \mathrm{lcm}(f(x),g(x))=&\frac{f(x)g(x)}{\mathrm{gcd}(f(x),g(x))} \\
        =&\frac{(x^4+x^3+2x^2+x+1)(x^3+2x^2+2x+1)}{x^2+x+1}\\
        =&(x^2+1)(x+1)(x^2+x+1)
    \end{align*}
    注:以上计算最大公因式乘的系数为凑首$1$多项式。

\end{solution}



\begin{question}(p44.3)
求多项式$f(x)$和$g(x)$的最大公因式$\mathrm{gcd}(f(x),g(x))$,
以及满足等式$u(x)f(x)+v(x)g(x)=\mathrm{gcd}(f(x),g(x))$的多项式$u(x)$和$v(x)$。
\begin{enumerate}[label=(\arabic*)]
    \item $f(x)=x^4-x^3-4x^2+4x+1$,$g(x)=x^2-x-1$
\end{enumerate}
\end{question}

\begin{solution}
    $f(x)=q_1(x)g(x)+r_1(x)$,其中$q_1(x)=x^2-3$,$r_1(x)=x-2$

    $g(x)=q_2(x)r_1(x)+r_2(x)$,其中$q_2(x)=x$,$r_2(x)=x-1$

    $r_1(x)=q_3(x)r_2(x)+r_3(x)$,其中$q_3(x)=1$,$r_3(x)=-1$

    $r_2(x)=q_4(x)r_3(x)$,其中$q_4(x)=-x+1$

    $\mathrm{gcd}(f(x),g(x))=-r_3(x)=1$,于是$1=u(x)f(x)+v(x)g(x)$,
    其中$u(x)=-1-x$,$v(x)=x^3+x^2-3x-2$

    注:以上计算最大公因式乘的系数为凑首$1$多项式。

\end{solution}
    
\begin{question}(p45.6)
    若多项式$f(x),g(x),u(x),v(x)$满足$u(x)f(x)+v(x)g(x)=\mathrm{gcd}(f(x),g(x))$,证明$u(x),v(x)$互素。
\end{question}

\begin{proof}
    令$d(x)=\mathrm{gcd}(f(x),g(x))$,则$f(x)=d(x)p(x)$,$g(x)=d(x)q(x)$,
    
    $u(x)f(x)+v(x)g(x)=\mathrm{gcd}(f(x),g(x)) \Rightarrow 
    u(x)d(x)p(x)+v(x)d(x)q(x)=d(x) \Rightarrow u(x)p(x)+v(x)q(x)=1$,
    这便说明了$u(x),v(x)$互素。

\end{proof}

\ifx\allfiles\undefined
\end{document}
\fi