\ifx\allfiles\undefined
\documentclass[12pt, a4paper, oneside, UTF8]{ctexbook}
\def\configPath{../config}
\def\coverPath{\configPath/cover}
\def\packagePath{\configPath/package}
\def\theormPath{\configPath/theorem}
\def\customPath{\configPath/custom}
\def\prefacePath{\configPath/preface}

% 在这里定义需要的包
\usepackage{amsmath}
\usepackage{amsthm}
\usepackage{amssymb}
\usepackage{graphicx}
\usepackage{mathrsfs}
\usepackage{enumitem}
\usepackage{geometry}
\usepackage[colorlinks, linkcolor=black]{hyperref}
\usepackage{stackengine}
\usepackage{yhmath}
\usepackage{extarrows}
\usepackage{arydshln}
% \usepackage{unicode-math}
\usepackage{tasks}
\usepackage{fancyhdr}
\usepackage[dvipsnames, svgnames]{xcolor}
\usepackage{listings}


\input{\theormPath/theorem1_zh}
\input{\customPath/custom}




\begin{document}
\else
\fi

\chapter{矩阵的Jordan标准形}
\begin{question}(p63.1)
利用初等变换把下列$\lambda$矩阵化为等价标准形。
\begin{enumerate}[label=(\arabic*)]
    \item $\begin{pmatrix}
        \lambda+1 & \lambda^2+1 &\lambda^2 \\
        \lambda-1 & \lambda^2-1 & \lambda \\
        3\lambda-1 & 3\lambda^2-1 & \lambda^2+2\lambda 
    \end{pmatrix}$
\end{enumerate}
\end{question}

\begin{solution}
    \begin{align*}
        &\begin{pmatrix}
            \lambda+1 & \lambda^2+1 &\lambda^2 \\
            \lambda-1 & \lambda^2-1 & \lambda \\
            3\lambda-1 & 3\lambda^2-1 & \lambda^2+2\lambda 
        \end{pmatrix} \xrightarrow{r_3-r_1}
        \begin{pmatrix}
            \lambda+1 & \lambda^2+1 &\lambda^2  \\
            \lambda-1 & \lambda^2-1 & \lambda \\
            2\lambda-2 & 2\lambda^2-2 & 2\lambda 
        \end{pmatrix}\xrightarrow{r_3-2r_2}
        \begin{pmatrix}
            \lambda+1 & \lambda^2+1 &\lambda^2 \\
            \lambda-1 & \lambda^2-1 & \lambda \\
            0 & 0& 0
        \end{pmatrix} \\
        &\xrightarrow{(r_1-r_2)\times \frac{1}{2}}
        \begin{pmatrix}
            1 & 1 &\frac{1}{2}(\lambda^2-\lambda) \\
            \lambda-1 & \lambda^2-1 & \lambda \\
            0 & 0& 0 
        \end{pmatrix}
        \xrightarrow[c_2-c_1]{r_2-(\lambda-1) r_1}
        \begin{pmatrix}
            1 & 0 &\frac{1}{2}(\lambda^2-\lambda) \\
            0 & \lambda^2-\lambda & -\frac{1}{2}\lambda(\lambda+1)^2 \\
            0 & 0& 0 
        \end{pmatrix}\\
        &\xrightarrow[c_3-(\lambda^2-\lambda) c_1]{c_3\times 2}
        \begin{pmatrix}
            1 & 0 &0 \\
            0 & \lambda^2-\lambda & -\lambda(\lambda+1)^2 \\
            0 & 0& 0 
        \end{pmatrix} 
        \xrightarrow{c_3+\lambda c_2}
        \begin{pmatrix}
            1 & 0 &0 \\
            0 & \lambda^2-\lambda & -3\lambda^2-\lambda \\
            0 & 0& 0 
        \end{pmatrix}
        \xrightarrow[c_3 \times -\frac{1}{4}]{(c_3+3\lambda c_2)}
        \begin{pmatrix}
            1 & 0 &0 \\
            0 & \lambda^2-\lambda & \lambda \\
            0 & 0& 0 
        \end{pmatrix} \\
        &\xrightarrow[c_3-(\lambda-1) c_2 ]{c_2 \leftrightarrow c_3}
        \begin{pmatrix}
            1 & 0 &0 \\
            0 & \lambda & 0\\
            0 & 0& 0 
        \end{pmatrix}
    \end{align*}
\end{solution}

\begin{question}(p64.2)
    求下列$\lambda$矩阵的行列式因子、不变因子和初等因子。
    \begin{tasks}[label=(\arabic*)](2)
        \task $\begin{pmatrix}
            \lambda+1 & -2 & 2 \\
            1 & \lambda-2 & 1\\
            -1 & 1& \lambda-2
        \end{pmatrix}$
        \task $\begin{pmatrix}
            \lambda+1 & -3 & -6 \\
            0 & \lambda-3 & 8\\
            0 & 2& \lambda+5
        \end{pmatrix}$
    \end{tasks}
\end{question}

\begin{solution}
    \begin{enumerate}[label=(\arabic*)]
        \item 
        $D_1=1,D_2=\lambda-1,D_3=(\lambda-1)^3$;
        $d_1=1,d_2=\lambda-1,d_3=(\lambda-1)^2$;
        初等因子组:$(\lambda-1)^2,\lambda-1$。
        \item $D_1=1,D_2=1,D_3=(\lambda+1)(\lambda-4\sqrt{2}+1)(\lambda+4\sqrt{2}+1)$;
        $d_1=1,d_2=1,d_3=(\lambda+1)(\lambda-4\sqrt{2}+1)(\lambda+4\sqrt{2}+1)$;
        初等因子组:$\lambda+1,\lambda-4\sqrt{2}+1,\lambda+4\sqrt{2}+1$。
    \end{enumerate}
\end{solution}


\begin{question}(p64.3)
    设6阶矩阵$\bs{A}(\lambda)$的秩为5,其初等因子是$\lambda-1,\lambda-1,(\lambda-2)^3,\lambda+2,(\lambda+2)^2$,
    求$\bs{A}(\lambda)$的行列式因子、不变因子,以及$\bs{A}(\lambda)$的等价标准形。
\end{question}


\begin{solution}
    $d_5=(\lambda-1)(\lambda-2)^3(\lambda+2)^2,d_4=(\lambda-1)(\lambda+2),d_3=d_2=d_1=1$;
    $D_1=D_2=D_3=1,D_4=(\lambda-1)(\lambda+2),D_5=(\lambda-1)(\lambda-2)^3(\lambda+2)^3$;等价标准型为:
    \begin{align*}
        \bs{S}=\begin{pmatrix}
            1 & & & & & \\
            & 1 & & & &\\
            & & 1 & & &\\
            & & & (\lambda-1)(\lambda+2) & &\\
            & & & & (\lambda-1)(\lambda-2)^3(\lambda+2)^2 &\\
            & & & & & 0 \\
        \end{pmatrix}
    \end{align*}
    其中,空白的位置全为0。
\end{solution}

\begin{question}(p64.5)
    证明:两个等价的$n$阶$\lambda$矩阵的行列式只相差一个常数因子。
\end{question}

\begin{proof}
    不妨令题中所说的两个矩阵为$\bs{A}(\lambda),\bs{B}(\lambda)$。

    若非满秩矩阵,则$|\bs{A}(\lambda)|=|\bs{B}(\lambda)|=0$,显然符合题意。

    若为满秩矩阵,则$|\bs{A}(\lambda)| \neq 0,|\bs{B}(\lambda)| \neq 0$。由于等价的$\lambda$矩阵具有相同的行列式因子和不变因子,
    则$|\bs{A}|=k_1D_n(\bs{A}(\lambda))=k_1D_n,|\bs{B}|=k_2D_n(\bs{B}(\lambda))=k_2D_n$,其中$k_1,k_2 \neq 0$。
    于是$\frac{|\bs{A}(\lambda)|}{|\bs{B}(\lambda)|}=\frac{k_1}{k_2}$。
\end{proof}

\begin{question}(p64.6)
    求下列矩阵的Jordan标准形。
    \begin{tasks}[label=(\arabic*)](2)
        \task $\begin{pmatrix}
            1 & -2 & 2 \\
            1 & -2 & 1\\
            -1 & 1& -2
        \end{pmatrix}$
        \task $\begin{pmatrix}
            1 & -3 & 3 \\
            -2 & -6 & 13\\
            -1 & -4& 8
        \end{pmatrix}$
    \end{tasks}
\end{question}

\begin{solution}
    \begin{enumerate}[label=(\arabic*)]
        \item $D_1=1,D_2=\lambda+1,D_3=(\lambda+1)^2$;$d_1=1,d_2=\lambda+1,d_3=(\lambda+1)^2$;
        初等因子组为$(\lambda+1)^2,\lambda+1$;Jordan标准型为$\bs{J}=\begin{pmatrix}
            -1 & 1& 0\\
            0 & -1& 0 \\
            0 &0 & -1
        \end{pmatrix}$。
        \item $D_1=1,D_2=1,D_3=(\lambda-1)^3$;$d_1=1,d_2=1,d_3=(\lambda-1)^3$;
        初等因子组为$(\lambda-1)^3$;Jordan标准型为$\bs{J}=\begin{pmatrix}
            1 & 1& 0\\
            0 & 1& 1 \\
            0 &0 & 1
        \end{pmatrix}$。
    \end{enumerate}
\end{solution}

\begin{question}(p64.7)
    证明:矩阵$\bs{A}$是幂零阵(即存在正整数$k$,使得$\bs{A}^k=\bs{O}$)当且仅当$\bs{A}$的特征值都等于零。
\end{question}

\begin{proof}
    先证必要性。若$\bs{A}^k=\bs{O}$,由Hamilton-Cayley定理可知,$\lambda^k=0$,则$\lambda=0$。

    再证充分性。矩阵$\bs{A}$必定相似于矩阵$\bs{J}
        =\begin{pmatrix}
            \bs{J}_1 & & & \\
            & \bs{J_2} & & \\
            & & \ddots & \\
            & & & \bs{J}_m
        \end{pmatrix}$,
    其中$\bs{J}_i$为Jordan标准型,$\bs{J}_i=\begin{pmatrix}
        0 & 1 & & \\
        & 0 & 1& &\\
        & & \ddots & \ddots\\
        & & & 0& 1\\
        & & & & 0
    \end{pmatrix}$,空白位置全是0。

    由于$\bs{A}=\bs{P}\bs{J}\bs{P}^{-1}$,
    若$\bs{A}^k=\bs{P}\bs{J}^k\bs{P}^{-1}=\bs{O}$,则$\bs{J}^k=\bs{O}$,
    于是有$\bs{J}_1,\ldots,\bs{J}_k=\bs{O}$。
    令$\bs{J}_i$的阶数为$\mc{N}_i$,容易证明$J_i^{\mc{N}_i}=\bs{O}$。
    只需取$k=\max\limits_i\{\mc{N}_1,\ldots,\mc{N}_m\}$,
    则$\bs{J}_i^k=\bs{O},i=1,\ldots,m$,
    于是$\bs{A}^k=\bs{J}^k=\bs{O}$,这便完成了证明。

\end{proof}

\begin{question}(p64.8)
    设非零矩阵$\bs{A}$是幂零阵,证明$\bs{A}$不相似于对角阵。
\end{question}

\begin{proof}
    由$\bs{A}^k=\bs{O}$与Hamilton-Cayley定理可知,$\lambda^k=0 \Rightarrow \lambda=0$。
    若$\bs{A}$相似于对角矩阵$\bs{\Lambda}$,则$\bs{\Lambda}=\begin{pmatrix}
        0 & & \\
        & \ddots & \\
        & & 0
    \end{pmatrix}=\bs{O}$,此时有$\bs{A}=\bs{P}\bs{\Lambda}\bs{P}^{-1}=\bs{O}$,
    与$\bs{A}$是零矩阵矛盾,所以$\bs{A}$不相似于对角矩阵。

\end{proof}

\begin{question}(p64.9)
    求$3$阶幂零阵的全部可能的Jordan标准形。
\end{question}

\begin{solution}
    由题意与Hamilton-Cayley定理可知,$\bs{A}^k=\bs{O} \Rightarrow \lambda^k=0 \Rightarrow \lambda=0$,
    则$\bs{A}$与$\bs{J}=\begin{pmatrix}
        0 & * &  &\\
        & 0 & * & \\
        & & \ddots &* \\
        & & & 0
    \end{pmatrix}$相似,其中$*$为$0$或$1$,空白处全为$0$,则:
    \begin{align*}
        \bs{J}=\begin{pmatrix}
            0 &0 &0\\
            0 &0 &0\\
            0 &0 &0
        \end{pmatrix} \text{或} \ \bs{J}=\begin{pmatrix}
            0 &1 &0\\
            0 &0 &0\\
            0 &0 &0
        \end{pmatrix}\text{或} \ \bs{J}=\begin{pmatrix}
            0 &1 &0\\
            0 &0 &1\\
            0 &0 &0
        \end{pmatrix}
    \end{align*}
\end{solution}

\begin{question}(p64.10)
    求$3$阶幂等阵(即满足$\bs{A}^2=\bs{A}$的矩阵$\bs{A}$)的全部可能的Jordan标准形。
\end{question}

\begin{solution}
    由题意与Hamilton-Cayley定理可知,$\bs{A}^k=\bs{O} \Rightarrow \lambda_1=0,\lambda_2=1,m_{\bs{A}}(\lambda)|\lambda^2-\lambda$,
    且$\bs{A}$一定可以相似对角化。
    \begin{enumerate}[label=(\arabic*)]
        \item 若$m_{\bs{A}}(\lambda)=\lambda$,此时初等因子为$\lambda,\lambda,\lambda$,$\bs{J}=\begin{pmatrix}
            0 &0 &0\\
            0 &0 &0\\
            0 &0 &0
        \end{pmatrix}$。
        \item 若$m_{\bs{A}}(\lambda)=\lambda-1$,此时初等因子为$\lambda-1,\lambda-1,\lambda-1$,$\bs{J}=\begin{pmatrix}
            1 &0 &0\\
            0 &1 &0\\
            0 &0 &1
        \end{pmatrix}$。
        \item 若$m_{\bs{A}}(\lambda)=\lambda(\lambda-1)$,
        此时初等因子为$\lambda,\lambda,\lambda-1$或$\lambda,\lambda-1,\lambda-1$。$\bs{J}=\begin{pmatrix}
        0 &0 &0\\
        0 &0 &0\\
        0 &0 &1
    \end{pmatrix}$或$\bs{J}=\begin{pmatrix}
        0 &0 &0\\
        0 &1 &0\\
        0 &0 &1
    \end{pmatrix}$。
    \end{enumerate}
\end{solution}


\begin{question}(p64.11)
    设$3$阶矩阵$\bs{A}^2=\bs{E}$,求$\bs{A}$的全部可能的Jordan标准形。
\end{question}

\begin{solution}
    由题意与Hamilton-Cayley定理可知,$\bs{A}^2=\bs{E} \Rightarrow \lambda_1=1,\lambda_2=-1,m_{\bs{A}}(\lambda)|\lambda^2-1$,
    且$\bs{A}$一定可以相似对角化。
    \begin{enumerate}[label=(\arabic*)]
        \item 若$m_{\bs{A}}(\lambda)=\lambda+1$,此时初等因子为$\lambda+1,\lambda+1,\lambda+1$,$\bs{J}=\begin{pmatrix}
            -1 &0 &0\\
            0 &-1 &0\\
            0 &0 &-1
        \end{pmatrix}$。
        \item 若$m_{\bs{A}}(\lambda)=\lambda-1$,此时初等因子为$\lambda-1,\lambda-1,\lambda-1$,$\bs{J}=\begin{pmatrix}
            1 &0 &0\\
            0 &1 &0\\
            0 &0 &1
        \end{pmatrix}$。
        \item 若$m_{\bs{A}}(\lambda)=(\lambda+1)(\lambda-1)$,
        此时初等因子为$\lambda+1,\lambda+1,\lambda-1$或$\lambda+1,\lambda-1,\lambda-1$。$\bs{J}=\begin{pmatrix}
        -1 &0 &0\\
        0 &-1 &0\\
        0 &0 &1
    \end{pmatrix}$或$\bs{J}=\begin{pmatrix}
        -1 &0 &0\\
        0 &1 &0\\
        0 &0 &1
    \end{pmatrix}$。
    \end{enumerate}
\end{solution}

\begin{question}(p64.12)
    设$n$阶矩阵$\bs{A}^2=\bs{E}$,证明:$\bs{A}$相似于对角阵。
\end{question}

\begin{proof}
    由题意与Hamilton-Cayley定理可知,$\bs{A}^2=\bs{E} \Rightarrow \lambda_1=1,\lambda_2=-1,m_{\bs{A}}(\lambda)|\lambda^2-1$,
    $\lambda-1,\lambda+1$两个多项式的次数均为$1$,所以一定可以相似于对角阵。
\end{proof}

\begin{question}(p64.13)
    设$n$阶矩阵$\bs{A}^3-\bs{A}=10\bs{E}$,证明:$\bs{A}$相似于对角阵。
\end{question}

\begin{proof}
    由题意与Hamilton-Cayley定理可知,$\bs{A}^3-\bs{A}=10\bs{E} \Rightarrow m_{\bs{A}}(\lambda)|\lambda^3-\lambda-10$,
    下面探究$g(\lambda)=\lambda^3-\lambda-10$在复数域$\mathbb{C}$上根的分布。

    $g'(\lambda)=3\lambda^2-1$,所以$g(\lambda)$在$(-\infty,-\frac{1}{\sqrt{3}}]$上单调递增,
    在$(-\frac{1}{\sqrt{3}},\frac{1}{\sqrt{3}}]$单调递减,在$(\frac{1}{\sqrt{3}},+\infty)$上单调递增。
    且$g(-\frac{1}{\sqrt{3}})=\frac{2}{3\sqrt{3}}-10<0,g(\frac{1}{\sqrt{3}})=-\frac{2}{3\sqrt{3}}-10<0$,
    $\lim_{\lambda \to +\infty}=+\infty$,由零点定理,存在$\lambda_0 \in (\frac{1}{\sqrt{3}},+\infty)$使$g(\lambda_0)=0$。
    这表明了$g(\lambda)$可以被分解为$(\lambda-\lambda_0)(\lambda^2+a\lambda+b)$的形式,其中$a^2-4b <0$。
    
    进一步的,我们可以在复数域上将其分解为$(\lambda-\lambda_0)(\lambda-\lambda_1)(\lambda-\lambda_2)$的形式,
    并且由于$\lambda_1,\lambda_2$共轭,所以$\lambda_0\neq\lambda_1 \neq \lambda_2$。
    $\lambda-\lambda_0,\lambda-\lambda_1,\lambda-\lambda_2$三个多项式的次数均为$1$,所以$\bs{A}$相似于对角阵。

\end{proof}

\begin{question}(p65.19)
    设$\bs{A}$为$n$阶非零复方阵,$d=\mathrm{deg}m_{\bs{A}}(\lambda)$。
    \begin{enumerate}[label=(\arabic*)]
        \item 证明:对一切$n \times 1$的列矩阵$\bs{x}$,都有$\bs{x},\bs{A}\bs{x},\ldots,\bs{A}^{d}\bs{x}$线性相关。
        \item 证明:对一切正整数$k<d$,都存在列矩阵$\bs{x}$,使得$\bs{x},\bs{A}\bs{x},\ldots,\bs{A}^{k-1}\bs{x}$线性无关。\label{线性无关2}
        \item 当$\bs{A}$为实方阵时,是否存在实的列矩阵$\bs{x}$,使得\ref{线性无关2}成立?
    \end{enumerate}
\end{question}

\begin{proof}
    \begin{enumerate}[label=(\arabic*)]
        \item 由$\mathrm{deg}m_{\bs{A}}(\lambda)=d$,
        则$m_{\bs{A}}(\lambda)=a_1\lambda^d+a_2\lambda^{d-1}+\cdots+a_d\lambda+a_{d+1}$,其中$a_1 \neq 0$。
        由Hamilton-Cayley定理,有$a_1\bs{A}^d+a_2\bs{A}^{d-1}+\cdots+a_d\bs{A}+a_{d+1}=\bs{O}$,
        等式两边同乘列向量$\bs{x}$,则$a_1(\bs{A}^d\bs{x})+a_2(\bs{A}^{d-1}\bs{x})+\cdots+a_d(\bs{A}\bs{x})+a_{d+1}\bs{x}=\bs{0}$,
        且$a_1,a_2,\ldots,a_{d+1}$不全为$0$,这是线性相关的定义,由此便完成了证明。
        \item 使用反证法。若对一切正整数$k<d$,任取列矩阵$\bs{x}$,都有$\bs{x},\bs{A}\bs{x},\ldots,\bs{A}^{k-1}\bs{x}$线性相关,
        即若$a_0\bs{x}+a_1\bs{A}\bs{x}+\cdots+a_{k-1}\bs{A}^{k-1}\bs{x}=\bs{0}$成立,
        则$a_0,\ldots,a_{k-1}$不能全为$0$。

        由于$\mathrm{deg}m_{\bs{A}}(\lambda)=d$,则$\bs{A}$在复数域上至少有$d$个特征值$\lambda_i,\ldots,\lambda_d$。
        不妨设$\bs{x}$为$\bs{A}$的特征向量($\bs{x}\neq \bs{0}$),
        于是$\bs{A}^j\bs{x}=\lambda^j_i \bs{x}$,其中$\lambda_i \in \{\lambda_1,\ldots,\lambda_d\}$。
        此时$a_0\bs{x}+a_1\bs{A}\bs{x}+\cdots+a_{k-1}\bs{A}^{k-1}\bs{x}=
        (a_0+a_1\lambda_i+\cdots+a_{k-1}\lambda_i^{k-1})\bs{x}=\bs{0} \Rightarrow h(\lambda_i)=a_0+a_1\lambda_i+\cdots+a_{k-1}\lambda_i^{k-1}= 0$
        (注:以上的等式对于所有的$\lambda_i$都是成立的)。

        由于$h(\lambda)$在复数域上有且只有$k-1$个根,则一定存在某些$\lambda_i \in \{\lambda_1,\ldots,\lambda_d\}$使得$h(\lambda_i)\neq 0$,
        与$h(\lambda_i)=0$矛盾。所以只有当$a_0=\ldots=a_{k-1}=0$才能保证全部的$\lambda_i$有$h(\lambda_i)=0$成立,
        这说明了$\bs{x},\bs{A}\bs{x},\ldots,\bs{A}^{k-1}\bs{x}$线性无关,反证法推出矛盾,所以原命题成立。 \label{第二问解答}
        \item \textbf{结论}:不一定存在。(具体证明过程暂时不会)。
        % 由于$\bs{A}$与$\bs{x}$都是实的,那么$\lambda \in \R$。
        % 于是\ref{第二问解答}中的$h(\lambda_i)$放在实数域中讨论根的个数,
        % 只需要保证$\{\lambda_1,\ldots,\lambda_d\}$中的实数特征值
        % 又因为\ref{第二问解答}中存在某些$\lambda_i \notin R$,
        % $h(\lambda_i)$,因为有
    \end{enumerate}
\end{proof}


\ifx\allfiles\undefined
\end{document}
\fi