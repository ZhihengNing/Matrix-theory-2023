\ifx\allfiles\undefined
\documentclass[12pt, a4paper, oneside, UTF8]{ctexbook}
\def\configPath{../config}
\def\coverPath{\configPath/cover}
\def\packagePath{\configPath/package}
\def\theormPath{\configPath/theorem}
\def\customPath{\configPath/custom}
\def\prefacePath{\configPath/preface}

% 在这里定义需要的包
\usepackage{amsmath}
\usepackage{amsthm}
\usepackage{amssymb}
\usepackage{graphicx}
\usepackage{mathrsfs}
\usepackage{enumitem}
\usepackage{geometry}
\usepackage[colorlinks, linkcolor=black]{hyperref}
\usepackage{stackengine}
\usepackage{yhmath}
\usepackage{extarrows}
\usepackage{arydshln}
% \usepackage{unicode-math}
\usepackage{tasks}
\usepackage{fancyhdr}
\usepackage[dvipsnames, svgnames]{xcolor}
\usepackage{listings}


\input{\theormPath/theorem1_zh}
\input{\customPath/custom}




\begin{document}
\else
\fi

\chapter{广义逆矩阵}
\begin{question}(P162.3)
求下列矩阵的广义逆$\bs{A}^{+}$。
\begin{tasks}[label=(\arabic*)](3)
    \task $\begin{pmatrix}
        1 & -1 & 0\\
        -1&2&0\\
    \end{pmatrix}$
    \task $\begin{pmatrix}
        1 & -1&2\\
        1&0&0\\
        -1 & -1 &2\\
        -1&0&0
    \end{pmatrix}$
    \task $\begin{pmatrix}
        -2& 0&0&-2\\
        1&2&-4&3\\
        2 & -1 &2&1\\
        0&2&-4&2
    \end{pmatrix}$
\end{tasks}
\end{question}

   
\begin{solution}
    \begin{enumerate}[label=(\arabic*)]
        \item \begin{align*}
            \bs{A}=\bs{B}\bs{C}=\begin{pmatrix}
                1&-1\\
                -1&2
            \end{pmatrix}\begin{pmatrix}
                1&0&0\\
                0&1&0
            \end{pmatrix}
        \end{align*}
        \begin{align*}
            \bs{C}\bs{C}^T=\begin{pmatrix}
                1&0\\
                0&1
            \end{pmatrix},\bs{B}^T\bs{B}=\begin{pmatrix}
                2&-3\\
                -3&5
            \end{pmatrix}
        \end{align*}
        \begin{align*}
            \bs{A}^+&=\bs{C}^T(\bs{C}\bs{C}^T)^{-1}(\bs{B}^T\bs{B})^{-1}\bs{B}^T\\
            &=\begin{pmatrix}
                1& 0\\
                0&1\\
                0&0
            \end{pmatrix}\begin{pmatrix}
                1&0\\
                0&1
            \end{pmatrix}^{-1}\begin{pmatrix}
                2&-3\\
                -3&5
            \end{pmatrix}^{-1}\begin{pmatrix}
                1&-1\\
                -1&2
            \end{pmatrix}\\
            &=\begin{pmatrix}
                2&1\\
                1&1\\
                0&0
            \end{pmatrix}
        \end{align*}
    
        \item \begin{align*}
            \bs{A}=\bs{B}\bs{C}=\begin{pmatrix}
                1&-1\\
                1&0 \\
                -1&-1\\
                -1&0
            \end{pmatrix}\begin{pmatrix}
                1&0&0\\
                0&1&-2
            \end{pmatrix}
        \end{align*}
        \begin{align*}
            \bs{C}\bs{C}^T=\begin{pmatrix}
                1&0\\
                0&5
            \end{pmatrix},\bs{B}^T\bs{B}=\begin{pmatrix}
                4&0\\
                0&2
            \end{pmatrix}
        \end{align*}
        \begin{align*}
            \bs{A}^+&=\bs{C}^T(\bs{C}\bs{C}^T)^{-1}(\bs{B}^T\bs{B})^{-1}\bs{B}^T\\
            &=\begin{pmatrix}
                1& 0\\
                0&1\\
                0&-2
            \end{pmatrix}\begin{pmatrix}
                1&0\\
                0&5
            \end{pmatrix}^{-1}\begin{pmatrix}
                4&0\\
                0&2
            \end{pmatrix}^{-1}\begin{pmatrix}
                1&1&-1&-1\\
                -1&0&-1&0
            \end{pmatrix}\\
            &=\begin{pmatrix}
                \frac{1}{4}&\frac{1}{4}&-\frac{1}{4}&-\frac{1}{4}\\
                -\frac{1}{10}&0&-\frac{1}{10}&0\\
                \frac{1}{5}&0&\frac{1}{5}&0
            \end{pmatrix}
        \end{align*}
    
        \item \begin{align*}
            \bs{A}=\bs{B}\bs{C}=\begin{pmatrix}
                -2&0\\
                1&2\\
                2&-1\\
                0&2
            \end{pmatrix}\begin{pmatrix}
                1&0&0&1\\
                0&1&-2&1
            \end{pmatrix}
        \end{align*}
        \begin{align*}
            \bs{C}\bs{C}^T=\begin{pmatrix}
                2&1\\
                1&6
            \end{pmatrix},\bs{B}^T\bs{B}=\begin{pmatrix}
                9&0\\
                0&9
            \end{pmatrix}
        \end{align*}
        \begin{align*}
            \bs{A}^+&=\bs{C}^T(\bs{C}\bs{C}^T)^{-1}(\bs{B}^T\bs{B})^{-1}\bs{B}^T\\
            &=\begin{pmatrix}
                1& 0\\
                0&1\\
                0&-2 \\
                1&1
            \end{pmatrix}\begin{pmatrix}
                2&1\\
                1&6
            \end{pmatrix}^{-1}\begin{pmatrix}
                9&0\\
                0&9
            \end{pmatrix}^{-1}\begin{pmatrix}
                -2&1&2&0\\
                0&2&-1&2
            \end{pmatrix}\\
            &=\frac{1}{99}\begin{pmatrix}
                -12&4&13&-2\\
                2&3&-4&4\\
                -4&-6&8&-8\\
                -10&7&9&2
            \end{pmatrix}
        \end{align*}
    \end{enumerate}
    
\end{solution}

\begin{question}(P163.9)
    验证线性方程组$\bs{A}\bs{x}=\bs{b}$有解,并求其通解和最小长度解,其中$\bs{A}=\begin{pmatrix}
        1&2\\
        0&0\\
        2&4
    \end{pmatrix},\bs{b}=\begin{pmatrix}
        -1\\
        0\\
        -2
    \end{pmatrix}$
\end{question}

\begin{solution}
    $(\bs{A}|\bs{b})=\begin{pmatrix}
        1&2&-1\\
        0&0&0\\
        2&4&-2
    \end{pmatrix} \rightarrow \begin{pmatrix}
        1&2&-1\\
        0&0&0\\
        0&0&0
    \end{pmatrix}$,则$\mathrm{r}(\bs{A}|\bs{b})=\mathrm{r}(\bs{A})$,于是方程组有解。
    \begin{align*}
        \bs{A}=\bs{B}\bs{C}=\begin{pmatrix}
            1\\
            0\\
            2
        \end{pmatrix}\begin{pmatrix}
            1&2
        \end{pmatrix}
    \end{align*}
    \begin{align*}
        \bs{C}\bs{C}^T=\begin{pmatrix}
            5
        \end{pmatrix},\bs{B}^T\bs{B}=\begin{pmatrix}
            5
        \end{pmatrix}
    \end{align*}
    \begin{align*}
        \bs{A}^+&=\bs{C}^T(\bs{C}\bs{C}^T)^{-1}(\bs{B}^T\bs{B})^{-1}\bs{B}^T\\
        &=\begin{pmatrix}
            1\\
            2
        \end{pmatrix}\begin{pmatrix}
            5
        \end{pmatrix}^{-1}\begin{pmatrix}
            5
        \end{pmatrix}^{-1}\begin{pmatrix}
            1&0&2
        \end{pmatrix}\\
        &=\frac{1}{25}\begin{pmatrix}
            1&0&2\\
            2&0&4
        \end{pmatrix}
    \end{align*}
    则通解为$\bs{x}=\bs{A}^+\bs{b}+(\bs{E}-\bs{A}^+\bs{A})\bs{y}
    =\begin{pmatrix}
        -\frac{1}{5}+\frac{4}{5}y_1-\frac{2}{5}y_2\\
        -\frac{2}{5}-\frac{2}{5}y_1+\frac{1}{5}y_1
    \end{pmatrix}$,其中$\bs{y}\in \R^2$。最小长度解为$\bs{x}^{*}=\begin{pmatrix}
        -\frac{1}{5}\\
        -\frac{2}{5}
    \end{pmatrix}$。

\end{solution}

\begin{question}(P163.10)
    验证下列线性方程组$\bs{A}\bs{x}=\bs{b}$为矛盾方程组,并求其最小二乘解的通解和
    最小长度二乘解。
    \begin{tasks}[label=(\arabic*)](2)
        \task $\bs{A}=\begin{pmatrix}
            1&2\\
            0&0\\
            2&4
        \end{pmatrix},\bs{b}=\begin{pmatrix}
            1\\
            1\\
            2
        \end{pmatrix}$
        \task $\bs{A}=\begin{pmatrix}
            1&2&-1\\
            -3&-6&3
        \end{pmatrix},\bs{b}=\begin{pmatrix}
            1\\
            1
        \end{pmatrix}$
        \task $\bs{A}=\begin{pmatrix}
            1&1\\
            2&0\\
            -1&3
        \end{pmatrix},\bs{b}=\begin{pmatrix}
            1\\
            0\\
            2
        \end{pmatrix}$
    \end{tasks}
\end{question}

\begin{solution}
    \begin{enumerate}[label=(\arabic*)]
        \item  则$\mathrm{r}(\bs{A}|\bs{b})=2,\mathrm{r}(\bs{A})=1,\mathrm{r}(\bs{A}) \neq \mathrm{r}(\bs{A}|\bs{b})$,
        于是为矛盾方程组。
        \begin{align*}
            \bs{A}=\bs{B}\bs{C}=\begin{pmatrix}
                1\\
                0\\
                2
            \end{pmatrix}\begin{pmatrix}
                1&2
            \end{pmatrix}
        \end{align*}
        \begin{align*}
            \bs{C}\bs{C}^T=\begin{pmatrix}
                5
            \end{pmatrix},\bs{B}^T\bs{B}=\begin{pmatrix}
                5
            \end{pmatrix}
        \end{align*}
        \begin{align*}
            \bs{A}^+&=\bs{C}^T(\bs{C}\bs{C}^T)^{-1}(\bs{B}^T\bs{B})^{-1}\bs{B}^T\\
            &=\begin{pmatrix}
                1\\
                2
            \end{pmatrix}\begin{pmatrix}
                5
            \end{pmatrix}^{-1}\begin{pmatrix}
                5
            \end{pmatrix}^{-1}\begin{pmatrix}
                1&0&2
            \end{pmatrix}\\
            &=\frac{1}{25}\begin{pmatrix}
                1&0&2\\
                2&0&4
            \end{pmatrix}
        \end{align*}
        最小二乘解为$\bs{x}=\bs{A}^+\bs{b}+(\bs{E}-\bs{A}^+\bs{A})\bs{y}
        =\begin{pmatrix}
            \frac{1}{5}+\frac{4}{5}y_1-\frac{2}{5}y_2\\
            \frac{2}{5}-\frac{2}{5}y_1+\frac{1}{5}y_1
        \end{pmatrix}$,其中$\bs{y}\in \R^2$。最小长度二乘解为$\bs{x}^{*}=\begin{pmatrix}
            \frac{1}{5}\\
            \frac{2}{5}
        \end{pmatrix}$。

        \item  则$\mathrm{r}(\bs{A}|\bs{b})=2,\mathrm{r}(\bs{A})=1,\mathrm{r}(\bs{A}) \neq \mathrm{r}(\bs{A}|\bs{b})$,
        于是为矛盾方程组。
        \begin{align*}
            \bs{A}=\bs{B}\bs{C}=\begin{pmatrix}
                1\\
                -3
            \end{pmatrix}\begin{pmatrix}
                1&2&-1
            \end{pmatrix}
        \end{align*}
        \begin{align*}
            \bs{C}\bs{C}^T=\begin{pmatrix}
                6
            \end{pmatrix},\bs{B}^T\bs{B}=\begin{pmatrix}
                10
            \end{pmatrix}
        \end{align*}
        \begin{align*}
            \bs{A}^+&=\bs{C}^T(\bs{C}\bs{C}^T)^{-1}(\bs{B}^T\bs{B})^{-1}\bs{B}^T\\
            &=\begin{pmatrix}
                1\\
                2\\
                -1
            \end{pmatrix}\begin{pmatrix}
                6
            \end{pmatrix}^{-1}\begin{pmatrix}
                10
            \end{pmatrix}^{-1}\begin{pmatrix}
                1&-3
            \end{pmatrix}\\
            &=\frac{1}{60}\begin{pmatrix}
                1&-3\\
                2&-6\\
                -1&3
            \end{pmatrix}
        \end{align*}
        最小二乘解为$\bs{x}=\bs{A}^+\bs{b}+(\bs{E}-\bs{A}^+\bs{A})\bs{y}
        =\begin{pmatrix}
            -\frac{1}{30}+\frac{5}{6}y_1-\frac{1}{3}y_2+\frac{1}{6}y_3\\
            -\frac{1}{15}-\frac{1}{3}y_1+\frac{1}{3}y_2+\frac{1}{3}y_3\\
            \frac{1}{30}+\frac{1}{6}y_1+\frac{1}{3}y_2+\frac{5}{6}y_3
        \end{pmatrix}$,其中$\bs{y}\in \R^3$。最小长度二乘解为$\bs{x}^{*}=\begin{pmatrix}
            -\frac{1}{30}\\
            -\frac{1}{15}\\
            \frac{1}{30}
        \end{pmatrix}$。

        \item  则$\mathrm{r}(\bs{A}|\bs{b})=3,\mathrm{r}(\bs{A})=2,\mathrm{r}(\bs{A}) \neq \mathrm{r}(\bs{A}|\bs{b})$,
        于是为矛盾方程组。
        \begin{align*}
            \bs{A}=\bs{B}\bs{C}=\begin{pmatrix}
                1&1\\
                2&0\\
                -1&3
            \end{pmatrix}\begin{pmatrix}
                1&0\\
                0&1
            \end{pmatrix}
        \end{align*}
        \begin{align*}
            \bs{C}\bs{C}^T=\begin{pmatrix}
                1&0\\
                0&1
            \end{pmatrix},\bs{B}^T\bs{B}=\begin{pmatrix}
                6&-2\\
                -2&10
            \end{pmatrix}
        \end{align*}
        \begin{align*}
            \bs{A}^+&=\bs{C}^T(\bs{C}\bs{C}^T)^{-1}(\bs{B}^T\bs{B})^{-1}\bs{B}^T\\
            &=\begin{pmatrix}
                1&0\\
                0&1
            \end{pmatrix}\begin{pmatrix}
                1&0\\
                0&1
            \end{pmatrix}^{-1}\begin{pmatrix}
                6&-2\\
                -2&10
            \end{pmatrix}^{-1}\begin{pmatrix}
                1&2&-1\\
                1&0&3
            \end{pmatrix}\\
            &=\frac{1}{14}\begin{pmatrix}
                3&5&-1\\
                2&1&4
            \end{pmatrix}
        \end{align*}
        最小二乘解为$\bs{x}=\bs{A}^+\bs{b}+(\bs{E}-\bs{A}^+\bs{A})\bs{y}
        =\begin{pmatrix}
            \frac{1}{14}\\
            \frac{10}{14}\\
        \end{pmatrix}$,其中$\bs{y}\in \R^3$。最小长度二乘解为$\bs{x}^{*}=\begin{pmatrix}
            \frac{1}{14}\\
            \frac{10}{14}\\
        \end{pmatrix}$。
    \end{enumerate}

\end{solution}


\ifx\allfiles\undefined
\end{document}
\fi