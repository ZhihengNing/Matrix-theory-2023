\ifx\allfiles\undefined
\documentclass[12pt, a4paper, oneside, UTF8]{ctexbook}
\def\configPath{../config}
\def\basicPath{\configPath/basic}


% 在这里定义需要的包
\usepackage{amsmath}
\usepackage{amsthm}
\usepackage{amssymb}
\usepackage{graphicx}
\usepackage{mathrsfs}
\usepackage{enumitem}
\usepackage{geometry}
\usepackage[colorlinks, linkcolor=black]{hyperref}
\usepackage{stackengine}
\usepackage{yhmath}
\usepackage{extarrows}
\usepackage{arydshln}
% \usepackage{unicode-math}
\usepackage{tasks}
\usepackage{fancyhdr}
\usepackage[dvipsnames, svgnames]{xcolor}
\usepackage{listings}

\definecolor{mygreen}{rgb}{0,0.6,0}
\definecolor{mygray}{rgb}{0.5,0.5,0.5}
\definecolor{mymauve}{rgb}{0.58,0,0.82}

\graphicspath{ {figure/},{../figure/}, {config/}, {../config/} }

\linespread{1.6}

\geometry{
    top=25.4mm, 
    bottom=25.4mm, 
    left=20mm, 
    right=20mm, 
    headheight=2.17cm, 
    headsep=4mm, 
    footskip=12mm
}

\setenumerate[1]{itemsep=5pt,partopsep=0pt,parsep=\parskip,topsep=5pt}
\setitemize[1]{itemsep=5pt,partopsep=0pt,parsep=\parskip,topsep=5pt}
\setdescription{itemsep=5pt,partopsep=0pt,parsep=\parskip,topsep=5pt}

\lstset{
    language=Mathematica,
    basicstyle=\tt,
    breaklines=true,
    keywordstyle=\bfseries\color{NavyBlue}, 
    emphstyle=\bfseries\color{Rhodamine},
    commentstyle=\itshape\color{black!50!white}, 
    stringstyle=\bfseries\color{PineGreen!90!black},
    columns=flexible,
    numbers=left,
    numberstyle=\footnotesize,
    frame=tb,
    breakatwhitespace=false,
} 
% 在这里定义自己顺手的环境
\def\d{\mathrm{d}}
\def\i{\mathrm{i}}
\def\R{\mathbb{R}}
\newcommand{\bs}[1]{\boldsymbol{#1}}
\newcommand{\mc}[1]{\mathcal{#1}}
\newcommand{\ora}[1]{\overrightarrow{#1}}
\newcommand{\myspace}[1]{\par\vspace{#1\baselineskip}}
\newcommand{\xrowht}[2][0]{\addstackgap[.5\dimexpr#2\relax]{\vphantom{#1}}}
\newenvironment{ca}[1][1]{\linespread{#1} \selectfont \begin{cases}}{\end{cases}}
\newenvironment{vx}[1][1]{\linespread{#1} \selectfont \begin{vmatrix}}{\end{vmatrix}}
\newcommand{\tabincell}[2]{\begin{tabular}{@{}#1@{}}#2\end{tabular}}
\newcommand{\pll}{\kern 0.56em/\kern -0.8em /\kern 0.56em}
\newcommand{\dive}[1][F]{\mathrm{div}\;\bs{#1}}
\newcommand{\rotn}[1][A]{\mathrm{rot}\;\bs{#1}} 
\usepackage[strict]{changepage} 
\usepackage{framed}

\definecolor{greenshade}{rgb}{0.90,1,0.92}
\definecolor{redshade}{rgb}{1.00,0.88,0.88}
\definecolor{brownshade}{rgb}{0.99,0.95,0.9}
\definecolor{lilacshade}{rgb}{0.95,0.93,0.98}
\definecolor{orangeshade}{rgb}{1.00,0.88,0.82}
\definecolor{lightblueshade}{rgb}{0.8,0.92,1}
\definecolor{purple}{rgb}{0.81,0.85,1}
\theoremstyle{definition}
\newtheorem{myDefn}{\indent 定义}[section]
% \newtheorem{myLemma}{\indent 引理}[section]
\newtheorem{myLemma}{\indent 引理}[chapter]
\newtheorem{myThm}[myLemma]{\indent 定理}
\newtheorem{myCorollary}[myLemma]{\indent 推论}
\newtheorem{myCriterion}[myLemma]{\indent 准则}
\newtheorem*{myRemark}{\indent 注}
\newtheorem{myProposition}{\indent 命题}[section]


\newenvironment{formal}[2][]{%
    \def\FrameCommand{%
        \hspace{1pt}%
        {\color{#1}\vrule width 2pt}%
        {\color{#2}\vrule width 4pt}%
        \colorbox{#2}%
    }%
    \MakeFramed{\advance\hsize-\width\FrameRestore}%
    \noindent\hspace{-4.55pt}%
    \begin{adjustwidth}{}{7pt}\vspace{2pt}\vspace{2pt}}{%
        \vspace{2pt}\end{adjustwidth}\endMakeFramed%
}

\newenvironment{defn}{\begin{formal}[Green]{greenshade}\vspace{-\baselineskip / 2}\begin{myDefn}}{\end{myDefn}\end{formal}}
\newenvironment{thm}{\begin{formal}[LightSkyBlue]{lightblueshade}\vspace{-\baselineskip / 2}\begin{myThm}}{\end{myThm}\end{formal}}
\newenvironment{lemma}{\begin{formal}[Plum]{lilacshade}\vspace{-\baselineskip / 2}\begin{myLemma}}{\end{myLemma}\end{formal}}
\newenvironment{corollary}{\begin{formal}[BurlyWood]{brownshade}\vspace{-\baselineskip / 2}\begin{myCorollary}}{\end{myCorollary}\end{formal}}
\newenvironment{criterion}{\begin{formal}[DarkOrange]{orangeshade}\vspace{-\baselineskip / 2}\begin{myCriterion}}{\end{myCriterion}\end{formal}}
\newenvironment{rmk}{\begin{formal}[LightCoral]{redshade}\vspace{-\baselineskip / 2}\begin{myRemark}}{\end{myRemark}\end{formal}}
\newenvironment{proposition}{\begin{formal}[RoyalPurple]{purple}\vspace{-\baselineskip / 2}\begin{myProposition}}{\end{myProposition}\end{formal}}

\newtheorem{example}{\indent \color{SeaGreen}{例}}[section]
\newtheorem{question}{\color{SeaGreen}{题}}[chapter]
% \renewenvironment{proof}{\indent\textcolor{SkyBlue}{\textbf{证明.}}\;}{\qed\par}
% \newenvironment{solution}{\indent\textcolor{SkyBlue}{\textbf{解.}}\;}{\qed\par}

\renewcommand{\proofname}{\textbf{\textcolor{TealBlue}{证明}}}
\newenvironment{solution}{\begin{proof}[\textbf{\textcolor{TealBlue}{解}}]}{\end{proof}}

\definecolor{mygreen}{rgb}{0,0.6,0}
\definecolor{mygray}{rgb}{0.5,0.5,0.5}
\definecolor{mymauve}{rgb}{0.58,0,0.82}

\graphicspath{ {figure/},{../figure/}, {config/}, {../config/},{cover/graph} }

\linespread{1.6}

\geometry{
    top=25.4mm, 
    bottom=25.4mm, 
    left=20mm, 
    right=20mm, 
    headheight=2.17cm, 
    headsep=4mm, 
    footskip=12mm
}

\setenumerate[1]{itemsep=5pt,partopsep=0pt,parsep=\parskip,topsep=5pt}
\setitemize[1]{itemsep=5pt,partopsep=0pt,parsep=\parskip,topsep=5pt}
\setdescription{itemsep=5pt,partopsep=0pt,parsep=\parskip,topsep=5pt}

\lstset{
    language=Mathematica,
    basicstyle=\tt,
    breaklines=true,
    keywordstyle=\bfseries\color{NavyBlue}, 
    emphstyle=\bfseries\color{Rhodamine},
    commentstyle=\itshape\color{black!50!white}, 
    stringstyle=\bfseries\color{PineGreen!90!black},
    columns=flexible,
    numbers=left,
    numberstyle=\footnotesize,
    frame=tb,
    breakatwhitespace=false,
} 

\begin{document}
\else
\fi

\chapter{广义逆矩阵}
\begin{question}(P162.3)
求下列矩阵的广义逆$\bs{A}^{+}$。
\begin{tasks}[label=(\arabic*)](3)
    \task $\begin{pmatrix}
        1 & -1 & 0\\
        -1&2&0\\
    \end{pmatrix}$
    \task $\begin{pmatrix}
        1 & -1&2\\
        1&0&0\\
        -1 & -1 &2\\
        -1&0&0
    \end{pmatrix}$
    \task $\begin{pmatrix}
        -2& 0&0&-2\\
        1&2&-4&3\\
        2 & -1 &2&1\\
        0&2&-4&2
    \end{pmatrix}$
\end{tasks}
\end{question}

   
\begin{solution}
    \begin{enumerate}[label=(\arabic*)]
        \item \begin{align*}
            \bs{A}=\bs{B}\bs{C}=\begin{pmatrix}
                1&-1\\
                -1&2
            \end{pmatrix}\begin{pmatrix}
                1&0&0\\
                0&1&0
            \end{pmatrix}
        \end{align*}
        \begin{align*}
            \bs{C}\bs{C}^T=\begin{pmatrix}
                1&0\\
                0&1
            \end{pmatrix},\bs{B}^T\bs{B}=\begin{pmatrix}
                2&-3\\
                -3&5
            \end{pmatrix}
        \end{align*}
        \begin{align*}
            \bs{A}^+&=\bs{C}^T(\bs{C}\bs{C}^T)^{-1}(\bs{B}^T\bs{B})^{-1}\bs{B}^T\\
            &=\begin{pmatrix}
                1& 0\\
                0&1\\
                0&0
            \end{pmatrix}\begin{pmatrix}
                1&0\\
                0&1
            \end{pmatrix}^{-1}\begin{pmatrix}
                2&-3\\
                -3&5
            \end{pmatrix}^{-1}\begin{pmatrix}
                1&-1\\
                -1&2
            \end{pmatrix}\\
            &=\begin{pmatrix}
                2&1\\
                1&1\\
                0&0
            \end{pmatrix}
        \end{align*}
    
        \item \begin{align*}
            \bs{A}=\bs{B}\bs{C}=\begin{pmatrix}
                1&-1\\
                1&0 \\
                -1&-1\\
                -1&0
            \end{pmatrix}\begin{pmatrix}
                1&0&0\\
                0&1&-2
            \end{pmatrix}
        \end{align*}
        \begin{align*}
            \bs{C}\bs{C}^T=\begin{pmatrix}
                1&0\\
                0&5
            \end{pmatrix},\bs{B}^T\bs{B}=\begin{pmatrix}
                4&0\\
                0&2
            \end{pmatrix}
        \end{align*}
        \begin{align*}
            \bs{A}^+&=\bs{C}^T(\bs{C}\bs{C}^T)^{-1}(\bs{B}^T\bs{B})^{-1}\bs{B}^T\\
            &=\begin{pmatrix}
                1& 0\\
                0&1\\
                0&-2
            \end{pmatrix}\begin{pmatrix}
                1&0\\
                0&5
            \end{pmatrix}^{-1}\begin{pmatrix}
                4&0\\
                0&2
            \end{pmatrix}^{-1}\begin{pmatrix}
                1&1&-1&-1\\
                -1&0&-1&0
            \end{pmatrix}\\
            &=\begin{pmatrix}
                \frac{1}{4}&\frac{1}{4}&-\frac{1}{4}&-\frac{1}{4}\\
                -\frac{1}{10}&0&-\frac{1}{10}&0\\
                \frac{1}{5}&0&\frac{1}{5}&0
            \end{pmatrix}
        \end{align*}
    
        \item \begin{align*}
            \bs{A}=\bs{B}\bs{C}=\begin{pmatrix}
                -2&0\\
                1&2\\
                2&-1\\
                0&2
            \end{pmatrix}\begin{pmatrix}
                1&0&0&1\\
                0&1&-2&1
            \end{pmatrix}
        \end{align*}
        \begin{align*}
            \bs{C}\bs{C}^T=\begin{pmatrix}
                2&1\\
                1&6
            \end{pmatrix},\bs{B}^T\bs{B}=\begin{pmatrix}
                9&0\\
                0&9
            \end{pmatrix}
        \end{align*}
        \begin{align*}
            \bs{A}^+&=\bs{C}^T(\bs{C}\bs{C}^T)^{-1}(\bs{B}^T\bs{B})^{-1}\bs{B}^T\\
            &=\begin{pmatrix}
                1& 0\\
                0&1\\
                0&-2 \\
                1&1
            \end{pmatrix}\begin{pmatrix}
                2&1\\
                1&6
            \end{pmatrix}^{-1}\begin{pmatrix}
                9&0\\
                0&9
            \end{pmatrix}^{-1}\begin{pmatrix}
                -2&1&2&0\\
                0&2&-1&2
            \end{pmatrix}\\
            &=\frac{1}{99}\begin{pmatrix}
                -12&4&13&-2\\
                2&3&-4&4\\
                -4&-6&8&-8\\
                -10&7&9&2
            \end{pmatrix}
        \end{align*}
    \end{enumerate}
    
\end{solution}

\begin{question}
    验证线性方程组$\bs{A}\bs{x}=\bs{b}$有解,并求其通解和最小长度解,其中$\bs{A}=\begin{pmatrix}
        1&2\\
        0&0\\
        2&4
    \end{pmatrix},\bs{b}=\begin{pmatrix}
        -1\\
        0\\
        -2
    \end{pmatrix}$
\end{question}

\begin{solution}
    $(\bs{A}|\bs{b})=\begin{pmatrix}
        1&2&-1\\
        0&0&0\\
        2&4&-2
    \end{pmatrix} \rightarrow \begin{pmatrix}
        1&2&-1\\
        0&0&0\\
        0&0&0
    \end{pmatrix}$,则$\mathrm{r}(\bs{A}|\bs{b})=\mathrm{r}(\bs{A})$,于是方程组有解。
    \begin{align*}
        \bs{A}=\bs{B}\bs{C}=\begin{pmatrix}
            1\\
            0\\
            2
        \end{pmatrix}\begin{pmatrix}
            1&2
        \end{pmatrix}
    \end{align*}
    \begin{align*}
        \bs{C}\bs{C}^T=\begin{pmatrix}
            5
        \end{pmatrix},\bs{B}^T\bs{B}=\begin{pmatrix}
            5
        \end{pmatrix}
    \end{align*}
    \begin{align*}
        \bs{A}^+&=\bs{C}^T(\bs{C}\bs{C}^T)^{-1}(\bs{B}^T\bs{B})^{-1}\bs{B}^T\\
        &=\begin{pmatrix}
            1\\
            2
        \end{pmatrix}\begin{pmatrix}
            5
        \end{pmatrix}^{-1}\begin{pmatrix}
            5
        \end{pmatrix}^{-1}\begin{pmatrix}
            1&0&2
        \end{pmatrix}\\
        &=\frac{1}{25}\begin{pmatrix}
            1&0&2\\
            2&0&4
        \end{pmatrix}
    \end{align*}
    则通解为$\bs{x}=\bs{A}^+\bs{b}+(\bs{E}-\bs{A}^+\bs{A})\bs{y}
    =\begin{pmatrix}
        -\frac{1}{5}+\frac{4}{5}y_1-\frac{2}{5}y_2\\
        -\frac{2}{5}-\frac{2}{5}y_1+\frac{1}{5}y_1
    \end{pmatrix}$,其中$\bs{y}\in \R^2$。最小长度解为$\bs{x}^{*}=\begin{pmatrix}
        -\frac{1}{5}\\
        -\frac{2}{5}
    \end{pmatrix}$。

\end{solution}

\begin{question}
    验证下列线性方程组$\bs{A}\bs{x}=\bs{b}$为矛盾方程组,并求其最小二乘解的通解和
    最小长度二乘解。
    \begin{tasks}[label=(\arabic*)](2)
        \task $\bs{A}=\begin{pmatrix}
            1&2\\
            0&0\\
            2&4
        \end{pmatrix},\bs{b}=\begin{pmatrix}
            1\\
            1\\
            2
        \end{pmatrix}$
        \task $\bs{A}=\begin{pmatrix}
            1&2&-1\\
            -3&-6&3
        \end{pmatrix},\bs{b}=\begin{pmatrix}
            1\\
            1
        \end{pmatrix}$
        \task $\bs{A}=\begin{pmatrix}
            1&1\\
            2&0\\
            -1&3
        \end{pmatrix},\bs{b}=\begin{pmatrix}
            1\\
            0\\
            2
        \end{pmatrix}$
    \end{tasks}
\end{question}

\begin{solution}
    \begin{enumerate}[label=(\arabic*)]
        \item  则$\mathrm{r}(\bs{A}|\bs{b})=2,\mathrm{r}(\bs{A})=1,\mathrm{r}(\bs{A}) \neq \mathrm{r}(\bs{A}|\bs{b})$,
        于是为矛盾方程组。
        \begin{align*}
            \bs{A}=\bs{B}\bs{C}=\begin{pmatrix}
                1\\
                0\\
                2
            \end{pmatrix}\begin{pmatrix}
                1&2
            \end{pmatrix}
        \end{align*}
        \begin{align*}
            \bs{C}\bs{C}^T=\begin{pmatrix}
                5
            \end{pmatrix},\bs{B}^T\bs{B}=\begin{pmatrix}
                5
            \end{pmatrix}
        \end{align*}
        \begin{align*}
            \bs{A}^+&=\bs{C}^T(\bs{C}\bs{C}^T)^{-1}(\bs{B}^T\bs{B})^{-1}\bs{B}^T\\
            &=\begin{pmatrix}
                1\\
                2
            \end{pmatrix}\begin{pmatrix}
                5
            \end{pmatrix}^{-1}\begin{pmatrix}
                5
            \end{pmatrix}^{-1}\begin{pmatrix}
                1&0&2
            \end{pmatrix}\\
            &=\frac{1}{25}\begin{pmatrix}
                1&0&2\\
                2&0&4
            \end{pmatrix}
        \end{align*}
        最小二乘解为$\bs{x}=\bs{A}^+\bs{b}+(\bs{E}-\bs{A}^+\bs{A})\bs{y}
        =\begin{pmatrix}
            \frac{1}{5}+\frac{4}{5}y_1-\frac{2}{5}y_2\\
            \frac{2}{5}-\frac{2}{5}y_1+\frac{1}{5}y_1
        \end{pmatrix}$,其中$\bs{y}\in \R^2$。最小长度二乘解为$\bs{x}^{*}=\begin{pmatrix}
            \frac{1}{5}\\
            \frac{2}{5}
        \end{pmatrix}$。

        \item  则$\mathrm{r}(\bs{A}|\bs{b})=2,\mathrm{r}(\bs{A})=1,\mathrm{r}(\bs{A}) \neq \mathrm{r}(\bs{A}|\bs{b})$,
        于是为矛盾方程组。
        \begin{align*}
            \bs{A}=\bs{B}\bs{C}=\begin{pmatrix}
                1\\
                -3
            \end{pmatrix}\begin{pmatrix}
                1&2&-1
            \end{pmatrix}
        \end{align*}
        \begin{align*}
            \bs{C}\bs{C}^T=\begin{pmatrix}
                6
            \end{pmatrix},\bs{B}^T\bs{B}=\begin{pmatrix}
                10
            \end{pmatrix}
        \end{align*}
        \begin{align*}
            \bs{A}^+&=\bs{C}^T(\bs{C}\bs{C}^T)^{-1}(\bs{B}^T\bs{B})^{-1}\bs{B}^T\\
            &=\begin{pmatrix}
                1\\
                2\\
                -1
            \end{pmatrix}\begin{pmatrix}
                6
            \end{pmatrix}^{-1}\begin{pmatrix}
                10
            \end{pmatrix}^{-1}\begin{pmatrix}
                1&-3
            \end{pmatrix}\\
            &=\frac{1}{60}\begin{pmatrix}
                1&-3\\
                2&-6\\
                -1&3
            \end{pmatrix}
        \end{align*}
        最小二乘解为$\bs{x}=\bs{A}^+\bs{b}+(\bs{E}-\bs{A}^+\bs{A})\bs{y}
        =\begin{pmatrix}
            -\frac{1}{30}+\frac{5}{6}y_1-\frac{1}{3}y_2+\frac{1}{6}y_3\\
            -\frac{1}{15}-\frac{1}{3}y_1-\frac{1}{3}y_2+\frac{1}{3}y_3\\
            \frac{1}{30}+\frac{1}{6}y_1+\frac{1}{3}y_2+\frac{5}{6}y_3
        \end{pmatrix}$,其中$\bs{y}\in \R^3$。最小长度二乘解为$\bs{x}^{*}=\begin{pmatrix}
            -\frac{1}{30}\\
            -\frac{1}{15}\\
            \frac{1}{30}
        \end{pmatrix}$。

        \item  则$\mathrm{r}(\bs{A}|\bs{b})=3,\mathrm{r}(\bs{A})=2,\mathrm{r}(\bs{A}) \neq \mathrm{r}(\bs{A}|\bs{b})$,
        于是为矛盾方程组。
        \begin{align*}
            \bs{A}=\bs{B}\bs{C}=\begin{pmatrix}
                1&1\\
                2&0\\
                -1&3
            \end{pmatrix}\begin{pmatrix}
                1&0\\
                0&1
            \end{pmatrix}
        \end{align*}
        \begin{align*}
            \bs{C}\bs{C}^T=\begin{pmatrix}
                1&0\\
                0&1
            \end{pmatrix},\bs{B}^T\bs{B}=\begin{pmatrix}
                6&-2\\
                -2&10
            \end{pmatrix}
        \end{align*}
        \begin{align*}
            \bs{A}^+&=\bs{C}^T(\bs{C}\bs{C}^T)^{-1}(\bs{B}^T\bs{B})^{-1}\bs{B}^T\\
            &=\begin{pmatrix}
                1&0\\
                0&1
            \end{pmatrix}\begin{pmatrix}
                1&0\\
                0&1
            \end{pmatrix}^{-1}\begin{pmatrix}
                6&-2\\
                -2&10
            \end{pmatrix}^{-1}\begin{pmatrix}
                1&2&-1\\
                1&0&3
            \end{pmatrix}\\
            &=\frac{1}{14}\begin{pmatrix}
                3&5&-1\\
                2&1&4
            \end{pmatrix}
        \end{align*}
        最小二乘解为$\bs{x}=\bs{A}^+\bs{b}+(\bs{E}-\bs{A}^+\bs{A})\bs{y}
        =\begin{pmatrix}
            \frac{1}{14}\\
            \frac{10}{14}\\
        \end{pmatrix}$,其中$\bs{y}\in \R^3$。最小长度二乘解为$\bs{x}^{*}=\begin{pmatrix}
            \frac{1}{14}\\
            \frac{10}{14}\\
        \end{pmatrix}$。
    \end{enumerate}

\end{solution}



\ifx\allfiles\undefined
\end{document}
\fi