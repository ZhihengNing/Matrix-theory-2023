\ifx\allfiles\undefined
\documentclass[12pt, a4paper, oneside, UTF8]{ctexbook}
\def\configPath{../config}
\def\basicPath{\configPath/basic}


% 在这里定义需要的包
\usepackage{amsmath}
\usepackage{amsthm}
\usepackage{amssymb}
\usepackage{graphicx}
\usepackage{mathrsfs}
\usepackage{enumitem}
\usepackage{geometry}
\usepackage[colorlinks, linkcolor=black]{hyperref}
\usepackage{stackengine}
\usepackage{yhmath}
\usepackage{extarrows}
\usepackage{arydshln}
% \usepackage{unicode-math}
\usepackage{tasks}
\usepackage{fancyhdr}
\usepackage[dvipsnames, svgnames]{xcolor}
\usepackage{listings}

\definecolor{mygreen}{rgb}{0,0.6,0}
\definecolor{mygray}{rgb}{0.5,0.5,0.5}
\definecolor{mymauve}{rgb}{0.58,0,0.82}

\graphicspath{ {figure/},{../figure/}, {config/}, {../config/} }

\linespread{1.6}

\geometry{
    top=25.4mm, 
    bottom=25.4mm, 
    left=20mm, 
    right=20mm, 
    headheight=2.17cm, 
    headsep=4mm, 
    footskip=12mm
}

\setenumerate[1]{itemsep=5pt,partopsep=0pt,parsep=\parskip,topsep=5pt}
\setitemize[1]{itemsep=5pt,partopsep=0pt,parsep=\parskip,topsep=5pt}
\setdescription{itemsep=5pt,partopsep=0pt,parsep=\parskip,topsep=5pt}

\lstset{
    language=Mathematica,
    basicstyle=\tt,
    breaklines=true,
    keywordstyle=\bfseries\color{NavyBlue}, 
    emphstyle=\bfseries\color{Rhodamine},
    commentstyle=\itshape\color{black!50!white}, 
    stringstyle=\bfseries\color{PineGreen!90!black},
    columns=flexible,
    numbers=left,
    numberstyle=\footnotesize,
    frame=tb,
    breakatwhitespace=false,
} 
% 在这里定义自己顺手的环境
\def\d{\mathrm{d}}
\def\i{\mathrm{i}}
\def\R{\mathbb{R}}
\newcommand{\bs}[1]{\boldsymbol{#1}}
\newcommand{\mc}[1]{\mathcal{#1}}
\newcommand{\ora}[1]{\overrightarrow{#1}}
\newcommand{\myspace}[1]{\par\vspace{#1\baselineskip}}
\newcommand{\xrowht}[2][0]{\addstackgap[.5\dimexpr#2\relax]{\vphantom{#1}}}
\newenvironment{ca}[1][1]{\linespread{#1} \selectfont \begin{cases}}{\end{cases}}
\newenvironment{vx}[1][1]{\linespread{#1} \selectfont \begin{vmatrix}}{\end{vmatrix}}
\newcommand{\tabincell}[2]{\begin{tabular}{@{}#1@{}}#2\end{tabular}}
\newcommand{\pll}{\kern 0.56em/\kern -0.8em /\kern 0.56em}
\newcommand{\dive}[1][F]{\mathrm{div}\;\bs{#1}}
\newcommand{\rotn}[1][A]{\mathrm{rot}\;\bs{#1}} 
\usepackage[strict]{changepage} 
\usepackage{framed}

\definecolor{greenshade}{rgb}{0.90,1,0.92}
\definecolor{redshade}{rgb}{1.00,0.88,0.88}
\definecolor{brownshade}{rgb}{0.99,0.95,0.9}
\definecolor{lilacshade}{rgb}{0.95,0.93,0.98}
\definecolor{orangeshade}{rgb}{1.00,0.88,0.82}
\definecolor{lightblueshade}{rgb}{0.8,0.92,1}
\definecolor{purple}{rgb}{0.81,0.85,1}
\theoremstyle{definition}
\newtheorem{myDefn}{\indent 定义}[section]
% \newtheorem{myLemma}{\indent 引理}[section]
\newtheorem{myLemma}{\indent 引理}[chapter]
\newtheorem{myThm}[myLemma]{\indent 定理}
\newtheorem{myCorollary}[myLemma]{\indent 推论}
\newtheorem{myCriterion}[myLemma]{\indent 准则}
\newtheorem*{myRemark}{\indent 注}
\newtheorem{myProposition}{\indent 命题}[section]


\newenvironment{formal}[2][]{%
    \def\FrameCommand{%
        \hspace{1pt}%
        {\color{#1}\vrule width 2pt}%
        {\color{#2}\vrule width 4pt}%
        \colorbox{#2}%
    }%
    \MakeFramed{\advance\hsize-\width\FrameRestore}%
    \noindent\hspace{-4.55pt}%
    \begin{adjustwidth}{}{7pt}\vspace{2pt}\vspace{2pt}}{%
        \vspace{2pt}\end{adjustwidth}\endMakeFramed%
}

\newenvironment{defn}{\begin{formal}[Green]{greenshade}\vspace{-\baselineskip / 2}\begin{myDefn}}{\end{myDefn}\end{formal}}
\newenvironment{thm}{\begin{formal}[LightSkyBlue]{lightblueshade}\vspace{-\baselineskip / 2}\begin{myThm}}{\end{myThm}\end{formal}}
\newenvironment{lemma}{\begin{formal}[Plum]{lilacshade}\vspace{-\baselineskip / 2}\begin{myLemma}}{\end{myLemma}\end{formal}}
\newenvironment{corollary}{\begin{formal}[BurlyWood]{brownshade}\vspace{-\baselineskip / 2}\begin{myCorollary}}{\end{myCorollary}\end{formal}}
\newenvironment{criterion}{\begin{formal}[DarkOrange]{orangeshade}\vspace{-\baselineskip / 2}\begin{myCriterion}}{\end{myCriterion}\end{formal}}
\newenvironment{rmk}{\begin{formal}[LightCoral]{redshade}\vspace{-\baselineskip / 2}\begin{myRemark}}{\end{myRemark}\end{formal}}
\newenvironment{proposition}{\begin{formal}[RoyalPurple]{purple}\vspace{-\baselineskip / 2}\begin{myProposition}}{\end{myProposition}\end{formal}}

\newtheorem{example}{\indent \color{SeaGreen}{例}}[section]
\newtheorem{question}{\color{SeaGreen}{题}}[chapter]
% \renewenvironment{proof}{\indent\textcolor{SkyBlue}{\textbf{证明.}}\;}{\qed\par}
% \newenvironment{solution}{\indent\textcolor{SkyBlue}{\textbf{解.}}\;}{\qed\par}

\renewcommand{\proofname}{\textbf{\textcolor{TealBlue}{证明}}}
\newenvironment{solution}{\begin{proof}[\textbf{\textcolor{TealBlue}{解}}]}{\end{proof}}

\definecolor{mygreen}{rgb}{0,0.6,0}
\definecolor{mygray}{rgb}{0.5,0.5,0.5}
\definecolor{mymauve}{rgb}{0.58,0,0.82}

\graphicspath{ {figure/},{../figure/}, {config/}, {../config/},{cover/graph} }

\linespread{1.6}

\geometry{
    top=25.4mm, 
    bottom=25.4mm, 
    left=20mm, 
    right=20mm, 
    headheight=2.17cm, 
    headsep=4mm, 
    footskip=12mm
}

\setenumerate[1]{itemsep=5pt,partopsep=0pt,parsep=\parskip,topsep=5pt}
\setitemize[1]{itemsep=5pt,partopsep=0pt,parsep=\parskip,topsep=5pt}
\setdescription{itemsep=5pt,partopsep=0pt,parsep=\parskip,topsep=5pt}

\lstset{
    language=Mathematica,
    basicstyle=\tt,
    breaklines=true,
    keywordstyle=\bfseries\color{NavyBlue}, 
    emphstyle=\bfseries\color{Rhodamine},
    commentstyle=\itshape\color{black!50!white}, 
    stringstyle=\bfseries\color{PineGreen!90!black},
    columns=flexible,
    numbers=left,
    numberstyle=\footnotesize,
    frame=tb,
    breakatwhitespace=false,
} 

\begin{document}
\else
\fi

\chapter{矩阵分解}
\begin{question}(p140.1)
计算下列矩阵的Doolittle分解,Crout三角分解和LDU三角分解。
\begin{tasks}[label=(\arabic*)](2)
    \task $\begin{pmatrix}
        2 & 4 & 6\\
        2&7&12\\
        -2 & -10& -13
    \end{pmatrix}$
    \task $\begin{pmatrix}
        4 & 8&0\\
        4& 11 & 6\\
        -6 & -12 &10
    \end{pmatrix}$
\end{tasks}
\end{question}

\begin{solution}
    \begin{enumerate}[label=(\arabic*)]
        \item 先对矩阵进行初等行变换变成上三角矩阵
            \begin{align*}
                \begin{pmatrix}
                    2 & 4 & 6\\
                    2&7&12\\
                    -2 & -10& -13
                \end{pmatrix}\xrightarrow{r_2-r_1}
                \begin{pmatrix}
                    2 & 4 & 6\\
                    0&3&6\\
                    -2 & -10& -13
                \end{pmatrix}\xrightarrow{r_3+r_1}
                \begin{pmatrix}
                    2 & 4 & 6\\
                    0&3&6\\
                    0 & -6& -7
                \end{pmatrix}\xrightarrow{r_3+2r_2}
                \begin{pmatrix}
                    2 & 4 & 6\\
                    0&3&6\\
                    0 & 0& 5
                \end{pmatrix}
            \end{align*}
            则$\bs{R}$为上面所求,
            将每一步变换用初等矩阵表示,则$\bs{P}_3\bs{P}_2\bs{P}_1\bs{A}=\bs{R}$,
            于是
            \begin{align*}
                \bs{L}=(\bs{P}_3\bs{P}_2\bs{P}_1)^{-1}
                =\bs{P}_1^{-1}\bs{P}_2^{-1}\bs{P}_3^{-1}
                =\begin{pmatrix}
                    1& 0 &0\\
                    1&1&0\\
                    0&0&1
                \end{pmatrix}
                \begin{pmatrix}
                    1& 0 &0\\
                    0&1&0\\
                    -1&0&1
                \end{pmatrix}
                \begin{pmatrix}
                    1& 0 &0\\
                    0&1&0\\
                    0&-2&1
                \end{pmatrix}
                =\begin{pmatrix}
                    1& 0 &0\\
                    1&1&0\\
                    -1&-2&1
                \end{pmatrix}
            \end{align*}
            Doolittle分解、Crout分解、LDU分解
        \begin{align*}
            \bs{L}\bs{R}&=
            \begin{pmatrix}
                1& 0 &0\\
                1&1&0\\
                -1&-2&1
            \end{pmatrix}\begin{pmatrix}
                2 & 4 & 6\\
                0&3&6\\
                0 & 0& 5
            \end{pmatrix}\\
            \bs{L}\bs{D}\bs{U}&=\begin{pmatrix}
                1& 0 &0\\
                1&1&0\\
                -1&-2&1
            \end{pmatrix}
            \begin{pmatrix}
                2 & & \\
                & 3& \\
                & & 5
            \end{pmatrix}
            \begin{pmatrix}
                1 & 2 & 3\\
                0&1&2\\
                0 & 0& 1
            \end{pmatrix}\\
            \bs{L}\bs{U}&=
            \begin{pmatrix}
                2 & 0 & 0\\
                2&3&0\\
                -2 & -6& 5
            \end{pmatrix}
            \begin{pmatrix}
                1 & 2 & 3\\
                0&1&2\\
                0 & 0& 1
            \end{pmatrix}
        \end{align*}
        其中空白位置全是$0$。
    \item 先对矩阵进行初等行变换变成上三角矩阵
    \begin{align*}
        \begin{pmatrix}
        4 & 8&0\\
        4& 11 & 6\\
        -6 & -12 &10
        \end{pmatrix}\xrightarrow{r_2-r_1}
        \begin{pmatrix}
        4 & 8&0\\
        0& 3 & 6\\
        -6 & -12 &10
        \end{pmatrix}\xrightarrow{r_3+\frac{3}{2}r_1}
        \begin{pmatrix}
        4 & 8&0\\
        0& 3 & 6\\
        0 & 0 &10
        \end{pmatrix}
    \end{align*}
    则$\bs{R}$为上面所求,
    将每一步变换用初等矩阵表示,则$\bs{P}_2\bs{P}_1\bs{A}=\bs{R}$,
    于是
    \begin{align*}
        \bs{L}=(\bs{P}_2\bs{P}_1)^{-1}
        =\bs{P}_1^{-1}\bs{P}_2^{-1}
        =\begin{pmatrix}
            1& 0 &0\\
            1&1&0\\
            0&0&1
        \end{pmatrix}
        \begin{pmatrix}
            1& 0 &0\\
            0&1&0\\
            -\frac{3}{2}&0&1
        \end{pmatrix}
        =\begin{pmatrix}
            1& 0 &0\\
            1&1&0\\
            -\frac{3}{2}&0&1
        \end{pmatrix}
    \end{align*}
    Doolittle分解、Crout分解、LDU分解
\begin{align*}
    \bs{L}\bs{R}&=
    \begin{pmatrix}
        1& 0 &0\\
        1&1&0\\
        -\frac{3}{2}&0&1
    \end{pmatrix}\begin{pmatrix}
        4 & 8&0\\
        0& 3 & 6\\
        0 & 0 &10
    \end{pmatrix}\\
    \bs{L}\bs{D}\bs{U}&=\begin{pmatrix}
        1& 0 &0\\
        1&1&0\\
        -\frac{3}{2}&0&1
    \end{pmatrix}
    \begin{pmatrix}
        4 & & \\
        & 3& \\
        & & 10
    \end{pmatrix}
    \begin{pmatrix}
        1 & 2 & 0\\
        0&1&2\\
        0 & 0& 1
    \end{pmatrix}\\
    \bs{L}\bs{U}&=
    \begin{pmatrix}
        4 & 0 & 0\\
        4&3&0\\
        -6 & 0& 10
    \end{pmatrix}
    \begin{pmatrix}
        1 & 2 & 0\\
        0&1&2\\
        0 & 0& 1
    \end{pmatrix}
\end{align*}
其中空白位置全是$0$。
    \end{enumerate}
\end{solution}

\begin{question}(p141.2)
    计算下列矩阵的Cholesky分解。
    \begin{tasks}[label=(\arabic*)](2)
        \task $\begin{pmatrix}
            1 & 1 & -1\\
            1&2&-3\\
            -1 & -3& 6
        \end{pmatrix}$
        \task $\begin{pmatrix}
            4 & 4&-6\\
            4& 5 & -6\\
            -6 & -6 &13
        \end{pmatrix}$
    \end{tasks}
\end{question}

\begin{solution}
    \begin{enumerate}[label=(\arabic*)]
        \item 设$\bs{G}=\begin{pmatrix}
            g_{11}& &\\
            g_{21}&g_{22}&\\
            g_{31}&g_{32}&g_{33}
        \end{pmatrix}$,则$\bs{G}^T=\begin{pmatrix}
            g_{11}& g_{21}&g_{31}\\
            &g_{22}&g_{32}\\
            & &g_{33}
        \end{pmatrix}$,其中空白位置全是$0$。
        于是有:
        \begin{align*}
            \left\{
                \begin{array}{ll}
                    g_{11}^2=1\\
                    g_{11}g_{21}=1\\
                    g_{11}g_{31}=-1\\
                    g_{21}g_{11}=1\\
                    g_{21}^2+g_{22}^2=2\\
                    g_{21}g_{31}+g_{22}g_{32}=-3\\
                    g_{31}g_{11}=-1\\
                    g_{31}g_{21}+g_{32}g_{22}=-3\\
                    g_{31}^2+g_{32}^2+g_{33}^2=6
                \end{array}
                \right.
                \Rightarrow
            \left\{
                \begin{array}{ll}
                    g_{11}=1\\
                    g_{21}=1\\
                    g_{31}=-1\\
                    g_{22}=1\\
                    g_{32}=-2\\
                    g_{33}=1
                \end{array}
                \right.
        \end{align*}
        $\bs{G}=\begin{pmatrix}
            1 & 0&0\\
            1& 1&0\\
            -1& -2&1
        \end{pmatrix}\quad \bs{A}=\bs{G}\bs{G}^T=\begin{pmatrix}
            1 & 0&0\\
            1& 1&0\\
            -1& -2&1
        \end{pmatrix}\begin{pmatrix}
            1 &1 &-1\\
            0& 1&-2\\
            0& 0&1
        \end{pmatrix}$
        \item 设$\bs{G}=\begin{pmatrix}
            g_{11}& &\\
            g_{21}&g_{22}&\\
            g_{31}&g_{32}&g_{33}
        \end{pmatrix}$,则$\bs{G}^T=\begin{pmatrix}
            g_{11}& g_{21}&g_{31}\\
            &g_{22}&g_{32}\\
            & &g_{33}
        \end{pmatrix}$,其中空白位置全是$0$。
        于是有:
        \begin{align*}
            \left\{
                \begin{array}{ll}
                    g_{11}^2=4\\
                    g_{11}g_{21}=4\\
                    g_{11}g_{31}=-6\\
                    g_{21}g_{11}=4\\
                    g_{21}^2+g_{22}^2=5\\
                    g_{21}g_{31}+g_{22}g_{32}=-6\\
                    g_{31}g_{11}=-6\\
                    g_{31}g_{21}+g_{32}g_{22}=-6\\
                    g_{31}^2+g_{32}^2+g_{33}^2=13
                \end{array}
                \right.
                \Rightarrow
            \left\{
                \begin{array}{ll}
                    g_{11}=2\\
                    g_{21}=2\\
                    g_{31}=-3\\
                    g_{22}=1\\
                    g_{32}=0\\
                    g_{33}=2
                \end{array}
                \right.
        \end{align*}
        $\bs{G}=\begin{pmatrix}
            2 &0 &0\\
            2& 1&0\\
            -3& 0&2
        \end{pmatrix}\quad \bs{A}=\bs{G}\bs{G}^T=\begin{pmatrix}
            2 &0 &0\\
            2& 1&0\\
            -3& 0&2
        \end{pmatrix}\begin{pmatrix}
            2&2&-3\\
            0&1&0\\
            0&0&2
        \end{pmatrix}$
    \end{enumerate}
\end{solution}

\begin{question}(p141.3)
    计算矩阵$\begin{pmatrix}
        2&4&6&8\\
        4&12&20&16\\
        3&10&20&18\\
        1&-4&-9&14
    \end{pmatrix}$的Doolittle分解。
\end{question}


\begin{solution}
    先对矩阵进行初等行变换变成上三角矩阵
            \begin{align*}
                &\begin{pmatrix}
                    2&4&6&8\\
                    4&12&20&16\\
                    3&10&20&18\\
                    1&-4&-9&14
                \end{pmatrix}\xrightarrow{r_2-2r_1}
                \begin{pmatrix}
                    2&4&6&8\\
                    0&4&8&0\\
                    3&10&20&18\\
                    1&-4&-9&14
                \end{pmatrix}\xrightarrow{r_3-\frac{3}{2}r_1}
                \begin{pmatrix}
                    2&4&6&8\\
                    0&4&8&0\\
                    0&4&11&6\\
                    1&-4&-9&14
                \end{pmatrix}\xrightarrow{r_4-\frac{1}{2}r_1}
                \begin{pmatrix}
                    2&4&6&8\\
                    0&4&8&0\\
                    0&4&11&6\\
                    0&-6&-12&10
                \end{pmatrix}\\
                &\xrightarrow{r_3-r_2}
                \begin{pmatrix}
                    2&4&6&8\\
                    0&4&8&0\\
                    0&0&3&6\\
                    0&-6&-12&10
                \end{pmatrix}\xrightarrow{r_4+\frac{3}{2}r_2}
                \begin{pmatrix}
                    2&4&6&8\\
                    0&4&8&0\\
                    0&0&3&6\\
                    0&0&0&10
                \end{pmatrix}
            \end{align*}
            则$\bs{R}$为上面所求,
            将每一步变换用初等矩阵表示,则$\bs{P}_5\bs{P}_4\bs{P}_3\bs{P}_2\bs{P}_1\bs{A}=\bs{R}$,
            于是
            \begin{align*}
                \bs{L}&=(\bs{P}_5\bs{P}_4\bs{P}_3\bs{P}_2\bs{P}_1)^{-1}
                =\bs{P}_1^{-1}\bs{P}_2^{-1}\bs{P}_3^{-1}\bs{P}_4^{-1}\bs{P}_5^{-1}\\
                &=\begin{pmatrix}
                    1& 0 &0 &0\\
                    2&1&0&0\\
                    0&0&1&0\\
                    0&0&0&1
                \end{pmatrix}
                \begin{pmatrix}
                    1& 0 &0 &0\\
                    0&1&0&0\\
                    \frac{3}{2}&0&1&0\\
                    0&0&0&1
                \end{pmatrix}
                \begin{pmatrix}
                    1& 0 &0 &0\\
                    0&1&0&0\\
                    0&0&1&0\\
                    \frac{1}{2}&0&0&1
                \end{pmatrix}
                \begin{pmatrix}
                    1& 0 &0 &0\\
                    0&1&0&0\\
                    0&1&1&0\\
                    0&0&0&1
                \end{pmatrix}
                \begin{pmatrix}
                    1& 0 &0 &0\\
                    0&1&0&0\\
                    0&0&1&0\\
                    0&-\frac{3}{2}&0&1
                \end{pmatrix}\\
                &=\begin{pmatrix}
                    1& 0 &0 &0\\
                    2&1&0&0\\
                    \frac{3}{2}&1&1&0\\
                    \frac{1}{2}&-\frac{3}{2}&0&1
                \end{pmatrix}
            \end{align*}
\begin{align*}
    \bs{A}=\bs{L}\bs{R}=\begin{pmatrix}
        1& 0 &0 &0\\
        2&1&0&0\\
        \frac{3}{2}&1&1&0\\
        \frac{1}{2}&-\frac{3}{2}&0&1
    \end{pmatrix}
    \begin{pmatrix}
        2&4&6&8\\
        0&4&8&0\\
        0&0&3&6\\
        0&0&0&10
    \end{pmatrix}
\end{align*}
\end{solution}

\begin{question}(p141.4)
    计算矩阵$\begin{pmatrix}
        1&2&3&4\\
        2&8&10&2\\
        3&10&14&6\\
        4&2&6&29
    \end{pmatrix}$的Cholesky分解。
\end{question}

\begin{solution}
    设$\bs{G}=\begin{pmatrix}
        g_{11}& & &\\
        g_{21}&g_{22}& &\\
        g_{31}&g_{32}&g_{33}& \\
        g_{41}&g_{42}&g_{43}&g_{44}\\
    \end{pmatrix}$,则$\bs{G}^T=\begin{pmatrix}
        g_{11}& g_{21}&g_{31}&g_{41}\\
        &g_{22}&g_{32}&g_{42}&\\
        & &g_{33}&g_{43}\\
        & & & g_{44}
    \end{pmatrix}$,其中空白位置全是$0$。
    于是有:
    \begin{align*}
        \left\{
            \begin{array}{ll}
                g_{11}^2=1\\
                g_{11}g_{21}=2\\
                g_{11}g_{31}=3\\
                g_{11}g_{41}=4 \\
                g_{21}g_{11}=2\\
                g_{21}^2+g_{22}^2=8\\
                g_{21}g_{31}+g_{22}g_{32}=10\\
                g_{21}g_{41}+g_{22}g_{42}=2 \\
                g_{31}g_{11}=3\\
                g_{31}g_{21}+g_{32}g_{22}=10\\
                g_{31}^2+g_{32}^2+g_{33}^2=14\\
                g_{31}g_{41}+g_{32}g_{42}+g_{33}g_{43}=6 \\
                g_{41}g_{11}=4 \\
                g_{41}g_{21}+g_{42}g_{22}=2 \\
                g_{41}g_{31}+g_{42}g_{32}+g_{43}g_{33}= 6\\
                g_{41}^2+g_{42}^2+g_{43}^2+g_{44}^2=29
            \end{array}
            \right.
            \Rightarrow
        \left\{
            \begin{array}{ll}
                g_{11}=1\\
                g_{21}=2\\
                g_{31}=3\\
                g_{41}= 4\\
                g_{22}=2\\
                g_{32}=2\\
                g_{42}=-3 \\
                g_{33}=1 \\
                g_{43}= 0\\
                g_{44}= 2
            \end{array}
            \right.
    \end{align*}
    $\bs{G}=\begin{pmatrix}
        1 &0 &0&0\\
        2& 2&0&0\\
        3& 2&1&0\\
        4&-3&0&2
    \end{pmatrix}\quad \bs{A}=\bs{G}\bs{G}^T=\begin{pmatrix}
        1 &0 &0&0\\
        2& 2&0&0\\
        3& 2&1&0\\
        4&-3&0&2
    \end{pmatrix}\begin{pmatrix}
        1 &2 &3&4\\
        0& 2&2&-3\\
        0& 0&1&0\\
        0&0&0&2
    \end{pmatrix}$

\end{solution}

\begin{question}(p141.5)
    计算下列矩阵的满秩分解。
    \begin{tasks}[label=(\arabic*)](2)
        \task $\begin{pmatrix}
            1&2&3&3\\
            4&5&9&6\\
            7&8&15&9\\
            2&5&7&8
        \end{pmatrix}$
        \task $\begin{pmatrix}
            1&3&-3&4\\
            3&5&-5&8\\
            6&-1&1&5\\
            8&-6&6&2
        \end{pmatrix}$
    \end{tasks}
\end{question}
   
\begin{solution}
    \begin{enumerate}[label=(\arabic*)]
        \item 先对矩阵进行初等行变换变成行阶梯型矩阵
        \begin{align*}
            \begin{pmatrix}
                1&2&3&3\\
                4&5&9&6\\
                7&8&15&9\\
                2&5&7&8
            \end{pmatrix} \xrightarrow{}
            \begin{pmatrix}
                1&2&3&3\\
                0&1&1&2\\
                0&0&0&0\\
                0&0&0&0
            \end{pmatrix}
        \end{align*}
        于是$\bs{A}=\bs{B}\bs{C}=\begin{pmatrix}
            1&2\\
            4&5\\
            7&8\\
            2&5
        \end{pmatrix}\begin{pmatrix}
            1&0&1&-1\\
            0&1&1&2\\
        \end{pmatrix}$。
    \item 先对矩阵进行初等行变换变成行阶梯型矩阵
    \begin{align*}
        \begin{pmatrix}
            1&3&-3&4\\
            3&5&-5&8\\
            6&-1&1&5\\
            8&-6&6&2
        \end{pmatrix} \xrightarrow{}
        \begin{pmatrix}
            1&3&-3&4\\
            0&-1&1&-1\\
            0&0&0&0\\
            0&0&0&0
        \end{pmatrix}
    \end{align*}
    于是$\bs{A}=\bs{B}\bs{C}=\begin{pmatrix}
        1&3\\
        3&5\\
        6&-1\\
        8&-6
    \end{pmatrix}\begin{pmatrix}
        1&0&0&1\\
        0&1&-1&1\\
    \end{pmatrix}$
    \end{enumerate}
\end{solution}

\begin{question}(p141.6)
    计算下列矩阵的谱分解。
    \begin{tasks}[label=(\arabic*)](2)
        \task $\begin{pmatrix}
            3&0&1\\
            0&2&0\\
            1&0&3
        \end{pmatrix}$
        \task $\begin{pmatrix}
            5&-2&0\\
            -2&6&-2\\
            0&-2&7
        \end{pmatrix}$
    \end{tasks}
\end{question}

\begin{solution}
    \begin{enumerate}[label=(\arabic*)]
        \item $\lambda_1=\lambda_2=2,\lambda_3=4$
        所以
        \begin{align*}
            \bs{A}&=\bs{P}\bs{\Lambda}\bs{P}^T\\
            &=\begin{pmatrix}
                0 & -\frac{1}{\sqrt{2}} &\frac{1}{\sqrt{2}} \\
                1 & 0& 0\\
                0 & \frac{1}{\sqrt{2}} &\frac{1}{\sqrt{2}}
            \end{pmatrix}
            \begin{pmatrix}
                2 & & \\
                & 2& \\
                & & 4
            \end{pmatrix}
            \begin{pmatrix}
                0 &1&0 \\
                -\frac{1}{\sqrt{2}} & 0& \frac{1}{\sqrt{2}}\\
                \frac{1}{\sqrt{2}} & 0& \frac{1}{\sqrt{2}}
            \end{pmatrix}\\
            &=2\begin{pmatrix}
                0 \\
                1\\
                0
            \end{pmatrix}\begin{pmatrix}
                0 & 1&0
            \end{pmatrix}+2\begin{pmatrix}
                -\frac{1}{\sqrt{2}}\\
                0\\
                \frac{1}{\sqrt{2}} 
            \end{pmatrix}\begin{pmatrix}
                -\frac{1}{\sqrt{2}} & 0& \frac{1}{\sqrt{2}} 
            \end{pmatrix}+4\begin{pmatrix}
                \frac{1}{\sqrt{2}} \\
                0\\
                \frac{1}{\sqrt{2}} 
            \end{pmatrix}\begin{pmatrix}
                \frac{1}{\sqrt{2}} & 0&\frac{1}{\sqrt{2}} 
            \end{pmatrix}
        \end{align*}
        其中空白位置全是$0$。
        \item $\lambda_1=3,\lambda_2=6,\lambda_3=9$
        所以
        \begin{align*}
            \bs{A}&=\bs{P}\bs{\Lambda}\bs{P}^T\\
            &=\begin{pmatrix}
                \frac{2}{3} & -\frac{2}{3} &\frac{1}{3} \\
                \frac{2}{3} & \frac{1}{3}& -\frac{2}{3}\\
                \frac{1}{3} & \frac{2}{3} &\frac{2}{3}
            \end{pmatrix}
            \begin{pmatrix}
                3 & & \\
                & 6& \\
                & & 9
            \end{pmatrix}
            \begin{pmatrix}
                \frac{2}{3} &\frac{2}{3}&\frac{1}{3} \\
                -\frac{2}{3} & \frac{1}{3}&\frac{2}{3}\\
                \frac{1}{3} & -\frac{2}{3}& \frac{2}{3}
            \end{pmatrix}\\
            &=3\begin{pmatrix}
                \frac{2}{3}  \\
                \frac{2}{3} \\
                \frac{1}{3} 
            \end{pmatrix}\begin{pmatrix}
                \frac{2}{3} & \frac{2}{3} &\frac{1}{3} 
            \end{pmatrix}+6\begin{pmatrix}
                -\frac{2}{3}\\
                \frac{1}{3}\\
                \frac{2}{3}
            \end{pmatrix}\begin{pmatrix}
                -\frac{2}{3} & \frac{1}{3}& \frac{2}{3}
            \end{pmatrix}+9\begin{pmatrix}
                \frac{1}{3} \\
                -\frac{2}{3}\\
                \frac{2}{3}
            \end{pmatrix}\begin{pmatrix}
                \frac{1}{3} & -\frac{2}{3}&\frac{2}{3}
            \end{pmatrix}
        \end{align*}
        其中空白位置全是$0$。
    \end{enumerate}
\end{solution}

\begin{question}(p141.7)
    计算下列矩阵的QR分解。
    \begin{tasks}[label=(\arabic*)](4)
        \task $\begin{pmatrix}
            2&1\\
            1&1\\
            2&1
        \end{pmatrix}$
        \task $\begin{pmatrix}
            1&0\\
            0&1\\
            1&1
        \end{pmatrix}$
        \task $\begin{pmatrix}
            0&1&1\\
            1&1&0\\
            1&0&0
        \end{pmatrix}$
        \task $\begin{pmatrix}
            2&2&1\\
            0&2&2\\
            2&1&2
        \end{pmatrix}$
    \end{tasks}
\end{question}

\begin{solution}
    把矩阵记作$\bs{A}_{m\times n}$,并进行列分块$(\bs{a}_1,\ldots,\bs{a}_n)$。
    \begin{enumerate}[label=(\arabic*)]
        \item 对$\bs{A}$进行Schmidt正交化,得:
        \begin{align*}
            \bs{\beta}_1&=\bs{\alpha}_1=(2,1,2)^T \\
            \bs{\beta}_2&=\bs{\alpha}_2-\frac{(\bs{\alpha}_2,\bs{\beta}_1)}{(\bs{\beta}_1,\bs{\beta}_1)}\bs{\beta}_1
            =\bs{\alpha}_2-\frac{5}{9}\bs{\beta}_1=
            (-\frac{1}{9},\frac{4}{9},-\frac{1}{9})^T
        \end{align*}
        于是
        \begin{align*}
            \bs{A}&
            =(\bs{\alpha}_1,\bs{\alpha}_2)=(\bs{\beta}_1,\bs{\beta}_2)\begin{pmatrix}
                1 &\frac{5}{9}\\
                0 & 1
            \end{pmatrix}\\
            &=\begin{pmatrix}
                2& -\frac{1}{9}\\
                1& \frac{4}{9}\\
                2&-\frac{1}{9}
            \end{pmatrix}
                \begin{pmatrix}
                1 &\frac{5}{9}\\
                0 & 1
            \end{pmatrix}\\
            &=\begin{pmatrix}
                \frac{2}{3}& -\frac{1}{3\sqrt{2}}\\
                \frac{1}{3}& \frac{4}{3\sqrt{2}}\\
                \frac{2}{3}&-\frac{1}{3\sqrt{2}}
            \end{pmatrix}
            \begin{pmatrix}
                3 & \\
                & \frac{\sqrt{2}}{3}
            \end{pmatrix}
                \begin{pmatrix}
                1 &\frac{5}{9}\\
                0 & 1
            \end{pmatrix}\\
            &=\begin{pmatrix}
                \frac{2}{3}& -\frac{1}{3\sqrt{2}}\\
                \frac{1}{3}& \frac{4}{3\sqrt{2}}\\
                \frac{2}{3}&-\frac{1}{3\sqrt{2}}
            \end{pmatrix}
            \begin{pmatrix}
                3 &\frac{5}{3}\\
                0 & \frac{\sqrt{2}}{3}
            \end{pmatrix}
        \end{align*}
        \item 对$\bs{A}$进行Schmidt正交化,得:
        \begin{align*}
            \bs{\beta}_1&=\bs{\alpha}_1=(1,0,1)^T \\
            \bs{\beta}_2&=\bs{\alpha}_2-\frac{(\bs{\alpha}_2,\bs{\beta}_1)}{(\bs{\beta}_1,\bs{\beta}_1)} \bs{\beta}_1
            =\bs{\alpha}_2-\frac{1}{2}\bs{\beta}_1=
            (-\frac{1}{2},1,-\frac{1}{2})^T
        \end{align*}
        于是
        \begin{align*}
            \bs{A}&=(\bs{\alpha}_1,\bs{\alpha}_2)
            =(\bs{\beta}_1,\bs{\beta}_2)\begin{pmatrix}
                1 &\frac{1}{2}\\
                0 & 1
            \end{pmatrix}\\
            &=\begin{pmatrix}
                1& -\frac{1}{2}\\
                0& 1\\
                1&\frac{1}{2}
            \end{pmatrix}
            \begin{pmatrix}
                1 &\frac{1}{2}\\
                0 & 1
            \end{pmatrix}\\
            &=\begin{pmatrix}
                \frac{1}{\sqrt{2}}& -\frac{1}{\sqrt{6}}\\
                0& \frac{\sqrt{6}}{3}\\
                \frac{1}{\sqrt{2}}&\frac{1}{\sqrt{6}}
            \end{pmatrix}
            \begin{pmatrix}
                \sqrt{2} & \\
                & \frac{\sqrt{3}}{\sqrt{2}}
            \end{pmatrix}
                \begin{pmatrix}
                    1 &\frac{1}{2}\\
                    0 & 1
            \end{pmatrix}\\
            &=\begin{pmatrix}
                \frac{1}{\sqrt{2}}& -\frac{1}{\sqrt{6}}\\
                0& \frac{\sqrt{6}}{3}\\
                \frac{1}{\sqrt{2}}&\frac{1}{\sqrt{6}}
            \end{pmatrix}
            \begin{pmatrix}
                \sqrt{2} &\frac{\sqrt{2}}{2}\\
                0 & \frac{\sqrt{6}}{2}
            \end{pmatrix}
        \end{align*}
        \item 对$\bs{A}$进行Schmidt正交化,得:
        \begin{align*}
            \bs{\beta}_1&=\bs{\alpha}_1=(0,1,1)^T\\
            \bs{\beta}_2&=\bs{\alpha}_2-\frac{(\bs{\alpha}_2,\bs{\beta}_1)}{(\bs{\beta}_1,\bs{\beta}_1)}\bs{\beta}_1
            =\bs{\alpha}_2-\frac{1}{2}\bs{\beta}_1=
            (1,\frac{1}{2},-\frac{1}{2})^T\\
            \bs{\beta}_3&=\bs{\alpha}_3
            -\frac{(\bs{\alpha}_3,\bs{\beta}_2)}{(\bs{\beta}_2,\bs{\beta}_2)}\bs{\beta}_2
            -\frac{(\bs{\alpha}_3,\bs{\beta}_1)}{(\bs{\beta}_1,\bs{\beta}_1)}\bs{\beta}_1
            =\bs{\alpha}_3-\frac{2}{3}\bs{\beta}_2=
            (\frac{1}{3},-\frac{1}{3},\frac{1}{3})^T
        \end{align*}
        于是
        \begin{align*}
            \bs{A}
            &=(\bs{\alpha}_1,\bs{\alpha}_2,\bs{\alpha}_3)
            =(\bs{\beta}_1,\bs{\beta}_2,\bs{\beta}_3)
            \begin{pmatrix}
                1 &\frac{1}{2}&0\\
                0 & 1&\frac{2}{3} \\
                0&0&1
            \end{pmatrix}\\
            &=\begin{pmatrix}
                0& 1 & \frac{1}{3}\\
                1& \frac{1}{2}& -\frac{1}{3} \\
                1 & -\frac{1}{2} & \frac{1}{3}
            \end{pmatrix} 
            \begin{pmatrix}
                1 &\frac{1}{2}&0\\
                0 & 1&\frac{2}{3} \\
                0&0&1
            \end{pmatrix}\\
            &=\begin{pmatrix}
                0& \frac{\sqrt{2}}{\sqrt{3}} & \frac{\sqrt{3}}{3}\\
                \frac{1}{\sqrt{2}}& \frac{\sqrt{2}}{2\sqrt{3}}& -\frac{\sqrt{3}}{3} \\
                \frac{1}{\sqrt{2}} & -\frac{\sqrt{2}}{2\sqrt{3}}& \frac{\sqrt{3}}{3}
            \end{pmatrix}
            \begin{pmatrix}
                \sqrt{2} & & \\
                & \frac{\sqrt{6}}{2}& \\
                & & \frac{1}{\sqrt{3}}
            \end{pmatrix}
            \begin{pmatrix}
                1 &\frac{1}{2}&0\\
                0 & 1&\frac{2}{3} \\
                0&0&1
            \end{pmatrix}\\
            &=\begin{pmatrix}
                0& \frac{\sqrt{2}}{\sqrt{3}} & \frac{\sqrt{3}}{3}\\
                \frac{1}{\sqrt{2}}& \frac{\sqrt{2}}{2\sqrt{3}}& -\frac{\sqrt{3}}{3} \\
                \frac{1}{\sqrt{2}} & -\frac{\sqrt{2}}{2\sqrt{3}}& \frac{\sqrt{3}}{3}
            \end{pmatrix}
            \begin{pmatrix}
                \sqrt{2} &\frac{\sqrt{2}}{2}&0\\
                0 & \frac{\sqrt{6}}{2}&\frac{\sqrt{6}}{3} \\
                0&0&\frac{\sqrt{3}}{3}
            \end{pmatrix}
        \end{align*}
        \item 对$\bs{A}$进行Schmidt正交化,得:
        \begin{align*}
            \bs{\beta}_1&=\bs{\alpha}_1=(2,0,2)^T\\
            \bs{\beta}_2&=\bs{\alpha}_2-\frac{(\bs{\alpha}_2,\bs{\beta}_1)}{(\bs{\beta}_1,\bs{\beta}_1)}\bs{\beta}_1
            =\bs{\alpha}_2-\frac{3}{4}\bs{\beta}_1=
            (\frac{1}{2},2,-\frac{1}{2})^T\\
            \bs{\beta}_3&=\bs{\alpha}_3
            -\frac{(\bs{\alpha}_3,\bs{\beta}_2)}{(\bs{\beta}_2,\bs{\beta}_2)}\bs{\beta}_2
            -\frac{(\bs{\alpha}_3,\bs{\beta}_1)}{(\bs{\beta}_1,\bs{\beta}_1)}\bs{\beta}_1
            =\bs{\alpha}_3-\frac{7}{9}\bs{\beta}_2-\frac{3}{4}\bs{\beta}_1=
            (-\frac{8}{9},\frac{4}{9},\frac{8}{9})^T
        \end{align*}
        于是
        \begin{align*}
            \bs{A}
            &=(\bs{\alpha}_1,\bs{\alpha}_2,\bs{\alpha}_3)
            =(\bs{\beta}_1,\bs{\beta}_2,\bs{\beta}_3)
            \begin{pmatrix}
                1 &\frac{3}{4}&\frac{3}{4}\\
                0 & 1&\frac{7}{9} \\
                0&0&1
            \end{pmatrix}\\
            &=\begin{pmatrix}
                2& \frac{1}{2} & -\frac{8}{9}\\
                0& 2& \frac{4}{9} \\
                2 & -\frac{1}{2} & \frac{8}{9}
            \end{pmatrix} 
            \begin{pmatrix}
                1 &\frac{3}{4}&\frac{3}{4}\\
                0 & 1&\frac{7}{9} \\
                0&0&1
            \end{pmatrix}\\
            &=\begin{pmatrix}
                \frac{1}{\sqrt{2}}& \frac{1}{3\sqrt{2}} & -\frac{2}{3}\\
                0& \frac{4}{3\sqrt{2}}& \frac{1}{3} \\
                \frac{1}{\sqrt{2}} & -\frac{1}{3\sqrt{2}}& \frac{2}{3}
            \end{pmatrix}
            \begin{pmatrix}
                2\sqrt{2} & & \\
                & \frac{3}{\sqrt{2}}& \\
                & & \frac{4}{3}
            \end{pmatrix}
            \begin{pmatrix}
                1 &\frac{3}{4}&\frac{3}{4}\\
                0 & 1&\frac{7}{9} \\
                0&0&1
            \end{pmatrix}\\
            &=\begin{pmatrix}
                \frac{1}{\sqrt{2}}& \frac{1}{3\sqrt{2}} & -\frac{2}{3}\\
                0& \frac{4}{3\sqrt{2}}& \frac{1}{3} \\
                \frac{1}{\sqrt{2}} & -\frac{1}{3\sqrt{2}}& \frac{2}{3}
            \end{pmatrix}
            \begin{pmatrix}
                2\sqrt{2} &\frac{3\sqrt{2}}{2}&\frac{3\sqrt{2}}{2}\\
                0 & \frac{3}{\sqrt{2}}&\frac{7}{3\sqrt{2}} \\
                0&0&\frac{4}{3}
            \end{pmatrix}
        \end{align*}
    \end{enumerate}
\end{solution}


\begin{question}(p141.8)
    计算下列矩阵的奇异值分解。
    \begin{tasks}[label=(\arabic*)](2)
        \task $\begin{pmatrix}
            2&0\\
            0&2\\
            1&1
        \end{pmatrix}$
        \task $\begin{pmatrix}
            1&0&1\\
            0&1&-1\\
        \end{pmatrix}$
    \end{tasks}
\end{question}


\begin{solution}
    \begin{enumerate}[label=(\arabic*)]
        \item 不妨令$\bs{A}=\bs{U}\bs{\Sigma}\bs{V}^T$,其中$\bs{\Sigma}=\begin{pmatrix}
            \bs{D}& \bs{O}\\
            \bs{O}&\bs{O}
        \end{pmatrix}$。
        先计算$\bs{V}^T$,即利用相似对角化计算$\bs{V}^T\bs{A}^T\bs{A}\bs{V}=\begin{pmatrix}
            \bs{D}^2&\bs{O}\\
            \bs{O}&\bs{O}
        \end{pmatrix}$,
        其中$\bs{A}^T\bs{A}=\begin{pmatrix}
            5& 1\\
            1 & 5
        \end{pmatrix}$,$\bs{V}^T=\begin{pmatrix}
            \frac{1}{\sqrt{2}}&-\frac{1}{\sqrt{2}}\\
            \frac{1}{\sqrt{2}}&\frac{1}{\sqrt{2}}
        \end{pmatrix}$,$\bs{\Sigma}=\begin{pmatrix}
            2 & 0\\
            0 & \sqrt{6} \\
            0 & 0
        \end{pmatrix}$,
        此时$\bs{V}_1=\begin{pmatrix}
            \frac{1}{\sqrt{2}}&\frac{1}{\sqrt{2}}\\
            -\frac{1}{\sqrt{2}}&\frac{1}{\sqrt{2}}
        \end{pmatrix}
        $,$\bs{D}=\begin{pmatrix}
            2 & 0\\
            0& \sqrt{6}
        \end{pmatrix}$。

        \begin{align*}
            \bs{U}_1=\bs{A}\bs{V}_1\bs{D}^{-1}=
            \begin{pmatrix}
                2 & 0\\
                0 & \sqrt{6} \\
                0 & 0
            \end{pmatrix}\begin{pmatrix}
                \frac{1}{\sqrt{2}}&-\frac{1}{\sqrt{2}}\\
            \frac{1}{\sqrt{2}}&\frac{1}{\sqrt{2}}
            \end{pmatrix} \begin{pmatrix}
                \frac{1}{2}& 0\\
                0& \frac{1}{\sqrt{6}}
            \end{pmatrix}=\begin{pmatrix}
                \frac{1}{\sqrt{2}}&\frac{1}{\sqrt{3}}\\
                -\frac{1}{\sqrt{2}}&\frac{1}{\sqrt{3}}\\
                0&\frac{1}{\sqrt{3}}&
            \end{pmatrix}
        \end{align*}
        把$\bs{U}_1$扩充成正交矩阵$\bs{U}=\begin{pmatrix}
            \frac{1}{\sqrt{2}}&\frac{1}{\sqrt{3}}&\frac{1}{\sqrt{6}}\\
            -\frac{1}{\sqrt{2}}&\frac{1}{\sqrt{3}}&\frac{1}{\sqrt{6}}\\
            0&\frac{1}{\sqrt{3}}&-\frac{2}{\sqrt{6}}
        \end{pmatrix}$,于是:
        \begin{align*}
            \bs{A}=\bs{U}\bs{\Sigma}\bs{V}^T=\begin{pmatrix}
                \frac{1}{\sqrt{2}}&\frac{1}{\sqrt{3}}&\frac{1}{\sqrt{6}}\\
                -\frac{1}{\sqrt{2}}&\frac{1}{\sqrt{3}}&\frac{1}{\sqrt{6}}\\
                0&\frac{1}{\sqrt{3}}&-\frac{2}{\sqrt{6}}
            \end{pmatrix}
            \begin{pmatrix}
                2 & 0\\
                0 & \sqrt{6} \\
                0 & 0
            \end{pmatrix}
            \begin{pmatrix}
                \frac{1}{\sqrt{2}}&-\frac{1}{\sqrt{2}}\\
                \frac{1}{\sqrt{2}}&\frac{1}{\sqrt{2}}
            \end{pmatrix}
        \end{align*}

        \item 不妨令$\bs{A}=\bs{U}\bs{\Sigma}\bs{V}^T$,其中$\bs{\Sigma}=\begin{pmatrix}
            \bs{D}& \bs{O}\\
            \bs{O}&\bs{O}
        \end{pmatrix}$。
        先计算$\bs{V}^T$,即利用相似对角化计算$\bs{V}^T\bs{A}^T\bs{A}\bs{V}=\begin{pmatrix}
            \bs{D}^2&\bs{O}\\
            \bs{O}&\bs{O}
        \end{pmatrix}$,
        其中$\bs{A}^T\bs{A}=\begin{pmatrix}
            1& 0&1\\
            0 & 1&-1 \\
            1&-1&2
        \end{pmatrix}$,$\bs{V}^T=\begin{pmatrix}
            \frac{1}{\sqrt{2}}&\frac{1}{\sqrt{2}}&0\\
            \frac{1}{\sqrt{6}}&-\frac{1}{\sqrt{6}}&\frac{2}{\sqrt{6}}\\
            -\frac{1}{\sqrt{3}}&\frac{1}{\sqrt{3}}&\frac{1}{\sqrt{3}}
        \end{pmatrix}$,$\bs{\Sigma}=\begin{pmatrix}
            1 & 0&0\\
            0 & \sqrt{3}&0 \\
        \end{pmatrix}$,
        此时$\bs{V}_1=\begin{pmatrix}
            \frac{1}{\sqrt{2}}&\frac{1}{\sqrt{6}}\\
            \frac{1}{\sqrt{2}}&-\frac{1}{\sqrt{6}}\\
            0&\frac{2}{\sqrt{6}}\\
        \end{pmatrix}
        $,$\bs{D}=\begin{pmatrix}
            1 & 0 \\
            0& \sqrt{3}\\
        \end{pmatrix}$。

        \begin{align*}
            \bs{U}_1=\bs{A}\bs{V}_1\bs{D}^{-1}=
            \begin{pmatrix}
                1& 0&1\\
                0 & 1&-1 \\
            \end{pmatrix}\begin{pmatrix}
                \frac{1}{\sqrt{2}}&\frac{1}{\sqrt{6}}\\
                \frac{1}{\sqrt{2}}&-\frac{1}{\sqrt{6}}\\
                0&\frac{2}{\sqrt{6}}\\
            \end{pmatrix} \begin{pmatrix}
                1& 0\\
                0& \frac{1}{\sqrt{3}}
            \end{pmatrix}=\begin{pmatrix}
                \frac{1}{\sqrt{2}}&\frac{1}{\sqrt{2}}\\
                \frac{1}{\sqrt{2}}&-\frac{1}{\sqrt{2}}\\
            \end{pmatrix}
        \end{align*}
        由于$\bs{U}_1$已经是正交矩阵,无需扩充,令$\bs{U}=\bs{U}_1$,于是:
        \begin{align*}
            \bs{A}=\bs{U}\bs{\Sigma}\bs{V}^T=\begin{pmatrix}
                \frac{1}{\sqrt{2}}&\frac{1}{\sqrt{2}}\\
                \frac{1}{\sqrt{2}}&-\frac{1}{\sqrt{2}}\\
            \end{pmatrix}
            \begin{pmatrix}
                1 & 0&0\\
                0 & \sqrt{3}&0 \\
            \end{pmatrix}
            \begin{pmatrix}
                \frac{1}{\sqrt{2}}&\frac{1}{\sqrt{2}}&0\\
                \frac{1}{\sqrt{6}}&-\frac{1}{\sqrt{6}}&\frac{2}{\sqrt{6}}\\
                -\frac{1}{\sqrt{3}}&\frac{1}{\sqrt{3}}&\frac{1}{\sqrt{3}}
            \end{pmatrix}
        \end{align*}
    \end{enumerate}
\end{solution}


\begin{question}(p142.9)
    证明:对任意实(复)非退化方阵$\bs{A}$,存在唯一的正交(酉)矩阵$\bs{Q}$和
    正定矩阵$\bs{H}_1$和$\bs{H}_2$,使得$\bs{A}=\bs{Q}\bs{H}_1=\bs{H}_2\bs{Q}$,
    该分解称为矩阵的\textbf{极分解},若去掉矩阵的非退化条件,结论改如何修正?
\end{question}


\begin{proof}
    先给出需要用到的引理:
    \begin{lemma} \label{正定引理}
        任意一个正定矩阵$\bs{A}$,一定存在唯一的一个正定矩阵$\bs{S}$使得
        $\bs{A}=\bs{S}^2$
    \end{lemma}
    \begin{proof}
        \textbf{存在性}:由于$\bs{H} \succ \bs{0}$,则存在正交矩阵$\bs{P}$,
        使得$\bs{H}=\bs{P}\bs{\Lambda}\bs{P}^T$,其中$\bs{\Lambda}
        =\mathrm{diag}\{\lambda_1,\ldots,\lambda_n\}$,$\lambda_i > 0$。
        令$\bs{Z}=\bs{\Lambda}^{\frac{1}{2}}=\mathrm{diag}\{\sqrt{\lambda_1},\ldots,\sqrt{\lambda_n}\}$,
        则
        \begin{align*}
            \bs{H}=\bs{P}\bs{\Lambda}^{\frac{1}{2}}\bs{\Lambda}^{\frac{1}{2}}\bs{P}^T=(\bs{P}\bs{Z}\bs{P}^T)(\bs{P}\bs{Z}\bs{P}^T)
            =(\bs{P}\bs{Z}\bs{P}^T)^2=\bs{S}^2
        \end{align*}
        而任取$\bs{x}\in \R^n$,$\bs{x}^T\bs{S}\bs{x}=
        \bs{x}^T\bs{P}\bs{Z}\bs{P}^T\bs{x}=
        (\bs{P}^T\bs{x})^T\bs{Z}(\bs{P}^T\bs{x}) >0$,
        这说明了$\bs{S}$是正定阵。

        \textbf{唯一性}:记矩阵$\bs{A}$的特征值与对应的特征向量为$\lambda,\bs{\nu}$。
        若存在两个正定阵$\bs{S}_0,\bs{S}_1$,
        使得$\bs{H}=\bs{S}_0^2=\bs{S}_1^2$,显然有$\bs{S}_0\bs{\nu}=\bs{S}_1\bs{\nu}=\sqrt{\lambda}\bs{\nu}$。
        于是$\bs{S}_0\bs{S}_1\bs{\nu}=\sqrt{\lambda}\bs{S}_0\bs{\nu}=\lambda \bs{\nu},
        \bs{S}_0\bs{S}_1\bs{\nu}=\sqrt{\lambda}\bs{S}_0\bs{\nu}=\lambda \bs{\nu},
        (\bs{S}_0\bs{S}_1-\bs{S}_1\bs{S}_0)\bs{\nu}=\bs{0}$,
        由于$\bs{H}$是对称的,一定存在$n$个线性无关的特征向量,即特征子空间的维数$\mathrm{dim}(\bs{V})=n$,
        则$\mathrm{r}(\bs{S}_0\bs{S}_1-\bs{S}_1\bs{S}_0)=n-\mathrm{dim}(\bs{V})=0$,即$\bs{S}_0\bs{S}_1-\bs{S}_1\bs{S}_0=\bs{O}$。
        此时$(\bs{S}_0+\bs{S}_1)(\bs{S}_0-\bs{S}_1)=\bs{S}_0^2-\bs{S}_0\bs{S}_1+\bs{S}_1\bs{S}_0-\bs{S}_1^2=\bs{O}$,于是$\mathrm{r}(\bs{S}_0+\bs{S}_1)+\mathrm{r}(\bs{S}_0-\bs{S}_1)=n$,
        而$\bs{S}_0+\bs{S}_1$是正定矩阵,即$\mathrm{r}(\bs{S}_0-\bs{S}_1)=n$,所以$\mathrm{r}(\bs{S}_0-\bs{S}_1)=0$,$\bs{S}_0=\bs{S}_1$。
    
    \end{proof}
    下面给出两种证明方法。
    \begin{enumerate}[label=(\arabic*)]
        \item 
    \textbf{存在性}:由于$\bs{A}^T\bs{A}$是一个正定矩阵,
    由引理\ref{正定引理}可知,$\bs{A}^T\bs{A}=\bs{H}_1^2$,
    则有$\bs{E}=\bs{H}_1^{-1}\bs{A}^T\bs{A}\bs{H}_1^{-1}=
    (\bs{H}_1^T)^{-1}\bs{A}^T\bs{A}\bs{H}_1^{-1}
    =(\bs{A}\bs{H}_1^{-1})^T\bs{A}\bs{H}_1^{-1}$,
    令$\bs{Q}_1=\bs{A}\bs{H}_1^{-1}$,有$\bs{Q}_1^T\bs{Q}_1=\bs{E}$,
    故$\bs{Q}_1$是正交矩阵,同时$\bs{A}=\bs{Q}_1\bs{H}_1$。
    
    同理:$\bs{A}\bs{A}^T=\bs{H}_2^2$,
    则有$\bs{E}=\bs{H}_2^{-1}\bs{A}\bs{A}^T\bs{H}_2^{-1}=\bs{H}_2^{-1}\bs{A}\bs{A}^T(\bs{H}_2^T)^{-1}
    =\bs{H}_2^{-1}\bs{A}(\bs{H}_2^{-1}\bs{A})^T$,令$\bs{Q}_2=\bs{H}_2^{-1}\bs{A}$,有$\bs{Q}_2\bs{Q}_2^T=\bs{E}$,
    故$\bs{Q}_2$是正交矩阵,同时$\bs{A}=\bs{H}_2\bs{Q}_2$。

    下面证明$\bs{Q}_1=\bs{Q}_2=\bs{Q}$,即证$\bs{A}\bs{H}_1^{-1}=\bs{H}_2^{-1}\bs{A} \Leftrightarrow \bs{H}_2\bs{A}=\bs{A}\bs{H}_1$,


    \textbf{唯一性}:假设存在另外一个正交矩阵$\bs{U}$与正定矩阵$\bs{W}$,使得
    $\bs{A}=\bs{U}\bs{W}$
    由引理\ref{正定引理}唯一性可知,$\bs{W}=\bs{H}_1$;而
    $\bs{U}=\bs{A}\bs{W}^{-1}=\bs{A}\bs{H}_1^{-1}=\bs{Q}$,同理可以说明$\bs{A}=\bs{H}_2\bs{Q}$分解
    的唯一性。
        
    \item
    \textbf{存在性}:由SVD分解可知,$\bs{A}=\bs{U}\bs{\Sigma}\bs{V}^T$,其中$\bs{U},\bs{V}^T$是正交矩阵,
    $\bs{\Sigma}=\mathrm{diag}\{\lambda_1,\ldots,\lambda_n\}$,$\lambda_i >0$。又因为
    $\bs{A}=\bs{U}\bs{\Sigma}\bs{V}^T=\bs{U}(\bs{V}^T\bs{V})\bs{\Sigma}\bs{V}^T=(\bs{U}\bs{V}^T)\bs{V}\bs{\Sigma}\bs{V}^T$,
    令$\bs{Q}=\bs{U}\bs{V}^T,\bs{H}_1=\bs{V}\bs{\Sigma}\bs{V}^T$,容易验证$\bs{Q}$为正交矩阵,$\bs{H}_1$为正定矩阵。

    同理:$\bs{A}=\bs{U}\bs{\Sigma}\bs{V}^T=\bs{U}\bs{\Sigma}(\bs{U}^T\bs{U})\bs{V}^T=\bs{U}\bs{\Sigma}\bs{U}^T(\bs{U}\bs{V}^T)$,
    令$\bs{Q}=\bs{U}\bs{V}^T,\bs{H}_2=\bs{U}\bs{\Sigma}\bs{U}^T$。
    
    \textbf{唯一性}:假设存在另外一个正交矩阵$\bs{U}$与正定矩阵$\bs{W}$,使得
    $\bs{A}=\bs{U}\bs{W}$,$\bs{A}^T\bs{A}=(\bs{Q}\bs{H}_1)^T\bs{Q}\bs{H}_1=\bs{H}_1^2,
    \bs{A}^T\bs{A}=(\bs{U}\bs{W})^T\bs{U}\bs{W}=\bs{W}^2$,
    由引理\ref{正定引理}唯一性可知,$\bs{W}=\bs{H}_1$,而$\bs{A}\bs{H}_1^{-1}=\bs{Q},\bs{A}\bs{W}^{-1}=\bs{U}$,
    于是$\bs{U}=\bs{Q}$。同理可以说明$\bs{A}=\bs{H}_2\bs{Q}$分解
    的唯一性。
\end{enumerate}
结论修正为:存在唯一的酉矩阵$\bs{Q}$与半正定矩阵$\bs{H}_1$与$\bs{H}_2$使得$\bs{A}=\bs{Q}\bs{H}_1=\bs{H}_2\bs{Q}$。

\end{proof}


\begin{question}(p142.10)
    证明:对任何正定矩阵$\bs{H}$,存在唯一的正定矩阵$\bs{S}$,使得$\bs{H}=\bs{S}^2$。
    若将正定矩阵改为半正定矩阵,结论如何?
\end{question}

\begin{proof}
    \textbf{存在性}:由于$\bs{H} \succ \bs{0}$,则存在正交矩阵$\bs{P}$,
    使得$\bs{H}=\bs{P}\bs{\Lambda}\bs{P}^T$,其中$\bs{\Lambda}
    =\mathrm{diag}\{\lambda_1,\ldots,\lambda_n\}$,$\lambda_i > 0$。
    令$\bs{Z}=\bs{\Lambda}^{\frac{1}{2}}=\mathrm{diag}\{\sqrt{\lambda_1},\ldots,\sqrt{\lambda_n}\}$,
    则
    \begin{align*}
        \bs{H}=\bs{P}\bs{\Lambda}^{\frac{1}{2}}\bs{\Lambda}^{\frac{1}{2}}\bs{P}^T=(\bs{P}\bs{Z}\bs{P}^T)(\bs{P}\bs{Z}\bs{P}^T)
        =(\bs{P}\bs{Z}\bs{P}^T)^2=\bs{S}^2
    \end{align*}
    而任取$\bs{x}\in \R^n$,$\bs{x}^T\bs{S}\bs{x}=
    \bs{x}^T\bs{P}\bs{Z}\bs{P}^T\bs{x}=
    (\bs{P}^T\bs{x})^T\bs{Z}(\bs{P}^T\bs{x}) >0$,
    这说明了$\bs{S}$是正定阵。

    \textbf{唯一性}:记矩阵$\bs{A}$的特征值与对应的特征向量为$\lambda,\bs{\nu}$。
    若存在两个正定阵$\bs{S}_0,\bs{S}_1$,
    使得$\bs{H}=\bs{S}_0^2=\bs{S}_1^2$,显然有$\bs{S}_0\bs{\nu}=\bs{S}_1\bs{\nu}=\sqrt{\lambda}\bs{\nu}$。
    于是$\bs{S}_0\bs{S}_1\bs{\nu}=\sqrt{\lambda}\bs{S}_0\bs{\nu}=\lambda \bs{\nu},
    \bs{S}_0\bs{S}_1\bs{\nu}=\sqrt{\lambda}\bs{S}_0\bs{\nu}=\lambda \bs{\nu},
    (\bs{S}_0\bs{S}_1-\bs{S}_1\bs{S}_0)\bs{\nu}=\bs{0}$,
    由于$\bs{H}$是对称的,一定存在$n$个线性无关的特征向量,即特征子空间的维数$\mathrm{dim}(\bs{V})=n$,
    则$\mathrm{r}(\bs{S}_0\bs{S}_1-\bs{S}_1\bs{S}_0)=n-\mathrm{dim}(\bs{V})=0$,即$\bs{S}_0\bs{S}_1-\bs{S}_1\bs{S}_0=\bs{O}$。
    此时$(\bs{S}_0+\bs{S}_1)(\bs{S}_0-\bs{S}_1)=\bs{S}_0^2-\bs{S}_0\bs{S}_1+\bs{S}_1\bs{S}_0-\bs{S}_1^2=\bs{O}$,于是$\mathrm{r}(\bs{S}_0+\bs{S}_1)+\mathrm{r}(\bs{S}_0-\bs{S}_1)=n$,
    而$\bs{S}_0+\bs{S}_1$是正定矩阵,即$\mathrm{r}(\bs{S}_0-\bs{S}_1)=n$,所以$\mathrm{r}(\bs{S}_0-\bs{S}_1)=0$,$\bs{S}_0=\bs{S}_1$。
    
    结论修正为:存在唯一的半正定矩阵$\bs{S}$,使得$\bs{H}=\bs{S}^2$。
    
\end{proof}







\ifx\allfiles\undefined
\end{document}
\fi