\ifx\allfiles\undefined
\documentclass[12pt, a4paper, oneside, UTF8]{ctexbook}
\def\configPath{../config}
\def\coverPath{\configPath/cover}
\def\packagePath{\configPath/package}
\def\theormPath{\configPath/theorem}
\def\customPath{\configPath/custom}
\def\prefacePath{\configPath/preface}

% 在这里定义需要的包
\usepackage{amsmath}
\usepackage{amsthm}
\usepackage{amssymb}
\usepackage{graphicx}
\usepackage{mathrsfs}
\usepackage{enumitem}
\usepackage{geometry}
\usepackage[colorlinks, linkcolor=black]{hyperref}
\usepackage{stackengine}
\usepackage{yhmath}
\usepackage{extarrows}
\usepackage{arydshln}
% \usepackage{unicode-math}
\usepackage{tasks}
\usepackage{fancyhdr}
\usepackage[dvipsnames, svgnames]{xcolor}
\usepackage{listings}


\input{\theormPath/theorem1_zh}
\input{\customPath/custom}




\begin{document}
\else
\fi

\chapter{模拟卷二}
\begin{question} 
   设$\bs{A}=\begin{pmatrix}
    1&2&-2&3\\
    2&-1&1&1\\
    4&3&-3&7
   \end{pmatrix}$,
   求$\bs{A}$的广义逆$\bs{A}^+$。
\end{question}

\begin{solution}
    对矩阵$\bs{A}$进行满秩分解,得:
    $\bs{A}=\bs{B}\bs{C}=\begin{pmatrix}
        1 & 2\\
        2&-1\\
        4&3
    \end{pmatrix}\begin{pmatrix}
        1&0&0&1\\
        0&1&-1&1
    \end{pmatrix}$,于是
    \begin{align*}
        \bs{C}\bs{C}^T=\begin{pmatrix}
            2&1\\
            1&2
        \end{pmatrix} \quad \bs{B}^T\bs{B}=\begin{pmatrix}
            21&12\\
            12&14
        \end{pmatrix}
    \end{align*}
    \begin{align*}
        \bs{A}^+&=
    \bs{C}^T(\bs{C}\bs{C}^T)^{-1}(\bs{B}^T\bs{B})^{-1}\bs{B}^T\\
    &=\begin{pmatrix}
        1 & 2\\
        2&-1\\
        4&3
    \end{pmatrix}\begin{pmatrix}
        2&1\\
        1&2
    \end{pmatrix}^{-1}\begin{pmatrix}
        21&12\\
        12&14
    \end{pmatrix}^{-1}
    \begin{pmatrix}
        1&0&0&1\\
        0&1&-1&1
    \end{pmatrix}\\
    &=\frac{1}{18}\begin{pmatrix}
       4&-4&2\\
       -5&8&-1\\
       3&0&3 
    \end{pmatrix}
    \end{align*}
\end{solution}

\begin{question}
    设$\bs{A}=\begin{pmatrix}
        3&1&-2\\
        -7&-2&9\\
        -2&-1&4
    \end{pmatrix}$,
    求可逆阵$\bs{P}$和若当(Jordan)标准型$\bs{J}$,
    使$\bs{P}^{-1}\bs{A}\bs{P}=\bs{J}$,并求$e^{\bs{A}t}$。
\end{question}


\begin{question}
    用矩阵函数求解下常微分方程组初值问题的解
    \begin{align*}
    \left\{
        \begin{array}{ll}
            \frac{\d x_1}{\d t}=-3x_1+4x_2+1\\
            \frac{\d x_2}{\d t}=-x_1+x_2
        \end{array}
        \right.
    \left\{
        \begin{array}{ll}
            x_1(0)=2\\
            x_2(0)=1
        \end{array}
        \right.
    \end{align*}
\end{question}

\begin{question}
    设$\bs{V}$是二阶实方阵全体,,对任意$\bs{A}\in \bs{V}$,
    令$\mc{T}(\bs{A})=2\bs{A}^T-3\bs{A}$,
    证明$\mc{T}$是$\bs{V}$的线性变换。
    \begin{enumerate}[label=(\arabic{*})]
        \item 求$\mc{T}$在$\bs{V}$的基$\bs{B}_1=\begin{pmatrix}
            1&1\\
            0&0
        \end{pmatrix},\bs{B}_2=\begin{pmatrix}
            2&0\\
            0&0
        \end{pmatrix},\bs{B}_3=\begin{pmatrix}
            0&0\\
            1&1
        \end{pmatrix},\bs{B}_4=\begin{pmatrix}
            0&0\\
            1&2
        \end{pmatrix}$下的矩阵表示。
        \item 求$\mc{T}$的特征值。
        \item 判别$\mc{T}$是否可对角化。
    \end{enumerate}
\end{question}

\begin{question}
    设$\mc{T}(\bs{\alpha}_1,\bs{\alpha}_2,\bs{\alpha}_3)=(\bs{\alpha}_1,\bs{\alpha}_2,\bs{\alpha}_3)
    \begin{pmatrix}
        1&2&4\\
        2&1&5\\
        -1&0&-2
    \end{pmatrix}$,求$\mathrm{Im}\mc{T}$和$\mathrm{Ker}\mc{T}$的基和维数。
\end{question}

\begin{question}
    设$\mc{T}$是$n$维线性空间$\bs{V}$的线性变换,$\mathrm{rank}(\mc{T})=r$且$\mc{T}^2=3\mc{T}$,
    证明:存在$\bs{V}$的一组基,使$\mc{T}$在这组基下的矩阵为$\begin{pmatrix}
        \bs{O} &\bs{O}\\
        \bs{O} &3\bs{E}_r
    \end{pmatrix}$,其中$\bs{E}_r$为$r$阶单位阵。
\end{question}

\ifx\allfiles\undefined
\end{document}
\fi