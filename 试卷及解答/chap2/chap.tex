\ifx\allfiles\undefined
\documentclass[12pt, a4paper, oneside, UTF8]{ctexbook}
\def\configPath{../config}
\def\coverPath{\configPath/cover}
\def\packagePath{\configPath/package}
\def\theormPath{\configPath/theorem}
\def\customPath{\configPath/custom}
\def\prefacePath{\configPath/preface}

% 在这里定义需要的包
\usepackage{amsmath}
\usepackage{amsthm}
\usepackage{amssymb}
\usepackage{graphicx}
\usepackage{mathrsfs}
\usepackage{enumitem}
\usepackage{geometry}
\usepackage[colorlinks, linkcolor=black]{hyperref}
\usepackage{stackengine}
\usepackage{yhmath}
\usepackage{extarrows}
\usepackage{arydshln}
% \usepackage{unicode-math}
\usepackage{tasks}
\usepackage{fancyhdr}
\usepackage[dvipsnames, svgnames]{xcolor}
\usepackage{listings}


\input{\theormPath/theorem1_zh}
\input{\customPath/custom}




\begin{document}
\else
\fi

\chapter{模拟卷二}
\begin{question} 
   设$\bs{A}=\begin{pmatrix}
    1&2&-2&3\\
    2&-1&1&1\\
    4&3&-3&7
   \end{pmatrix}$,
   求$\bs{A}$的广义逆$\bs{A}^+$。
\end{question}

\begin{solution}
    对矩阵$\bs{A}$进行满秩分解,得:
    $\bs{A}=\bs{B}\bs{C}=\begin{pmatrix}
        1 & 2\\
        2&-1\\
        4&3
    \end{pmatrix}\begin{pmatrix}
        1&0&0&1\\
        0&1&-1&1
    \end{pmatrix}$,于是
    \begin{align*}
        \bs{C}\bs{C}^T=\begin{pmatrix}
            2&1\\
            1&3
        \end{pmatrix} \quad \bs{B}^T\bs{B}=\begin{pmatrix}
            21&12\\
            12&14
        \end{pmatrix}
    \end{align*}
    \begin{align*}
        \bs{A}^+&=
    \bs{C}^T(\bs{C}\bs{C}^T)^{-1}(\bs{B}^T\bs{B})^{-1}\bs{B}^T\\
    &=\begin{pmatrix}
        1 & 0\\
        0&1\\
        0&-1\\
        1&1
    \end{pmatrix}\begin{pmatrix}
        2&1\\
        1&3
    \end{pmatrix}^{-1}\begin{pmatrix}
        21&12\\
        12&14
    \end{pmatrix}^{-1}
    \begin{pmatrix}
        1&2&4\\
        2&-1&3
    \end{pmatrix}\\
    &=\frac{1}{150}\begin{pmatrix}
       -12&33&9\\
       -14&-26&2\\
       14&26&-2\\
       2&7&11
    \end{pmatrix}
    \end{align*}
\end{solution}

\begin{question}
    设$\bs{A}=\begin{pmatrix}
        3&1&-3\\
        -7&-2&9\\
        -2&-1&4
    \end{pmatrix}$,
    求可逆阵$\bs{P}$和若当(Jordan)标准型$\bs{J}$,
    使$\bs{P}^{-1}\bs{A}\bs{P}=\bs{J}$,并求$e^{\bs{A}t}$。
\end{question}

\begin{solution}
    特征值为$\lambda_1=1,\lambda_2=\lambda_3=2$,对应的若当标准型为$\bs{J}=\begin{pmatrix}
        1 & & \\
        & 2& 1\\
        & & 2
    \end{pmatrix}$,其中空白位置全是$0$。
    由于$\bs{J}=\bs{P}^{-1}\bs{A}\bs{P} \Rightarrow \bs{P}\bs{J}=\bs{A}\bs{P}$,
    不妨令$\bs{P}=(\bs{p}_1,\bs{p}_2,\bs{p}_3)$,其中$\bs{P}$为非奇异矩阵,则:
    \begin{align*}
        \left\{
            \begin{array}{ll}
                \bs{A}\bs{p}_1=\bs{p}_1 \\
                \bs{A}\bs{p}_2=2\bs{p}_2 \\
                \bs{A}\bs{p}_3=\bs{p}_2+2\bs{p}_3
            \end{array} 
            \right.
            \Rightarrow
            \left\{
            \begin{array}{ll}
                \bs{p}_1=(0,3,1)^T \\
                \bs{p}_2=(1,-4,-1)^T \\
                \bs{p}_3=(-1,5,1)^T
            \end{array} 
            \right.
    \end{align*}
    \begin{align*}
        \bs{P}=\begin{pmatrix}
            0 & 1 &-1 \\
            3 & -4 &5 \\
            1 & -1 &1
        \end{pmatrix} \quad
        \bs{P}^{-1}=\begin{pmatrix}
            1	&0	&1 \\
            2	&1	&-3\\
            1	&1	&-3  
        \end{pmatrix}
    \end{align*}
    于是:
    \begin{align*}
        e^{\bs{A}t}&=\bs{P}e^{\bs{J}t}\bs{P}^{-1} \\
        &=\begin{pmatrix}
            0 & 1 &-1 \\
            3 & -4 &5 \\
            1 & -1 &1
        \end{pmatrix} \begin{pmatrix}
            e^{t}& & \\
            & e^{2t}&te^{2t} \\
            & & e^{2t}
        \end{pmatrix}\begin{pmatrix}
            1	&0	&1 \\
            2	&1	&-3\\
            1	&1	&-3      
        \end{pmatrix}\\
        &=\begin{pmatrix}
            (t+1)e^{2t}&te^{2t}&-3te^{2t}\\
            3e^{t}-(4t+3)e^{2t}&(-4t+1)e^{2t}&3e^{t}+(12t-3)e^{2t}\\
            e^{t}-(t+1)e^{2t}&-te^{2t}&e^{t}+3te^{2t}
        \end{pmatrix}
    \end{align*}
\end{solution}

\begin{question}
    用矩阵函数求解下常微分方程组初值问题的解
    \begin{align*}
    \left\{
        \begin{array}{ll}
            \frac{\d x_1}{\d t}=-3x_1+4x_2+1\\
            \frac{\d x_2}{\d t}=-x_1+x_2
        \end{array}
        \right.
    \left\{
        \begin{array}{ll}
            x_1(0)=2\\
            x_2(0)=1
        \end{array}
        \right.
    \end{align*}
\end{question}

\begin{solution}
    由题意得:$\frac{\d \bs{x}}{\d t}=\bs{A}\bs{x}+\bs{b}$,其中$\bs{A}=\begin{pmatrix}
        -3&4\\
        -1&1
    \end{pmatrix},\bs{b}=(1,0)^T$。
    矩阵$\bs{A}$的特征值为$\lambda_1=\lambda_2=-1$,$m_{\bs{A}}(\lambda)=(\lambda+1)^2$
    不妨设$P(\lambda)=a_0+a_1\lambda$,则 \begin{align*}
        \left\{
            \begin{array}{ll}
                P(\lambda)=P(-1)=a_0-a_1=e^{-t}\\
                P'(\lambda)=P'(-1)=a_1=te^{-t}
            \end{array}
            \right.
        \Rightarrow
        \left\{
            \begin{array}{ll}
                a_0=(1+t)e^{-t}\\
                a_1=te^{-t}
            \end{array}
            \right.
    \end{align*}
    \begin{align*}
        e^{\bs{A}t}=P(\bs{A})
    &=(1+t)e^{-t}\bs{E}+te^{-t}\bs{A}\\
    &=\begin{pmatrix}
        (-2t+1)e^{-t} &4te^{-t} \\
        -te^{-t}  & (2t+1)e^{-t}
    \end{pmatrix}
    \end{align*}
    \begin{align*}
        \bs{x}(t)&=e^{\bs{A}t}\bs{x}(0)+e^{\bs{A}t}\int_{0}^t e^{-\bs{A}u}\bs{b} du\\
        &=\begin{pmatrix}
            (-2t+1)e^{-t} &4te^{-t} \\
        -te^{-t}  & (2t+1)e^{-t}
        \end{pmatrix}\begin{pmatrix}
            2 \\
            1
        \end{pmatrix}\\
        &+\begin{pmatrix}
            (-2t+1)e^{-t} &4te^{-t} \\
        -te^{-t}  & (2t+1)e^{-t}
        \end{pmatrix}\int_{0}^t 
        \begin{pmatrix}
            (2u+1)e^{u} &-4ue^{u} \\
            ue^{u}  & (-2u+1)e^{u}
        \end{pmatrix} \begin{pmatrix}
            1\\
            0
        \end{pmatrix} du \\
        &=\begin{pmatrix}
            (2t+3)e^{-t}-1\\
            (t+2)e^{-t}-1
        \end{pmatrix}
        \end{align*}
\end{solution}

\begin{question}
    设$\bs{V}$是二阶实方阵全体,,对任意$\bs{A}\in \bs{V}$,
    令$\mc{T}(\bs{A})=2\bs{A}^T-3\bs{A}$,
    证明$\mc{T}$是$\bs{V}$的线性变换。
    \begin{enumerate}[label=(\arabic{*})]
        \item 求$\mc{T}$在$\bs{V}$的基$\bs{B}_1=\begin{pmatrix}
            1&1\\
            0&0
        \end{pmatrix},\bs{B}_2=\begin{pmatrix}
            2&0\\
            0&0
        \end{pmatrix},\bs{B}_3=\begin{pmatrix}
            0&0\\
            1&1
        \end{pmatrix},\bs{B}_4=\begin{pmatrix}
            0&0\\
            1&2
        \end{pmatrix}$下的矩阵表示。
        \item 求$\mc{T}$的特征值。
        \item 判别$\mc{T}$是否可对角化。
    \end{enumerate}
\end{question}


\begin{solution}
    \begin{enumerate}[label=(\arabic{*})]
        \item 
        \begin{align*}
            \mc{T}\bs{B}_1=\begin{pmatrix}
                -1&-3\\
                2&0
            \end{pmatrix}\ 
            \mc{T}\bs{B}_2=\begin{pmatrix}
               -2&0\\
               0&0 
            \end{pmatrix}\ 
            \mc{T}\bs{B}_3=\begin{pmatrix}
                0&2\\
                -3&-1 
             \end{pmatrix}\ 
             \mc{T}\bs{B}_4=\begin{pmatrix}
                0&2\\
                -3&-2 
             \end{pmatrix}
        \end{align*}
        即
        \begin{align*}
            \mc{T}(\bs{B}_1,\bs{B}_2,\bs{B}_3,\bs{B}_4)=(\bs{B}_1,\bs{B}_2,\bs{B}_3,\bs{B}_4)
            \begin{pmatrix}
                -3& 0&2&2  \\
                1 & -1&-1 &-1 \\
                4& 0& -5& -4\\
                -2& 0& 2&0
            \end{pmatrix}
        \end{align*}
        \item $\lambda_1=-3,\lambda_2=\lambda_3=-1,\lambda_4=-4$。
        \item 可对角化,这是由于$\lambda=-1$的特征值有至少两个线性无关的特征向量。
    \end{enumerate}
\end{solution}


\begin{question}
    设$\mc{T}(\bs{\alpha}_1,\bs{\alpha}_2,\bs{\alpha}_3)=(\bs{\alpha}_1,\bs{\alpha}_2,\bs{\alpha}_3)
    \begin{pmatrix}
        1&2&4\\
        2&1&5\\
        -1&0&-2
    \end{pmatrix}$,求$\mathrm{Im}\mc{T}$和$\mathrm{Ker}\mc{T}$的基和维数。
\end{question}

\begin{solution}
    记$\bs{A}=\begin{pmatrix}
        1&2&4\\
        2&1&5\\
        -1&0&-2
    \end{pmatrix}$。

    $\mathrm{dim}(\mathrm{Im}\mc{T})=\mathrm{r}(\bs{A})=2,\mathrm{dim}(\mathrm{Ker}\mc{T})=3-\mathrm{r}(\bs{A})=1$。
    
    $\mathrm{Im}\mc{T}=<\bs{\alpha}_1+2\bs{\alpha}_2-\bs{\alpha}_3,2\bs{\alpha}_1+\bs{\alpha_2}>,
    \mathrm{Ker}\mc{T}=<2\bs{\alpha}_1+\bs{\alpha}_2-\bs{\alpha}_3>$。

\end{solution}

\begin{question}
    设$\mc{T}$是$n$维线性空间$\bs{V}$的线性变换,$\mathrm{rank}(\mc{T})=r$且$\mc{T}^2=3\mc{T}$,
    证明:存在$\bs{V}$的一组基,使$\mc{T}$在这组基下的矩阵为$\begin{pmatrix}
        \bs{O} &\bs{O}\\
        \bs{O} &3\bs{E}_r
    \end{pmatrix}$,其中$\bs{E}_r$为$r$阶单位阵。
\end{question}

\begin{proof}
    由题意与Hamilton-Cayley定理可知,$\lambda=3$或$0$。
    取$n$维线性空间$\bs{V}$下的一组标准正交基,并记线性变换$\mc{T}$在该组基下的矩阵为$\bs{A}$。
    \begin{enumerate}[label=(\arabic{*})]
        \item 若$r=n$,则$\lambda=3$,$m_{\bs{A}}(\lambda)=\lambda-3$,说明$\bs{A}$可相似对角化。
        则存在可逆矩阵$\bs{P}$,
        使得$\bs{P}^{-1}\bs{A}\bs{P}=\bs{\Lambda}$,其中$\bs{\Lambda}=\mathrm{diag}\{3,\ldots,3\}=3\bs{E}$,满足题意。
        \item 若$r=0$,则$\lambda=0$,$m_{\bs{A}}(\lambda)=\lambda$,说明$\bs{A}$可相似对角化。
        则存在可逆矩阵$\bs{P}$,
        使得$\bs{P}^{-1}\bs{A}\bs{P}=\bs{\Lambda}$,其中$\bs{\Lambda}=\mathrm{diag}\{0,\ldots,0\}=\bs{O}$,满足题意。
        \item 若$0<r<n$,则$\lambda=3$或$0$,$m_{\bs{A}}(\lambda)=\lambda(\lambda-3)$,说明$\bs{A}$可相似对角化。
        则存在可逆矩阵$\bs{P}$,
        使得$\bs{P}^{-1}\bs{A}\bs{P}=\bs{\Lambda}$,
        其中$\bs{\Lambda}=\mathrm{diag}\{0,\ldots,0,\underbrace{3,\ldots,3}_{r\text{个}3}\}=
        \begin{pmatrix}
            \bs{O} &\bs{O}\\
            \bs{O} &3\bs{E}_r
        \end{pmatrix}$,满足题意。
    \end{enumerate}
\end{proof}

\ifx\allfiles\undefined
\end{document}
\fi