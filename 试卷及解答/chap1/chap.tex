\ifx\allfiles\undefined
\documentclass[12pt, a4paper, oneside, UTF8]{ctexbook}
\def\configPath{../config}
\def\coverPath{\configPath/cover}
\def\packagePath{\configPath/package}
\def\theormPath{\configPath/theorem}
\def\customPath{\configPath/custom}
\def\prefacePath{\configPath/preface}

% 在这里定义需要的包
\usepackage{amsmath}
\usepackage{amsthm}
\usepackage{amssymb}
\usepackage{graphicx}
\usepackage{mathrsfs}
\usepackage{enumitem}
\usepackage{geometry}
\usepackage[colorlinks, linkcolor=black]{hyperref}
\usepackage{stackengine}
\usepackage{yhmath}
\usepackage{extarrows}
\usepackage{arydshln}
% \usepackage{unicode-math}
\usepackage{tasks}
\usepackage{fancyhdr}
\usepackage[dvipsnames, svgnames]{xcolor}
\usepackage{listings}


\input{\theormPath/theorem1_zh}
\input{\customPath/custom}




\begin{document}
\else
\fi

\chapter{模拟卷一}
\begin{question} 
   设$\bs{A}=\begin{pmatrix}
    1&0&2\\
    0&2&2\\
    1&1&3
   \end{pmatrix},\bs{\beta}=\begin{pmatrix}
    1\\
    1\\
    2
   \end{pmatrix}$,求不相容方程组$\bs{A}\bs{x}=\bs{\beta}$的最优最小二乘解。
\end{question}

\begin{solution}
    对矩阵$\bs{A}$进行满秩分解,得:
    $\bs{A}=\bs{B}\bs{C}=\begin{pmatrix}
        1 & 0\\
        0&2\\
        1&1
    \end{pmatrix}\begin{pmatrix}
        1&0&2\\
        0&1&1
    \end{pmatrix}$,于是
    \begin{align*}
        \bs{C}\bs{C}^T=\begin{pmatrix}
            5&2\\
            2&2
        \end{pmatrix} \quad \bs{B}^T\bs{B}=\begin{pmatrix}
            2&1\\
            1&5
        \end{pmatrix}
    \end{align*}
    \begin{align*}
        \bs{A}^+&=
    \bs{C}^T(\bs{C}\bs{C}^T)^{-1}(\bs{B}^T\bs{B})^{-1}\bs{B}^T\\
    &=\begin{pmatrix}
        1&0\\
        0&1\\
        2&1
    \end{pmatrix}\begin{pmatrix}
        5&2\\
        2&2
    \end{pmatrix}^{-1}\begin{pmatrix}
        2&1\\
        1&5
    \end{pmatrix}^{-1}
    \begin{pmatrix}
        1&0&1\\
        0&2&1
    \end{pmatrix}\\
    &=\frac{1}{18}\begin{pmatrix}
       4&-4&2\\
       -5&8&-1\\
       3&0&3 
    \end{pmatrix}
    \end{align*}

    \begin{align*}
        \bs{x}^*=\bs{A}^+\bs{\beta}
        =\frac{1}{18}\begin{pmatrix}
            4&-4&2\\
            -5&8&-1\\
            3&0&3 
         \end{pmatrix}\begin{pmatrix}
            1\\
            1\\
            2
           \end{pmatrix}
           =\frac{1}{18}\begin{pmatrix}
            4\\
            1\\
            9
           \end{pmatrix}
    \end{align*}
\end{solution}

\begin{question}
    设$\bs{A}=\begin{pmatrix}
        9&-4&-7\\
        -1&0&1\\
        10&-4&-8
    \end{pmatrix}$,求$\bs{A}$的谱分解。
\end{question}

\begin{solution}
    特征值为$\lambda_1=-2,\lambda_2=-1,\lambda_3=0$。
    其对应的特征向量分别是$(1,0,1)^T,(1,-1,2)^T,(2,1,2)^T$。
    于是:
    \begin{align*}
        \bs{A}
        &=\bs{P}\bs{\Lambda}\bs{P}^{-1}\\
        &=
        \begin{pmatrix}
            1&1&2\\
            0&-1&1\\
            1&2&2
        \end{pmatrix}\begin{pmatrix}
            -2& & \\
            & -1& \\
            & & 0
        \end{pmatrix}\begin{pmatrix}
            4	&-2	&-3\\
            -1&0&1\\
            -1&	1&	1
        \end{pmatrix}\\
        &=-2\begin{pmatrix}
            1\\
            0\\
            1
        \end{pmatrix}\begin{pmatrix}
            4&-2&-3
        \end{pmatrix}
        -\begin{pmatrix}
            1\\
            -1\\
            2
        \end{pmatrix}\begin{pmatrix}
            -1&0&1
        \end{pmatrix}
    \end{align*}
    
\end{solution}


\begin{question}
    设$\bs{A}=\begin{pmatrix}
        9&-6&-7\\
        -1&-1&1\\
        10&-6&-8
    \end{pmatrix}$,求可逆阵$\bs{P}$和若当(Jordan)标准型$\bs{J}$,使$\bs{P}^{-1}\bs{A}\bs{P}=\bs{J}$,并求$e^{2\bs{A}t}$。
\end{question}

\begin{solution}
    特征值为$\lambda_1=2,\lambda_2=\lambda_3=-1$,对应的若当标准型为$\bs{J}=\begin{pmatrix}
        2 & & \\
        & -1& 1\\
        & & -1
    \end{pmatrix}$,其中空白位置全是$0$。
    由于$\bs{J}=\bs{P}^{-1}\bs{A}\bs{P} \Rightarrow \bs{P}\bs{J}=\bs{A}\bs{P}$,
    不妨令$\bs{P}=(\bs{p}_1,\bs{p}_2,\bs{p}_3)$,其中$\bs{P}$为非奇异矩阵,则:
    \begin{align*}
        \left\{
            \begin{array}{ll}
                \bs{A}\bs{p}_1=2\bs{p}_1 \\
                \bs{A}\bs{p}_2=-\bs{p}_2 \\
                \bs{A}\bs{p}_3=\bs{p}_2-\bs{p}_3
            \end{array} 
            \right.
            \Rightarrow
            \left\{
            \begin{array}{ll}
                \bs{p}_1=(1,0,1)^T \\
                \bs{p}_2=(2,1,2)^T \\
                \bs{p}_3=(1,-1,2)^T
            \end{array} 
            \right.
    \end{align*}
    \begin{align*}
        \bs{P}=\begin{pmatrix}
            1 & 2 &1 \\
            0 & 1 &-1 \\
            1 & 2 &2
        \end{pmatrix} \quad
        \bs{P}^{-1}=\begin{pmatrix}
            4&-2&-3\\
            -1&1&1\\
            -1&0&1       
        \end{pmatrix}
    \end{align*}
    于是:
    \begin{align*}
        e^{2\bs{A}t}&=\bs{P}e^{\bs{J}t}\bs{P}^{-1} \\
        &=\begin{pmatrix}
            1 & 2 &1 \\
            0 & 1 &-1 \\
            1 & 2 &2
        \end{pmatrix} \begin{pmatrix}
            e^{4t}& & \\
            & e^{-2t}&2te^{-2t} \\
            & & e^{-2t}
        \end{pmatrix}\begin{pmatrix}
            4&-2&-3\\
            -1&1&1\\
            -1&0&1       
        \end{pmatrix}\\
        &=\begin{pmatrix}
            4e^{4t}-(2t+3)e^{-2t}&-2e^{4t}+2e^{-2t}&-3e^{4t}+(2t+3)e^{-2t}\\
            -te^{-2t}&e^{-2t}&te^{-2t}\\
            4e^{4t}-(2t+4)e^{-2t}&-2e^{4t}+2e^{-2t}&-3e^{4t}+(2t+4)e^{-2t}
        \end{pmatrix}
    \end{align*}
\end{solution}

\begin{question}
    用矩阵函数求解常微分方程组初值问题的解
    \begin{align*}
    \left\{
        \begin{array}{ll}
            \frac{\d x_1}{\d t}=-3x_1+4x_2\\
            \frac{\d x_2}{\d t}=-x_1+x_2+1
        \end{array}
        \right.
    \left\{
        \begin{array}{ll}
            x_1(0)=1\\
            x_2(0)=0
        \end{array}
        \right.
    \end{align*}
\end{question}

\begin{solution}
    由题意得:$\frac{\d \bs{x}}{\d t}=\bs{A}\bs{x}+\bs{b}$,其中$\bs{A}=\begin{pmatrix}
        -3&4\\
        -1&1
    \end{pmatrix},\bs{b}=(0,1)^T$。
    矩阵$\bs{A}$的特征值为$\lambda_1=\lambda_2=-1$,$m_{\bs{A}}(\lambda)=(\lambda+1)^2$
    不妨设$P(\lambda)=a_0+a_1\lambda$,则 \begin{align*}
        \left\{
            \begin{array}{ll}
                P(\lambda)=P(-1)=a_0-a_1=e^{-t}\\
                P'(\lambda)=P'(-1)=a_1=te^{-t}
            \end{array}
            \right.
        \Rightarrow
        \left\{
            \begin{array}{ll}
                a_0=(1+t)e^{-t}\\
                a_1=te^{-t}
            \end{array}
            \right.
    \end{align*}
    \begin{align*}
        e^{\bs{A}t}=P(\bs{A})
    &=(1+t)e^{-t}\bs{E}+te^{-t}\bs{A}\\
    &=\begin{pmatrix}
        (-2t+1)e^{-t} &4te^{-t} \\
        -te^{-t}  & (2t+1)e^{-t}
    \end{pmatrix}
    \end{align*}
    \begin{align*}
        \bs{x}(t)&=e^{\bs{A}t}\bs{x}(0)+e^{\bs{A}t}\int_{0}^t e^{-\bs{A}u}\bs{b} du\\
        &=\begin{pmatrix}
            (-2t+1)e^{-t} &4te^{-t} \\
        -te^{-t}  & (2t+1)e^{-t}
        \end{pmatrix}\begin{pmatrix}
            1 \\
            0
        \end{pmatrix}\\
        &+\begin{pmatrix}
            (-2t+1)e^{-t} &4te^{-t} \\
        -te^{-t}  & (2t+1)e^{-t}
        \end{pmatrix}\int_{0}^t 
        \begin{pmatrix}
            (2u+1)e^{u} &-4ue^{u} \\
            ue^{u}  & (-2u+1)e^{u}
        \end{pmatrix} \begin{pmatrix}
            0\\
            1
        \end{pmatrix} du \\
        &=\begin{pmatrix}
            (-6t-3)e^{-t}+4 \\
            (-3t-3)e^{-t}+3
        \end{pmatrix}
        \end{align*}
\end{solution}

\begin{question}
    设$\bs{V}$是二阶实方阵全体,$\bs{C}=\begin{pmatrix}
        1&1\\
        0&0
    \end{pmatrix}$,对任意$\bs{A}\in \bs{V}$,令$\mc{T}(\bs{A})=\bs{A}\bs{C}+\bs{C}\bs{A}$,
    证明$\mc{T}$是$\bs{V}$的线性变换。
    \begin{enumerate}[label=(\arabic{*})]
        \item 求$\mc{T}$在$\bs{V}$的基$\bs{B}_1=\begin{pmatrix}
            1&-1\\
            0&0
        \end{pmatrix},\bs{B}_2=\begin{pmatrix}
            2&0\\
            0&0
        \end{pmatrix},\bs{B}_3=\begin{pmatrix}
            0&0\\
            1&1
        \end{pmatrix},\bs{B}_4=\begin{pmatrix}
            0&0\\
            1&2
        \end{pmatrix}$下的矩阵表示。
        \item 求$\mc{T}$的特征值。
        \item 判别$\mc{T}$是否可对角化。
    \end{enumerate}
\end{question}
\begin{solution}
    \begin{enumerate}[label=(\arabic{*})]
        \item 
        \begin{align*}
            \mc{T}\bs{B}_1=\begin{pmatrix}
                2&0\\
                0&0
            \end{pmatrix},\mc{T}\bs{B}_2=\begin{pmatrix}
                4&2\\
                0&0
            \end{pmatrix},\mc{T}\bs{B}_3=\begin{pmatrix}
                1&1\\
                1&1
            \end{pmatrix},\mc{T}\bs{B}_4=\begin{pmatrix}
                1&2\\
                1&1
            \end{pmatrix}
        \end{align*}
        即
        \begin{align*}
            \mc{T}(\bs{B}_1,\bs{B}_2,\bs{B}_3,\bs{B}_4)=(\bs{B}_1,\bs{B}_2,\bs{B}_3,\bs{B}_4)
            \begin{pmatrix}
                0& -2&-1&-2  \\
                1 & 3&1 &\frac{3}{2} \\
                0& 0& 1& 1\\
                0& 0& 0&0
            \end{pmatrix}
        \end{align*}
        \item $\lambda_1=\lambda_2=1,\lambda_3=2,\lambda_4=0$。
        \item 可对角化,这是由于$\lambda=1$的特征值有至少两个线性无关的特征向量。
    \end{enumerate} 
\end{solution}



\begin{question}
    设$\mc{T}$是$n$维线性空间$\bs{V}$的线性变换且$\mc{T}^2=3\mc{T}$,证明:$\bs{V}=\mathrm{Im}\mc{T}\oplus \mathrm{Ker}\mc{T}$
,其中$\mathrm{Im}\mc{T}$是$\mc{T}$的像空间,$\mathrm{Ker}\mc{T}$是$\mc{T}$的核空间。
\end{question}

\begin{proof}
    由于$\mathrm{Im}\mc{T}$和$\mathrm{Ker}\mc{T}$均为$\bs{V}$的子空间,则
    $\mathrm{Im}\mc{T}+ \mathrm{Ker}\mc{T} \subset \bs{V}$。
    任取$\bs{\alpha}\in \mathrm{Im}\mc{T}\cap \mathrm{Ker}\mc{T}$,则存在$\bs{\beta}\in \bs{V}$,
    使得$\mc{T}\bs{\alpha}=\bs{0},\mc{T} \bs{\beta}=\bs{\alpha} $,
    于是$3\bs{\alpha}=3\mc{T}\bs{\beta}=\mc{T}(\mc{T} \bs{\beta})=
    \mc{T}\bs{\alpha}=\bs{0}$,即$\bs{\alpha}=\bs{0}$,此时$\mathrm{Im}\mc{T}\cap \mathrm{Ker}\mc{T}=\{\bs{0}\}$,
    这说明$\mathrm{Im}\mc{T}$和$\mathrm{Ker}\mc{T}$构成直和。
    又因为$\mathrm{dim}(\mathrm{Im}\mc{T}+\mathrm{Ker}\mc{T})=\mathrm{dim}(\mathrm{Im}\mc{T})
    +\mathrm{dim}(\mathrm{Ker}\mc{T})=n$,所以$\mathrm{Im}\mc{T}+ \mathrm{Ker}\mc{T} = \bs{V}$。
    综上,$\mathrm{Im}\mc{T}\oplus \mathrm{Ker}\mc{T} = \bs{V}$。
    
\end{proof}


\ifx\allfiles\undefined
\end{document}
\fi