\ifx\allfiles\undefined
\documentclass[12pt, a4paper, oneside, UTF8]{ctexbook}
\def\configPath{../config}
\def\basicPath{\configPath/basic}


% 在这里定义需要的包
\usepackage{amsmath}
\usepackage{amsthm}
\usepackage{amssymb}
\usepackage{graphicx}
\usepackage{mathrsfs}
\usepackage{enumitem}
\usepackage{geometry}
\usepackage[colorlinks, linkcolor=black]{hyperref}
\usepackage{stackengine}
\usepackage{yhmath}
\usepackage{extarrows}
\usepackage{arydshln}
% \usepackage{unicode-math}
\usepackage{tasks}
\usepackage{fancyhdr}
\usepackage[dvipsnames, svgnames]{xcolor}
\usepackage{listings}

\definecolor{mygreen}{rgb}{0,0.6,0}
\definecolor{mygray}{rgb}{0.5,0.5,0.5}
\definecolor{mymauve}{rgb}{0.58,0,0.82}

\graphicspath{ {figure/},{../figure/}, {config/}, {../config/} }

\linespread{1.6}

\geometry{
    top=25.4mm, 
    bottom=25.4mm, 
    left=20mm, 
    right=20mm, 
    headheight=2.17cm, 
    headsep=4mm, 
    footskip=12mm
}

\setenumerate[1]{itemsep=5pt,partopsep=0pt,parsep=\parskip,topsep=5pt}
\setitemize[1]{itemsep=5pt,partopsep=0pt,parsep=\parskip,topsep=5pt}
\setdescription{itemsep=5pt,partopsep=0pt,parsep=\parskip,topsep=5pt}

\lstset{
    language=Mathematica,
    basicstyle=\tt,
    breaklines=true,
    keywordstyle=\bfseries\color{NavyBlue}, 
    emphstyle=\bfseries\color{Rhodamine},
    commentstyle=\itshape\color{black!50!white}, 
    stringstyle=\bfseries\color{PineGreen!90!black},
    columns=flexible,
    numbers=left,
    numberstyle=\footnotesize,
    frame=tb,
    breakatwhitespace=false,
} 
% 在这里定义自己顺手的环境
\def\d{\mathrm{d}}
\def\i{\mathrm{i}}
\def\R{\mathbb{R}}
\newcommand{\bs}[1]{\boldsymbol{#1}}
\newcommand{\mc}[1]{\mathcal{#1}}
\newcommand{\ora}[1]{\overrightarrow{#1}}
\newcommand{\myspace}[1]{\par\vspace{#1\baselineskip}}
\newcommand{\xrowht}[2][0]{\addstackgap[.5\dimexpr#2\relax]{\vphantom{#1}}}
\newenvironment{ca}[1][1]{\linespread{#1} \selectfont \begin{cases}}{\end{cases}}
\newenvironment{vx}[1][1]{\linespread{#1} \selectfont \begin{vmatrix}}{\end{vmatrix}}
\newcommand{\tabincell}[2]{\begin{tabular}{@{}#1@{}}#2\end{tabular}}
\newcommand{\pll}{\kern 0.56em/\kern -0.8em /\kern 0.56em}
\newcommand{\dive}[1][F]{\mathrm{div}\;\bs{#1}}
\newcommand{\rotn}[1][A]{\mathrm{rot}\;\bs{#1}} 
\usepackage[strict]{changepage} 
\usepackage{framed}

\definecolor{greenshade}{rgb}{0.90,1,0.92}
\definecolor{redshade}{rgb}{1.00,0.88,0.88}
\definecolor{brownshade}{rgb}{0.99,0.95,0.9}
\definecolor{lilacshade}{rgb}{0.95,0.93,0.98}
\definecolor{orangeshade}{rgb}{1.00,0.88,0.82}
\definecolor{lightblueshade}{rgb}{0.8,0.92,1}
\definecolor{purple}{rgb}{0.81,0.85,1}
\theoremstyle{definition}
\newtheorem{myDefn}{\indent 定义}[section]
% \newtheorem{myLemma}{\indent 引理}[section]
\newtheorem{myLemma}{\indent 引理}[chapter]
\newtheorem{myThm}[myLemma]{\indent 定理}
\newtheorem{myCorollary}[myLemma]{\indent 推论}
\newtheorem{myCriterion}[myLemma]{\indent 准则}
\newtheorem*{myRemark}{\indent 注}
\newtheorem{myProposition}{\indent 命题}[section]


\newenvironment{formal}[2][]{%
    \def\FrameCommand{%
        \hspace{1pt}%
        {\color{#1}\vrule width 2pt}%
        {\color{#2}\vrule width 4pt}%
        \colorbox{#2}%
    }%
    \MakeFramed{\advance\hsize-\width\FrameRestore}%
    \noindent\hspace{-4.55pt}%
    \begin{adjustwidth}{}{7pt}\vspace{2pt}\vspace{2pt}}{%
        \vspace{2pt}\end{adjustwidth}\endMakeFramed%
}

\newenvironment{defn}{\begin{formal}[Green]{greenshade}\vspace{-\baselineskip / 2}\begin{myDefn}}{\end{myDefn}\end{formal}}
\newenvironment{thm}{\begin{formal}[LightSkyBlue]{lightblueshade}\vspace{-\baselineskip / 2}\begin{myThm}}{\end{myThm}\end{formal}}
\newenvironment{lemma}{\begin{formal}[Plum]{lilacshade}\vspace{-\baselineskip / 2}\begin{myLemma}}{\end{myLemma}\end{formal}}
\newenvironment{corollary}{\begin{formal}[BurlyWood]{brownshade}\vspace{-\baselineskip / 2}\begin{myCorollary}}{\end{myCorollary}\end{formal}}
\newenvironment{criterion}{\begin{formal}[DarkOrange]{orangeshade}\vspace{-\baselineskip / 2}\begin{myCriterion}}{\end{myCriterion}\end{formal}}
\newenvironment{rmk}{\begin{formal}[LightCoral]{redshade}\vspace{-\baselineskip / 2}\begin{myRemark}}{\end{myRemark}\end{formal}}
\newenvironment{proposition}{\begin{formal}[RoyalPurple]{purple}\vspace{-\baselineskip / 2}\begin{myProposition}}{\end{myProposition}\end{formal}}

\newtheorem{example}{\indent \color{SeaGreen}{例}}[section]
\newtheorem{question}{\color{SeaGreen}{题}}[chapter]
% \renewenvironment{proof}{\indent\textcolor{SkyBlue}{\textbf{证明.}}\;}{\qed\par}
% \newenvironment{solution}{\indent\textcolor{SkyBlue}{\textbf{解.}}\;}{\qed\par}

\renewcommand{\proofname}{\textbf{\textcolor{TealBlue}{证明}}}
\newenvironment{solution}{\begin{proof}[\textbf{\textcolor{TealBlue}{解}}]}{\end{proof}}

\definecolor{mygreen}{rgb}{0,0.6,0}
\definecolor{mygray}{rgb}{0.5,0.5,0.5}
\definecolor{mymauve}{rgb}{0.58,0,0.82}

\graphicspath{ {figure/},{../figure/}, {config/}, {../config/},{cover/graph} }

\linespread{1.6}

\geometry{
    top=25.4mm, 
    bottom=25.4mm, 
    left=20mm, 
    right=20mm, 
    headheight=2.17cm, 
    headsep=4mm, 
    footskip=12mm
}

\setenumerate[1]{itemsep=5pt,partopsep=0pt,parsep=\parskip,topsep=5pt}
\setitemize[1]{itemsep=5pt,partopsep=0pt,parsep=\parskip,topsep=5pt}
\setdescription{itemsep=5pt,partopsep=0pt,parsep=\parskip,topsep=5pt}

\lstset{
    language=Mathematica,
    basicstyle=\tt,
    breaklines=true,
    keywordstyle=\bfseries\color{NavyBlue}, 
    emphstyle=\bfseries\color{Rhodamine},
    commentstyle=\itshape\color{black!50!white}, 
    stringstyle=\bfseries\color{PineGreen!90!black},
    columns=flexible,
    numbers=left,
    numberstyle=\footnotesize,
    frame=tb,
    breakatwhitespace=false,
} 

\begin{document}
\else
\fi

\chapter{模拟卷四}
\begin{question} 
   设$\bs{A}=\begin{pmatrix}
    1&-2\\
    -1&0
   \end{pmatrix}$,
   求$\bs{A}$的谱分解。
\end{question}

\begin{question}
    设$\bs{A}=\begin{pmatrix}
        2&0&0\\
        -3&1&-1\\
        3&1&3
    \end{pmatrix}$,求$\bs{A}$的若当(Jordan)标准型$\bs{J}$。
\end{question}

\begin{question}
    设$\bs{a}_1,\bs{a}_2,\bs{a}_3$为内积空间$\bs{V}$的一个标准正交基,
    $\bs{b}_1=\bs{a}_1+\bs{a}_2,\bs{b}_2=\bs{a}_2-\bs{a}_3,\bs{b}_3=\bs{a}_3+\bs{a}_1$,
    $\bs{S}=<\bs{b}_1,\bs{b}_2,\bs{b}_3>$为由$\bs{b}_1,\bs{b}_2,\bs{b}_3$生成的子空间。
    \begin{enumerate}[label=(\arabic{*})]
        \item 求$\bs{S}$的维数。
        \item 求$\bs{S}$的一个标准正交基(用$\bs{a}_1,\bs{a}_2,\bs{a}_3$表示)。
    \end{enumerate}
\end{question}

\begin{question}
    设$\bs{x}=\begin{pmatrix}
        1\\
        2\\
        2
    \end{pmatrix}$,求$\R^3$中的初等反射矩阵$\bs{H}$,使$\bs{H}\bs{x}$与$\bs{\beta}=\begin{pmatrix}
        1\\
        0\\
        0
    \end{pmatrix}$同方向。
\end{question}

\begin{question}
    设方程$\bs{A}\bs{x}=\bs{b}$,其中$\bs{A}=\begin{pmatrix}
        -1&2&1\\
        -1&2&1\\
        0&3&1
    \end{pmatrix},\bs{b}=\begin{pmatrix}
        -1\\
        1\\
        1
    \end{pmatrix}$
    \begin{enumerate}[label=(\arabic{*})]
        \item 求$\bs{A}$的满秩分解(记为$\bs{A}=\bs{B}\bs{C}$)。
        \item 说明方程$\bs{A}\bs{x}=\bs{b}$为矛盾方程。
        \item 求方程$\bs{A}\bs{x}=\bs{b}$的长度最小的最小二乘解和最小二乘解通解。
    \end{enumerate}
\end{question}

\begin{question}
    用矩阵函数求解常微分方程组初值问题的解
    \begin{align*}
    \left\{
        \begin{array}{ll}
            \frac{\d x_1}{\d t}=x_1-x_2\\
            \frac{\d x_2}{\d t}=4x_1-3x_2+1
        \end{array}
        \right.
    \left\{
        \begin{array}{ll}
            x_1(0)=1\\
            x_2(0)=2
        \end{array}
        \right.
    \end{align*}
\end{question}

\begin{question}

    设$\mc{D}$是三维线性空间$\bs{V}=\{(a_2t^2+a_1t+a_0)e^t|a_2,a_1,a_0 \in \R\}$
    中的微分线性变换,$f_1=t^2e^t,f_2=te^t,f_3=e^t$为$\bs{V}$的一个基。
    \begin{enumerate}[label=(\arabic{*})]
        \item 求$\mc{D}$在该基下的矩阵。
        \item 求$\mathrm{Im}\mc{D}$和$\mathrm{Ker}\mc{D}$,其中$\mathrm{Im}\mc{D}$是$\mc{D}$的像空间,$\mathrm{Ker}\mc{D}$是$\mc{D}$的核空间。
    \end{enumerate}
\end{question}

\begin{question}
    设$\R^{2\times 2}$是二阶实方阵在方阵运算下构成的线性空间,对任意$\bs{A}\in \R^{2\times 2}$,
    令$\mc{T}(\bs{A})=\bs{A}^T+\bs{A}$,
    \begin{enumerate}[label=(\arabic{*})]
        \item 证明$\mc{T}$是$\R^{2\times 2}$的线性变换。
        \item 判断$\mc{T}$是否可对角化,并说明理由。
    \end{enumerate}
\end{question}

\begin{question}
    设$\mc{T}$是线性空间$\bs{V}$的线性变换且,$\mc{T}^2=\mc{T}$,证明:$\bs{V}=\mathrm{Im}\mc{T}\oplus \mathrm{Ker}\mc{T}$
,其中$\mathrm{Im}\mc{T}$是$\mc{T}$的像空间,$\mathrm{Ker}\mc{T}$是$\mc{T}$的核空间。
\end{question}

\ifx\allfiles\undefined
\end{document}
\fi