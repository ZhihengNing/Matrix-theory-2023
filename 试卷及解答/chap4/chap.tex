\ifx\allfiles\undefined
\documentclass[12pt, a4paper, oneside, UTF8]{ctexbook}
\def\configPath{../config}
\def\coverPath{\configPath/cover}
\def\packagePath{\configPath/package}
\def\theormPath{\configPath/theorem}
\def\customPath{\configPath/custom}
\def\prefacePath{\configPath/preface}

% 在这里定义需要的包
\usepackage{amsmath}
\usepackage{amsthm}
\usepackage{amssymb}
\usepackage{graphicx}
\usepackage{mathrsfs}
\usepackage{enumitem}
\usepackage{geometry}
\usepackage[colorlinks, linkcolor=black]{hyperref}
\usepackage{stackengine}
\usepackage{yhmath}
\usepackage{extarrows}
\usepackage{arydshln}
% \usepackage{unicode-math}
\usepackage{tasks}
\usepackage{fancyhdr}
\usepackage[dvipsnames, svgnames]{xcolor}
\usepackage{listings}


\input{\theormPath/theorem1_zh}
\input{\customPath/custom}




\begin{document}
\else
\fi

\chapter{模拟卷四}
\begin{question} 
   设$\bs{A}=\begin{pmatrix}
    1&-2\\
    -1&0
   \end{pmatrix}$,
   求$\bs{A}$的谱分解。
\end{question}

\begin{question}
    设$\bs{A}=\begin{pmatrix}
        2&0&0\\
        -3&1&-1\\
        3&1&3
    \end{pmatrix}$,求$\bs{A}$的若当(Jordan)标准型$\bs{J}$。
\end{question}

\begin{question}
    设$\bs{a}_1,\bs{a}_2,\bs{a}_3$为内积空间$\bs{V}$的一个标准正交基,
    $\bs{b}_1=\bs{a}_1+\bs{a}_2,\bs{b}_2=\bs{a}_2-\bs{a}_3,\bs{b}_3=\bs{a}_3+\bs{a}_1$,
    $\bs{S}=<\bs{b}_1,\bs{b}_2,\bs{b}_3>$为由$\bs{b}_1,\bs{b}_2,\bs{b}_3$生成的子空间。
    \begin{enumerate}[label=(\arabic{*})]
        \item 求$\bs{S}$的维数。
        \item 求$\bs{S}$的一个标准正交基(用$\bs{a}_1,\bs{a}_2,\bs{a}_3$表示)。
    \end{enumerate}
\end{question}

\begin{question}
    设$\bs{x}=\begin{pmatrix}
        1\\
        2\\
        2
    \end{pmatrix}$,求$\R^3$中的初等反射矩阵$\bs{H}$,使$\bs{H}\bs{x}$与$\bs{\beta}=\begin{pmatrix}
        1\\
        0\\
        0
    \end{pmatrix}$同方向。
\end{question}

\begin{question}
    设方程$\bs{A}\bs{x}=\bs{b}$,其中$\bs{A}=\begin{pmatrix}
        -1&2&1\\
        -1&2&1\\
        0&3&1
    \end{pmatrix},\bs{b}=\begin{pmatrix}
        -1\\
        1\\
        1
    \end{pmatrix}$
    \begin{enumerate}[label=(\arabic{*})]
        \item 求$\bs{A}$的满秩分解(记为$\bs{A}=\bs{B}\bs{C}$)。
        \item 说明方程$\bs{A}\bs{x}=\bs{b}$为矛盾方程。
        \item 求方程$\bs{A}\bs{x}=\bs{b}$的长度最小的最小二乘解和最小二乘解通解。
    \end{enumerate}
\end{question}

\begin{question}
    用矩阵函数求解常微分方程组初值问题的解
    \begin{align*}
    \left\{
        \begin{array}{ll}
            \frac{\d x_1}{\d t}=x_1-x_2\\
            \frac{\d x_2}{\d t}=4x_1-3x_2+1
        \end{array}
        \right.
    \left\{
        \begin{array}{ll}
            x_1(0)=1\\
            x_2(0)=2
        \end{array}
        \right.
    \end{align*}
\end{question}

\begin{question}

    设$\mc{D}$是三维线性空间$\bs{V}=\{(a_2t^2+a_1t+a_0)e^t|a_2,a_1,a_0 \in \R\}$
    中的微分线性变换,$f_1=t^2e^t,f_2=te^t,f_3=e^t$为$\bs{V}$的一个基。
    \begin{enumerate}[label=(\arabic{*})]
        \item 求$\mc{D}$在该基下的矩阵。
        \item 求$\mathrm{Im}\mc{D}$和$\mathrm{Ker}\mc{D}$,其中$\mathrm{Im}\mc{D}$是$\mc{D}$的像空间,$\mathrm{Ker}\mc{D}$是$\mc{D}$的核空间。
    \end{enumerate}
\end{question}

\begin{question}
    设$\R^{2\times 2}$是二阶实方阵在方阵运算下构成的线性空间,对任意$\bs{A}\in \R^{2\times 2}$,
    令$\mc{T}(\bs{A})=\bs{A}^T+\bs{A}$,
    \begin{enumerate}[label=(\arabic{*})]
        \item 证明$\mc{T}$是$\R^{2\times 2}$的线性变换。
        \item 判断$\mc{T}$是否可对角化,并说明理由。
    \end{enumerate}
\end{question}

\begin{question}
    设$\mc{T}$是线性空间$\bs{V}$的线性变换且,$\mc{T}^2=\mc{T}$,证明:$\bs{V}=\mathrm{Im}\mc{T}\oplus \mathrm{Ker}\mc{T}$
,其中$\mathrm{Im}\mc{T}$是$\mc{T}$的像空间,$\mathrm{Ker}\mc{T}$是$\mc{T}$的核空间。
\end{question}

\ifx\allfiles\undefined
\end{document}
\fi