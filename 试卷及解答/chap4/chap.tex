\ifx\allfiles\undefined
\documentclass[12pt, a4paper, oneside, UTF8]{ctexbook}
\def\configPath{../config}
\def\coverPath{\configPath/cover}
\def\packagePath{\configPath/package}
\def\theormPath{\configPath/theorem}
\def\customPath{\configPath/custom}
\def\prefacePath{\configPath/preface}

% 在这里定义需要的包
\usepackage{amsmath}
\usepackage{amsthm}
\usepackage{amssymb}
\usepackage{graphicx}
\usepackage{mathrsfs}
\usepackage{enumitem}
\usepackage{geometry}
\usepackage[colorlinks, linkcolor=black]{hyperref}
\usepackage{stackengine}
\usepackage{yhmath}
\usepackage{extarrows}
\usepackage{arydshln}
% \usepackage{unicode-math}
\usepackage{tasks}
\usepackage{fancyhdr}
\usepackage[dvipsnames, svgnames]{xcolor}
\usepackage{listings}


\input{\theormPath/theorem1_zh}
\input{\customPath/custom}




\begin{document}
\else
\fi

\chapter{模拟卷四}
\begin{question} 
   设$\bs{A}=\begin{pmatrix}
    1&-2\\
    -1&0
   \end{pmatrix}$,
   求$\bs{A}$的谱分解。
\end{question}

\begin{solution}
    特征值为$\lambda_1=2,\lambda_2=-1$。
    其对应的特征向量分别是$(2,-1)^T,(1,1)^T$。
    于是:
    \begin{align*}
        \bs{A}
        &=\bs{P}\bs{\Lambda}\bs{P}^{-1}\\
        &=
        \begin{pmatrix}
            2&1\\
            -1&1
        \end{pmatrix}\begin{pmatrix}
            2&  \\
            & -1
        \end{pmatrix}\begin{pmatrix}
            \frac{1}{3}&-\frac{1}{3}\\
            \frac{1}{3}&\frac{2}{3}
        \end{pmatrix}\\
        &=2\begin{pmatrix}
            2\\
            -1
        \end{pmatrix}\begin{pmatrix}
            \frac{1}{3}&-\frac{1}{3}
        \end{pmatrix}
        -\begin{pmatrix}
            1\\
            1
        \end{pmatrix}\begin{pmatrix}
            \frac{1}{3}&\frac{2}{3}
        \end{pmatrix}
    \end{align*}
\end{solution}

\begin{question}
    设$\bs{A}=\begin{pmatrix}
        2&0&0\\
        -3&1&-1\\
        3&1&3
    \end{pmatrix}$,求$\bs{A}$的若当(Jordan)标准型$\bs{J}$。
\end{question}

\begin{solution}
    特征值为$\lambda_1=\lambda_2=\lambda_3=2$,对应的若当标准型$\bs{J}=\begin{pmatrix}
        2 & & \\
        & 2& 1\\
        & & 2
    \end{pmatrix}$,其中空白位置全是$0$。

\end{solution}


\begin{question}
    设$\bs{a}_1,\bs{a}_2,\bs{a}_3$为内积空间$\bs{V}$的一个标准正交基,
    $\bs{b}_1=\bs{a}_1+\bs{a}_2,\bs{b}_2=\bs{a}_2-\bs{a}_3,\bs{b}_3=\bs{a}_3+\bs{a}_1$,
    $\bs{S}=<\bs{b}_1,\bs{b}_2,\bs{b}_3>$为由$\bs{b}_1,\bs{b}_2,\bs{b}_3$生成的子空间。
    \begin{enumerate}[label=(\arabic{*})]
        \item 求$\bs{S}$的维数。
        \item 求$\bs{S}$的一个标准正交基(用$\bs{a}_1,\bs{a}_2,\bs{a}_3$表示)。
    \end{enumerate}
\end{question}

\begin{solution}
    \begin{enumerate}[label=(\arabic{*})]
        \item 由题意得,$(\bs{b}_1,\bs{b}_2,\bs{b}_3)=(\bs{a}_1,\bs{a}_2,\bs{a}_3)\bs{A}$,
        其中$\bs{A}=\begin{pmatrix}
            1&0&1\\
            1&1&0\\
            0&-1&1
        \end{pmatrix}$。由于$\mathrm{r}(\bs{A})=2$,
        则$\mathrm{dim}(\bs{S})=\mathrm{r}(\bs{b}_1,\bs{b}_2,\bs{b}_3)=2$。
        \item 显然,$\bs{b}_1,\bs{b}_2$构成了$\bs{S}$的一组基,
        对其进行施密特正交并单位化得,
        \begin{align*}
            \bs{e}_1&=\frac{\bs{b}_1}{||\bs{b}_1||}=\frac{\bs{a}_1+\bs{a}_2}{\sqrt{2}} \\
            \bs{e}_2&=\frac{1}{||\bs{e}_2||}(\bs{b}_2-\frac{(\bs{b}_2,\bs{e}_1)}{(\bs{e}_1,\bs{e}_1)}\bs{e}_1
            )=\frac{\bs{a}_2- 2\bs{a}_3-\bs{a}_1}{\sqrt{6}}
        \end{align*}
    \end{enumerate}
\end{solution}

\begin{question}
    设$\bs{x}=\begin{pmatrix}
        1\\
        2\\
        2
    \end{pmatrix}$,求$\R^3$中的初等反射矩阵$\bs{H}$,使$\bs{H}\bs{x}$与$\bs{\beta}=\begin{pmatrix}
        1\\
        0\\
        0
    \end{pmatrix}$同方向。
\end{question}

\begin{solution}
    不妨令$\bs{H}=\bs{E}-2\bs{\omega}\bs{\omega}^T$,其中$||\bs{\omega}||=1$。
    由于$||\bs{H}\bs{x}||=||\bs{x}||=3$,所以$\bs{H}\bs{x}=3\bs{\beta}=\begin{pmatrix}
        3\\
        0\\
        0
    \end{pmatrix}$。
    于是:
    \begin{align*}
        3\bs{\beta}=\bs{H}\bs{x}=\bs{H}(k\bs{\omega}+\bs{\alpha})=k\bs{H}\bs{\omega}+\bs{H}\bs{\alpha}=-k\bs{\omega}+\bs{\alpha}
    \end{align*}
    又因为$\R^3=<\bs{\omega}> \oplus <\bs{\omega}>^{\perp}$,所以设$\bs{x}=k\bs{\omega}+\bs{\alpha}$,
    其中$k \in \R,\bs{\alpha} \in <\bs{\omega}>^{\perp}$。
   
    联立两个式子解得,$\bs{\omega}=\frac{\bs{x}-3\bs{\beta}}{2k}$。又因为$||\bs{\omega}||=1$,
    所以$\bs{\omega}=\frac{\bs{x}-3\bs{\beta}}{2\sqrt{3}}=\frac{1}{\sqrt{3}}\begin{pmatrix}
        -1 \\
        1\\
        1
    \end{pmatrix}$。

\end{solution}

\begin{question}
    设方程$\bs{A}\bs{x}=\bs{b}$,其中$\bs{A}=\begin{pmatrix}
        -1&2&1\\
        -1&2&1\\
        0&3&1
    \end{pmatrix},\bs{b}=\begin{pmatrix}
        -1\\
        1\\
        1
    \end{pmatrix}$
    \begin{enumerate}[label=(\arabic{*})]
        \item 求$\bs{A}$的满秩分解(记为$\bs{A}=\bs{B}\bs{C}$)。
        \item 说明方程$\bs{A}\bs{x}=\bs{b}$为矛盾方程。
        \item 求方程$\bs{A}\bs{x}=\bs{b}$的长度最小的最小二乘解和最小二乘解通解。
    \end{enumerate}
\end{question}

\begin{solution}
    \begin{enumerate}[label=(\arabic{*})]
        \item $\bs{A}=\bs{B}\bs{C}=\begin{pmatrix}
            -1&2\\
            -1&2\\
            0&3
        \end{pmatrix}\begin{pmatrix}
            1&0&-\frac{1}{3}\\
            0&1&\frac{1}{3}
        \end{pmatrix}$
        \item  
        \begin{align*}
            \bs{C}\bs{C}^T=\begin{pmatrix}
                \frac{10}{9}&-\frac{1}{9}\\
                -\frac{1}{9}&\frac{10}{9}
            \end{pmatrix} \quad \bs{B}^T\bs{B}=\begin{pmatrix}
                2&-4\\
                -4&17
            \end{pmatrix}
        \end{align*}
        \begin{align*}
            \bs{A}^+&=
        \bs{C}^T(\bs{C}\bs{C}^T)^{-1}(\bs{B}^T\bs{B})^{-1}\bs{B}^T\\
        &=\begin{pmatrix}
            1&0\\
            0&1\\
            -\frac{1}{3}&\frac{1}{3}
        \end{pmatrix}\begin{pmatrix}
            \frac{10}{9}&-\frac{1}{9}\\
                -\frac{1}{9}&\frac{10}{9}
        \end{pmatrix}^{-1}\begin{pmatrix}
            2&-4\\
                -4&17
        \end{pmatrix}^{-1}
        \begin{pmatrix}
            -1&-1&0\\
            2&2&3
        \end{pmatrix}\\
        &=\frac{1}{22}\begin{pmatrix}
           -10&-10&14\\
           -1&-1&8\\
           3&3&-2
        \end{pmatrix}
        \end{align*}
        则$\bs{A}\bs{A}^+\bs{b}=\begin{pmatrix}
            0\\
            0\\
            1
        \end{pmatrix} \neq \begin{pmatrix}
            -1\\
            1\\
            1
        \end{pmatrix}=\bs{b}$,
        \item 最小二乘解的通解为$\bs{x}=\bs{A}^+\bs{b}+(\bs{E}-\bs{A}^+\bs{A})\bs{y}
        =\frac{1}{11}\begin{pmatrix}
           7+y_1-y_2+3y_3\\
           4-y_1+y_2+3y_3\\
            -1+3y_1-3y_2+9y_3
        \end{pmatrix}$,其中$\bs{y}=(y_1,y_2)^T\in \R^2$。    
        最小长度的最小二乘解为$\frac{1}{11}\begin{pmatrix}
            7\\
            4\\
            -1
        \end{pmatrix}$。
    \end{enumerate}
\end{solution}

\begin{question}
    用矩阵函数求解常微分方程组初值问题的解
    \begin{align*}
    \left\{
        \begin{array}{ll}
            \frac{\d x_1}{\d t}=x_1-x_2\\
            \frac{\d x_2}{\d t}=4x_1-3x_2+1
        \end{array}
        \right.
    \left\{
        \begin{array}{ll}
            x_1(0)=1\\
            x_2(0)=2
        \end{array}
        \right.
    \end{align*}
\end{question}

\begin{solution}
    由题意得:$\frac{\d \bs{x}}{\d t}=\bs{A}\bs{x}+\bs{b}$,其中$\bs{A}=\begin{pmatrix}
        1&-1\\
        4&-3
    \end{pmatrix},\bs{b}=(0,1)^T$。
    矩阵$\bs{A}$的特征值为$\lambda_1=\lambda_2=-1$,$m_{\bs{A}}(\lambda)=(\lambda+1)^2$
    不妨设$P(\lambda)=a_0+a_1\lambda$,则 \begin{align*}
        \left\{
            \begin{array}{ll}
                P(\lambda)=P(-1)=a_0-a_1=e^{-t}\\
                P'(\lambda)=P'(-1)=a_1=te^{-t}
            \end{array}
            \right.
        \Rightarrow
        \left\{
            \begin{array}{ll}
                a_0=(1+t)e^{-t}\\
                a_1=te^{-t}
            \end{array}
            \right.
    \end{align*}
    \begin{align*}
        e^{\bs{A}t}=P(\bs{A})
    &=(1+t)e^{-t}\bs{E}+te^{-t}\bs{A}\\
    &=\begin{pmatrix}
        (2t+1)e^{-t} &-te^{-t} \\
        4te^{-t}  & (-2t+1)e^{-t}
    \end{pmatrix}
    \end{align*}
    \begin{align*}
        \bs{x}(t)&=e^{\bs{A}t}\bs{x}(0)+e^{\bs{A}t}\int_{0}^t e^{-\bs{A}u}\bs{b} du\\
        &=\begin{pmatrix}
            (2t+1)e^{-t} &-te^{-t} \\
        4te^{-t}  & (-2t+1)e^{-t}
        \end{pmatrix}\begin{pmatrix}
            1 \\
            2
        \end{pmatrix}\\
        &+\begin{pmatrix}
            (2t+1)e^{-t} &-te^{-t} \\
        4te^{-t}  & (-2t+1)e^{-t}
        \end{pmatrix}\int_{0}^t 
        \begin{pmatrix}
            (-2u+1)e^{u} &ue^{u} \\
        -4ue^{u}  & (2u+1)e^{u}
        \end{pmatrix} \begin{pmatrix}
            0\\
            1
        \end{pmatrix} du \\
        &=\begin{pmatrix}
            (t+2)e^{-t}-1 \\
            (2t+3)e^{-t}-1
        \end{pmatrix}
        \end{align*}
\end{solution}

\begin{question}

    设$\mc{D}$是三维线性空间$\bs{V}=\{(a_2t^2+a_1t+a_0)e^t|a_2,a_1,a_0 \in \R\}$
    中的微分线性变换,$f_1=t^2e^t,f_2=te^t,f_3=e^t$为$\bs{V}$的一个基。
    \begin{enumerate}[label=(\arabic{*})]
        \item 求$\mc{D}$在该基下的矩阵。
        \item 求$\mathrm{Im}\mc{D}$和$\mathrm{Ker}\mc{D}$,其中$\mathrm{Im}\mc{D}$是$\mc{D}$的像空间,$\mathrm{Ker}\mc{D}$是$\mc{D}$的核空间。
    \end{enumerate}
\end{question}

\begin{solution}
    \begin{enumerate}[label=(\arabic{*})]
        \item \begin{align*}
            \mc{D}f_1=(t^2+2t)e^t \quad
            \mc{D}f_2=(t+1)e^t \quad
            \mc{D}f_3=e^t
        \end{align*}
        即$\mc{D}(f_1,f_2,f_3)=(f_1,f_2,f_3)\begin{pmatrix}
            1& 0&0 \\
            2& 1&0 \\
            0& 1&1
        \end{pmatrix}$。
        \item $\mathrm{dim}(\mathrm{Im}\mc{D})=3,\mathrm{dim}(\mathrm{Ker}\mc{D})=3-3=0$。
        
        于是$\mathrm{Im}\mc{D}=<f_1+2f_2,f_2+f_3,f_3>=<(t^2+2t)e^t,(t+1)e^t,e^t>;\mathrm{Ker}\mc{D}=<\bs{0}>$。
    \end{enumerate}
\end{solution}

\begin{question}
    设$\R^{2\times 2}$是二阶实方阵在方阵运算下构成的线性空间,对任意$\bs{A}\in \R^{2\times 2}$,
    令$\mc{T}(\bs{A})=\bs{A}^T+\bs{A}$,
    \begin{enumerate}[label=(\arabic{*})]
        \item 证明$\mc{T}$是$\R^{2\times 2}$的线性变换。
        \item 判断$\mc{T}$是否可对角化,并说明理由。
    \end{enumerate}
\end{question}

\begin{solution}
    \begin{enumerate}[label=(\arabic{*})]
        \item 任取$\bs{X},\bs{Y} \in \R^{2\times 2},k \in \R$,
        $\mc{T}(\bs{X}+\bs{Y})=(\bs{X}+\bs{Y})^T+(\bs{X}+\bs{Y})=
        (\bs{X}^T+\bs{X})+((\bs{X}^T+\bs{X}))=\mc{T}(\bs{X})+\mc{T}(\bs{Y})$,
        这说明了对加法封闭;$\mc{T}(k\bs{X})=(k\bs{X})^T+k\bs{X}=k(\bs{X}^T+\bs{X})=k\mc{T}(\bs{X})$,
        这说明了对数乘封闭。即$\mc{T}$是$\R^{2\times 2}$的线性变换。
        \item 取$\R^{2\times 2}$的一组标准正交基$\bs{E}_1,\bs{E}_2,\bs{E}_3,\bs{E}_4$,其中
        \begin{align*}
            \bs{E}_1=\begin{pmatrix}
                1&0\\
                0&0
            \end{pmatrix} \ 
            \bs{E}_2=\begin{pmatrix}
                0&1\\
                0&0
            \end{pmatrix} \ 
            \bs{E}_3=\begin{pmatrix}
                0&0\\
                1&0
            \end{pmatrix} \ 
            \bs{E}_4=\begin{pmatrix}
                0&0\\
                0&1
            \end{pmatrix}
        \end{align*}
        计算它们在线性变换$\mc{T}$下的矩阵:
        \begin{align*}
            \mc{T}\bs{E}_1=\begin{pmatrix}
                2&0\\
                0&0
            \end{pmatrix} \ 
            \mc{T}\bs{E}_2=\begin{pmatrix}
                0&1\\
                1&0
            \end{pmatrix} \ 
            \mc{T}\bs{E}_3=\begin{pmatrix}
                0&1\\
                1&0
            \end{pmatrix} \ 
            \mc{T}\bs{E}_4=\begin{pmatrix}
                0&0\\
                0&2
            \end{pmatrix}
        \end{align*}
        于是$\mc{T}(\bs{E}_1,\bs{E}_2,\bs{E}_3,\bs{E}_4)=(\bs{E}_1,\bs{E}_2,\bs{E}_3,\bs{E}_4)\bs{A}$,
        其中$\bs{A}=\begin{pmatrix}
            2&0 &0 &0 \\
            0&1 &1 &0 \\
            0& 1& 1&0 \\
            0& 0&0 &2 
        \end{pmatrix}$,则$\bs{A}$的特征值为$\lambda_1=0,\lambda_1=\lambda_2=\lambda_3=2$。
        由于$\lambda=2$的特征值有至少三个线性无关的特征向量,所以$\mc{T}$可对角化。
    \end{enumerate}
\end{solution}

\begin{question}
    设$\mc{T}$是线性空间$\bs{V}$的线性变换且,$\mc{T}^2=\mc{T}$,证明:$\bs{V}=\mathrm{Im}\mc{T}\oplus \mathrm{Ker}\mc{T}$
,其中$\mathrm{Im}\mc{T}$是$\mc{T}$的像空间,$\mathrm{Ker}\mc{T}$是$\mc{T}$的核空间。
\end{question}

\begin{proof}
    由于$\mathrm{Im}\mc{T}$和$\mathrm{Ker}\mc{T}$均为$\bs{V}$的子空间,则
    $\mathrm{Im}\mc{T}+ \mathrm{Ker}\mc{T} \subset \bs{V}$。
    任取$\bs{\alpha}\in \mathrm{Im}\mc{T}\cap \mathrm{Ker}\mc{T}$,则存在$\bs{\beta}\in \bs{V}$,
    使得$\mc{T}\bs{\alpha}=\bs{0},\mc{T} \bs{\beta}=\bs{\alpha} $,
    于是$\bs{\alpha}=\mc{T}\bs{\beta}=\mc{T}(\mc{T} \bs{\beta})=
    \mc{T}\bs{\alpha}=\bs{0}$,即$\bs{\alpha}=\bs{0}$,此时$\mathrm{Im}\mc{T}\cap \mathrm{Ker}\mc{T}=\{\bs{0}\}$,
    这说明$\mathrm{Im}\mc{T}$和$\mathrm{Ker}\mc{T}$构成直和。
    又因为$\mathrm{dim}(\mathrm{Im}\mc{T}+\mathrm{Ker}\mc{T})=\mathrm{dim}(\mathrm{Im}\mc{T})
    +\mathrm{dim}(\mathrm{Ker}\mc{T})=n$,所以$\mathrm{Im}\mc{T}+ \mathrm{Ker}\mc{T} = \bs{V}$。
    综上,$\mathrm{Im}\mc{T}\oplus \mathrm{Ker}\mc{T} = \bs{V}$。
    
\end{proof}

\ifx\allfiles\undefined
\end{document}
\fi