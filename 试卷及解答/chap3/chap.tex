\ifx\allfiles\undefined
\documentclass[12pt, a4paper, oneside, UTF8]{ctexbook}
\def\configPath{../config}
\def\basicPath{\configPath/basic}


% 在这里定义需要的包
\usepackage{amsmath}
\usepackage{amsthm}
\usepackage{amssymb}
\usepackage{graphicx}
\usepackage{mathrsfs}
\usepackage{enumitem}
\usepackage{geometry}
\usepackage[colorlinks, linkcolor=black]{hyperref}
\usepackage{stackengine}
\usepackage{yhmath}
\usepackage{extarrows}
\usepackage{arydshln}
% \usepackage{unicode-math}
\usepackage{tasks}
\usepackage{fancyhdr}
\usepackage[dvipsnames, svgnames]{xcolor}
\usepackage{listings}

\definecolor{mygreen}{rgb}{0,0.6,0}
\definecolor{mygray}{rgb}{0.5,0.5,0.5}
\definecolor{mymauve}{rgb}{0.58,0,0.82}

\graphicspath{ {figure/},{../figure/}, {config/}, {../config/} }

\linespread{1.6}

\geometry{
    top=25.4mm, 
    bottom=25.4mm, 
    left=20mm, 
    right=20mm, 
    headheight=2.17cm, 
    headsep=4mm, 
    footskip=12mm
}

\setenumerate[1]{itemsep=5pt,partopsep=0pt,parsep=\parskip,topsep=5pt}
\setitemize[1]{itemsep=5pt,partopsep=0pt,parsep=\parskip,topsep=5pt}
\setdescription{itemsep=5pt,partopsep=0pt,parsep=\parskip,topsep=5pt}

\lstset{
    language=Mathematica,
    basicstyle=\tt,
    breaklines=true,
    keywordstyle=\bfseries\color{NavyBlue}, 
    emphstyle=\bfseries\color{Rhodamine},
    commentstyle=\itshape\color{black!50!white}, 
    stringstyle=\bfseries\color{PineGreen!90!black},
    columns=flexible,
    numbers=left,
    numberstyle=\footnotesize,
    frame=tb,
    breakatwhitespace=false,
} 
% 在这里定义自己顺手的环境
\def\d{\mathrm{d}}
\def\i{\mathrm{i}}
\def\R{\mathbb{R}}
\newcommand{\bs}[1]{\boldsymbol{#1}}
\newcommand{\mc}[1]{\mathcal{#1}}
\newcommand{\ora}[1]{\overrightarrow{#1}}
\newcommand{\myspace}[1]{\par\vspace{#1\baselineskip}}
\newcommand{\xrowht}[2][0]{\addstackgap[.5\dimexpr#2\relax]{\vphantom{#1}}}
\newenvironment{ca}[1][1]{\linespread{#1} \selectfont \begin{cases}}{\end{cases}}
\newenvironment{vx}[1][1]{\linespread{#1} \selectfont \begin{vmatrix}}{\end{vmatrix}}
\newcommand{\tabincell}[2]{\begin{tabular}{@{}#1@{}}#2\end{tabular}}
\newcommand{\pll}{\kern 0.56em/\kern -0.8em /\kern 0.56em}
\newcommand{\dive}[1][F]{\mathrm{div}\;\bs{#1}}
\newcommand{\rotn}[1][A]{\mathrm{rot}\;\bs{#1}} 
\usepackage[strict]{changepage} 
\usepackage{framed}

\definecolor{greenshade}{rgb}{0.90,1,0.92}
\definecolor{redshade}{rgb}{1.00,0.88,0.88}
\definecolor{brownshade}{rgb}{0.99,0.95,0.9}
\definecolor{lilacshade}{rgb}{0.95,0.93,0.98}
\definecolor{orangeshade}{rgb}{1.00,0.88,0.82}
\definecolor{lightblueshade}{rgb}{0.8,0.92,1}
\definecolor{purple}{rgb}{0.81,0.85,1}
\theoremstyle{definition}
\newtheorem{myDefn}{\indent 定义}[section]
% \newtheorem{myLemma}{\indent 引理}[section]
\newtheorem{myLemma}{\indent 引理}[chapter]
\newtheorem{myThm}[myLemma]{\indent 定理}
\newtheorem{myCorollary}[myLemma]{\indent 推论}
\newtheorem{myCriterion}[myLemma]{\indent 准则}
\newtheorem*{myRemark}{\indent 注}
\newtheorem{myProposition}{\indent 命题}[section]


\newenvironment{formal}[2][]{%
    \def\FrameCommand{%
        \hspace{1pt}%
        {\color{#1}\vrule width 2pt}%
        {\color{#2}\vrule width 4pt}%
        \colorbox{#2}%
    }%
    \MakeFramed{\advance\hsize-\width\FrameRestore}%
    \noindent\hspace{-4.55pt}%
    \begin{adjustwidth}{}{7pt}\vspace{2pt}\vspace{2pt}}{%
        \vspace{2pt}\end{adjustwidth}\endMakeFramed%
}

\newenvironment{defn}{\begin{formal}[Green]{greenshade}\vspace{-\baselineskip / 2}\begin{myDefn}}{\end{myDefn}\end{formal}}
\newenvironment{thm}{\begin{formal}[LightSkyBlue]{lightblueshade}\vspace{-\baselineskip / 2}\begin{myThm}}{\end{myThm}\end{formal}}
\newenvironment{lemma}{\begin{formal}[Plum]{lilacshade}\vspace{-\baselineskip / 2}\begin{myLemma}}{\end{myLemma}\end{formal}}
\newenvironment{corollary}{\begin{formal}[BurlyWood]{brownshade}\vspace{-\baselineskip / 2}\begin{myCorollary}}{\end{myCorollary}\end{formal}}
\newenvironment{criterion}{\begin{formal}[DarkOrange]{orangeshade}\vspace{-\baselineskip / 2}\begin{myCriterion}}{\end{myCriterion}\end{formal}}
\newenvironment{rmk}{\begin{formal}[LightCoral]{redshade}\vspace{-\baselineskip / 2}\begin{myRemark}}{\end{myRemark}\end{formal}}
\newenvironment{proposition}{\begin{formal}[RoyalPurple]{purple}\vspace{-\baselineskip / 2}\begin{myProposition}}{\end{myProposition}\end{formal}}

\newtheorem{example}{\indent \color{SeaGreen}{例}}[section]
\newtheorem{question}{\color{SeaGreen}{题}}[chapter]
% \renewenvironment{proof}{\indent\textcolor{SkyBlue}{\textbf{证明.}}\;}{\qed\par}
% \newenvironment{solution}{\indent\textcolor{SkyBlue}{\textbf{解.}}\;}{\qed\par}

\renewcommand{\proofname}{\textbf{\textcolor{TealBlue}{证明}}}
\newenvironment{solution}{\begin{proof}[\textbf{\textcolor{TealBlue}{解}}]}{\end{proof}}

\definecolor{mygreen}{rgb}{0,0.6,0}
\definecolor{mygray}{rgb}{0.5,0.5,0.5}
\definecolor{mymauve}{rgb}{0.58,0,0.82}

\graphicspath{ {figure/},{../figure/}, {config/}, {../config/},{cover/graph} }

\linespread{1.6}

\geometry{
    top=25.4mm, 
    bottom=25.4mm, 
    left=20mm, 
    right=20mm, 
    headheight=2.17cm, 
    headsep=4mm, 
    footskip=12mm
}

\setenumerate[1]{itemsep=5pt,partopsep=0pt,parsep=\parskip,topsep=5pt}
\setitemize[1]{itemsep=5pt,partopsep=0pt,parsep=\parskip,topsep=5pt}
\setdescription{itemsep=5pt,partopsep=0pt,parsep=\parskip,topsep=5pt}

\lstset{
    language=Mathematica,
    basicstyle=\tt,
    breaklines=true,
    keywordstyle=\bfseries\color{NavyBlue}, 
    emphstyle=\bfseries\color{Rhodamine},
    commentstyle=\itshape\color{black!50!white}, 
    stringstyle=\bfseries\color{PineGreen!90!black},
    columns=flexible,
    numbers=left,
    numberstyle=\footnotesize,
    frame=tb,
    breakatwhitespace=false,
} 

\begin{document}
\else
\fi

\chapter{模拟卷三}
\begin{question} 
   设$\bs{A}=\begin{pmatrix}
    1&-1&0\\
    2&0&-1\\
    0&4&1
   \end{pmatrix}$,
   求矩阵$\bs{A}$的LR分解。
\end{question}

\begin{question}
    设线性方程组$\left\{
        \begin{array}{ll}
            x_1-2x_2=1\\
            3x_1-6x_2=1\\
            -x_1+2x_2=-6
        \end{array}
        \right.$
    ,用广义逆验证它是矛盾方程,并求它的最小二乘解的通解。
\end{question}

\begin{question}
    设$\bs{A}=\begin{pmatrix}
        0&1&2\\
        -4&3&4\\
        1&0&1
    \end{pmatrix}$,
    求可逆阵$\bs{P}$和$\bs{A}$的若当(Jordan)标准型$\bs{J}$,
    使$\bs{P}^{-1}\bs{A}\bs{P}=\bs{J}$,并求$e^{2\bs{A}t}$。
\end{question}


\begin{question}
    设$\mc{T}$为线性空间$\R^{2\times 2}$上的变换,$\mc{T}(\bs{X})=\bs{A}\bs{X}\bs{A},\bs{X} \in \R^{2\times 2}$,
    其中$\bs{A}=\begin{pmatrix}
        1&-1\\
        0&1
    \end{pmatrix}$,
    求线性变换$\mc{T}$在基$\bs{A}_1=\begin{pmatrix}
            1&1\\
            1&1
        \end{pmatrix},\bs{A}_2=\begin{pmatrix}
            0&-1\\
            1&0
        \end{pmatrix},\bs{A}_3=\begin{pmatrix}
            1&-1\\
            0&0
        \end{pmatrix},\bs{A}_4=\begin{pmatrix}
            1&0\\
            0&0
        \end{pmatrix}$下的矩阵,
        并求$\mc{T}$的特征值。
\end{question}

\begin{question}
    用矩阵函数求解常微分方程组初值问题的解
    \begin{align*}
    \left\{
        \begin{array}{ll}
            \frac{\d \bs{x}}{\d t}=\begin{pmatrix}
                -5 &1\\
                -1&-3
            \end{pmatrix}\bs{x}\\
            \bs{x}(t)|_{t=0}=(1,0)^T
        \end{array}
        \right.
    \end{align*}
\end{question}

\begin{question}
    在线性空间$\R^{2\times 2}$中,对于任意的$\bs{A},\bs{B}\in \R^{2\times 2}$,
    定义$\bs{A}$与$\bs{B}$的内积为$(\bs{A},\bs{B})=\mathrm{tr}(\bs{A}^T\bs{B})$,
    $\bs{V}=\{\bs{A}|\bs{A}\in \R^{2\times 2} ,\mathrm{tr}(\bs{A})=0\}$为$\R^{2\times2}$的子集,
    其中$\mathrm{tr}(\bs{A})=a_{11}+a_{22}$为$\bs{A}=(a_{ij})_{2\times 2}$的迹。
    \begin{enumerate}[label=(\arabic{*})]
        \item 证明:$\bs{V}$是$\R^{2\times 2}$的子空间。
        \item 求$\bs{V}$的一组标准正交基,及$\bs{V}$的正交补$\bs{V}^{\perp}$。
    \end{enumerate}
\end{question}


\begin{question}
    设$\mc{T}$是$n$维线性空间$\bs{V}$的线性变换,$\mathrm{rank}(\mc{T})=r>0$,$\mc{T}^2=\mc{T}$,证明:
    \begin{enumerate}[label=(\arabic{*})]
        \item 存在$\bs{V}$的一组基$\bs{\alpha}_1,\bs{\alpha}_2,\ldots,\bs{\alpha}_n$,
        满足$\mc{T}(\bs{\alpha}_i)=\left\{
            \begin{array}{ll}
                \bs{\alpha}_i,1\leq i\leq r\\
                \bs{0},r \leq i \leq n
            \end{array}
            \right.$,其中$\bs{\alpha}_{r+1},\ldots,\bs{\alpha}_n$是$\mathrm{Ker}\mc{T}$的基。
        \item 写出$\mc{T}$在基$\bs{\alpha}_1,\bs{\alpha}_2,\ldots,\bs{\alpha}_n$下的矩阵,以及$\mc{T}$的最小多项式。
    \item $\bs{V}=\mathrm{Im}\mc{T}\oplus \mathrm{Ker}\mc{T}$
,其中$\mathrm{Im}\mc{T}$是$\mc{T}$的像空间,$\mathrm{Ker}\mc{T}$是$\mc{T}$的核空间。
    \end{enumerate}
\end{question}

\ifx\allfiles\undefined
\end{document}
\fi