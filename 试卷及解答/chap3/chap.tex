\ifx\allfiles\undefined
\documentclass[12pt, a4paper, oneside, UTF8]{ctexbook}
\def\configPath{../config}
\def\coverPath{\configPath/cover}
\def\packagePath{\configPath/package}
\def\theormPath{\configPath/theorem}
\def\customPath{\configPath/custom}
\def\prefacePath{\configPath/preface}

% 在这里定义需要的包
\usepackage{amsmath}
\usepackage{amsthm}
\usepackage{amssymb}
\usepackage{graphicx}
\usepackage{mathrsfs}
\usepackage{enumitem}
\usepackage{geometry}
\usepackage[colorlinks, linkcolor=black]{hyperref}
\usepackage{stackengine}
\usepackage{yhmath}
\usepackage{extarrows}
\usepackage{arydshln}
% \usepackage{unicode-math}
\usepackage{tasks}
\usepackage{fancyhdr}
\usepackage[dvipsnames, svgnames]{xcolor}
\usepackage{listings}


\input{\theormPath/theorem1_zh}
\input{\customPath/custom}




\begin{document}
\else
\fi

\chapter{模拟卷三}
\begin{question} 
   设$\bs{A}=\begin{pmatrix}
    1&-1&0\\
    2&0&-1\\
    0&4&1
   \end{pmatrix}$,
   求矩阵$\bs{A}$的LR分解。
\end{question}

\begin{solution}
    \begin{align*}
        \begin{pmatrix}
            1&-1&0\\
            2&0&-1\\
            0&4&1
           \end{pmatrix}\xrightarrow[]{r_2-2r_1}
           \begin{pmatrix}
            1&-1&0\\
            0&2&-1\\
            0&4&1
           \end{pmatrix}\xrightarrow[]{r_3-2r_2}
           \begin{pmatrix}
            1&-1&0\\
            0&2&-1\\
            0&0&3
           \end{pmatrix}
    \end{align*}
    上面矩阵即为所求矩阵$\bs{R}$。则$\bs{P}_2\bs{P}_1\bs{A}=\bs{R}$,于是:
    \begin{align*}
        \bs{L}=(\bs{P}_2\bs{P}_1)^{-1}=\bs{P}_1^{-1}\bs{P}_2^{-1}
        =\begin{pmatrix}
            1&0&0\\
            2&1&0\\
            0&0&1
        \end{pmatrix}
        \begin{pmatrix}
            1&0&0\\
            0&1&0\\
            0&2&1
        \end{pmatrix}=\begin{pmatrix}
            1&0&0\\
            2&1&0\\
            0&2&1
        \end{pmatrix}
    \end{align*}
    即
    \begin{align*}
        \bs{A}=\bs{L}\bs{R}=\begin{pmatrix}
            1&0&0\\
            2&1&0\\
            0&2&1
        \end{pmatrix}
        \begin{pmatrix}
            1&-1&0\\
            0&2&-1\\
            0&0&3
           \end{pmatrix}
    \end{align*}
\end{solution}

\begin{question}
    设线性方程组$\left\{
        \begin{array}{ll}
            x_1-2x_2=1\\
            3x_1-6x_2=1\\
            -x_1+2x_2=-6
        \end{array}
        \right.$
    ,用广义逆验证它是矛盾方程,并求它的最小二乘解的通解。
\end{question}

\begin{solution}
    由题意得:$\bs{A}\bs{x}=\bs{b}$,其中$\bs{A}=\begin{pmatrix}
        1&-2\\
        3&-6\\
        -1&2
    \end{pmatrix},\bs{x}=(x_1,x_2)^T,\bs{b}=(1,1,-6)^T$。下面计算$\bs{A}^+$:
    
    对矩阵$\bs{A}$进行满秩分解,得:
    $\bs{A}=\bs{B}\bs{C}=\begin{pmatrix}
        1 \\
        3\\
        -1
    \end{pmatrix}\begin{pmatrix}
        1&-2
    \end{pmatrix}$,于是:
    \begin{align*}
        \bs{C}\bs{C}^T=\begin{pmatrix}
            5
        \end{pmatrix} \quad \bs{B}^T\bs{B}=\begin{pmatrix}
            11
        \end{pmatrix}
    \end{align*}
    \begin{align*}
        \bs{A}^+&=
    \bs{C}^T(\bs{C}\bs{C}^T)^{-1}(\bs{B}^T\bs{B})^{-1}\bs{B}^T\\
    &=\begin{pmatrix}
        1 \\
        -2
    \end{pmatrix}\begin{pmatrix}
        5
    \end{pmatrix}^{-1}\begin{pmatrix}
        11
    \end{pmatrix}^{-1}
    \begin{pmatrix}
        1&3&-1
    \end{pmatrix}\\
    &=\frac{1}{55}\begin{pmatrix}
       1&3&-1\\
       -2&-6&2
    \end{pmatrix}
    \end{align*}
    则$\bs{A}\bs{A}^+\bs{b}=\frac{10}{11}\begin{pmatrix}
        1\\
        3\\
        -1
    \end{pmatrix} \neq \begin{pmatrix}
        1\\
        1\\
        -6
    \end{pmatrix}=\bs{b}$,
    最小二乘解的通解为$\bs{x}=\bs{A}^+\bs{b}+(\bs{E}-\bs{A}^+\bs{A})\bs{y}
    =\begin{pmatrix}
       \frac{2}{11}+\frac{4}{5}y_1+\frac{2}{5}y_2\\
       -\frac{4}{11}+\frac{2}{5}y_1+\frac{1}{5}y_2 
        
    \end{pmatrix}$,其中$\bs{y}=(y_1,y_2)^T\in \R^2$。
    
\end{solution}

\begin{question}
    设$\bs{A}=\begin{pmatrix}
        0&1&2\\
        -4&3&4\\
        1&0&1
    \end{pmatrix}$,
    求可逆阵$\bs{P}$和$\bs{A}$的若当(Jordan)标准型$\bs{J}$,
    使$\bs{P}^{-1}\bs{A}\bs{P}=\bs{J}$。
\end{question}

\begin{solution}
    特征值为$\lambda_1=2,\lambda_2=\lambda_3=1$,对应的若当标准型为$\bs{J}=\begin{pmatrix}
        2 & & \\
        & 1& 1\\
        & & 1
    \end{pmatrix}$,其中空白位置全是$0$。
    由于$\bs{J}=\bs{P}^{-1}\bs{A}\bs{P} \Rightarrow \bs{P}\bs{J}=\bs{A}\bs{P}$,
    不妨令$\bs{P}=(\bs{p}_1,\bs{p}_2,\bs{p}_3)$,其中$\bs{P}$为非奇异矩阵,则:
    \begin{align*}
        \left\{
            \begin{array}{ll}
                \bs{A}\bs{p}_1=2\bs{p}_1 \\
                \bs{A}\bs{p}_2=\bs{p}_2 \\
                \bs{A}\bs{p}_3=\bs{p}_2+\bs{p}_3
            \end{array} 
            \right.
            \Rightarrow
            \left\{
            \begin{array}{ll}
                \bs{p}_1=(1,0,1)^T \\
                \bs{p}_2=(0,2,-1)^T \\
                \bs{p}_3=(-1,-1,0)^T
            \end{array} 
            \right.
    \end{align*}
    \begin{align*}
        \bs{P}=\begin{pmatrix}
            1 & 0 &-1 \\
            0 & 2 &-1 \\
            1 & -1 &0
        \end{pmatrix} \quad
        \bs{P}^{-1}=\begin{pmatrix}
            -1&1&2\\
            -1&1&1\\
            -2&1&2       
        \end{pmatrix}
    \end{align*}
\end{solution}

\begin{question}
    设$\mc{T}$为线性空间$\R^{2\times 2}$上的变换,$\mc{T}(\bs{X})=\bs{A}\bs{X}\bs{A},\bs{X} \in \R^{2\times 2}$,
    其中$\bs{A}=\begin{pmatrix}
        1&-1\\
        0&1
    \end{pmatrix}$,
    求线性变换$\mc{T}$在基$\bs{A}_1=\begin{pmatrix}
            1&1\\
            1&1
        \end{pmatrix},\bs{A}_2=\begin{pmatrix}
            0&-1\\
            1&0
        \end{pmatrix},\bs{A}_3=\begin{pmatrix}
            1&-1\\
            0&0
        \end{pmatrix},\bs{A}_4=\begin{pmatrix}
            1&0\\
            0&0
        \end{pmatrix}$下的矩阵,
        并求$\mc{T}$的特征值。
\end{question}

\begin{solution}
    \begin{align*}
        \mc{T}\bs{A}_1=\begin{pmatrix}
            0&0\\
            1&0
        \end{pmatrix} \quad
        \mc{T}\bs{A}_2=\begin{pmatrix}
            -1&0\\
            1&-1
        \end{pmatrix}\quad
        \mc{T}\bs{A}_3=\begin{pmatrix}
            1&-2\\
            0&0
        \end{pmatrix}\quad
        \mc{T}\bs{A}_4=\begin{pmatrix}
            1&-1\\
            0&0
        \end{pmatrix}
    \end{align*}
        即
        \begin{align*}
            \mc{T}(\bs{A}_1,\bs{A}_2,\bs{A}_3,\bs{A}_4)=
            (\bs{A}_1,\bs{A}_2,\bs{A}_3,\bs{A}_4)
            \begin{pmatrix}
                0& -1&0&0  \\
                1 & 2&0 &0 \\
                -1& -3& 2& 1\\
                1& 3& -1&0
            \end{pmatrix}
        \end{align*}
        $\lambda_1=\lambda_2=\lambda_3=\lambda_4=1$。
\end{solution}

\begin{question}
    用矩阵函数求解常微分方程组初值问题的解
    \begin{align*}
    \left\{
        \begin{array}{ll}
            \frac{\d \bs{x}}{\d t}=\begin{pmatrix}
                -5 &1\\
                -1&-3
            \end{pmatrix}\bs{x}\\
            \bs{x}(t)|_{t=0}=(1,0)^T
        \end{array}
        \right.
    \end{align*}
\end{question}

\begin{solution}
    由题意得:$\frac{\d \bs{x}}{\d t}=\bs{A}\bs{x}$,其中$\bs{A}=\begin{pmatrix}
        -5&1\\
        -1&-3
    \end{pmatrix}$。
    矩阵$\bs{A}$的特征值为$\lambda_1=\lambda_2=-4$,$m_{\bs{A}}(\lambda)=(\lambda+4)^2$
    不妨设$P(\lambda)=a_0+a_1\lambda$,则 \begin{align*}
        \left\{
            \begin{array}{ll}
                P(\lambda)=P(-4)=a_0-4a_1=e^{-4t}\\
                P'(\lambda)=P'(-4)=a_1=te^{-4t}
            \end{array}
            \right.
        \Rightarrow
        \left\{
            \begin{array}{ll}
                a_0=(1+4t)e^{-4t}\\
                a_1=te^{-4t}
            \end{array}
            \right.
    \end{align*}
    \begin{align*}
        e^{\bs{A}t}=P(\bs{A})
    &=(1+4t)e^{-4t}\bs{E}+te^{-4t}\bs{A}\\
    &=\begin{pmatrix}
        (-t+1)e^{-4t} &te^{-4t} \\
        -te^{-4t}  & (t+1)e^{-4t}
    \end{pmatrix}
    \end{align*}
    \begin{align*}
        \bs{x}(t)&=e^{\bs{A}t}\bs{x}(0)\\
        &=\begin{pmatrix}
            (-t+1)e^{-4t} &te^{-4t} \\
        -te^{-4t}  & (t+1)e^{-4t}
        \end{pmatrix}\begin{pmatrix}
            1 \\
            0
        \end{pmatrix}\\
        &=\begin{pmatrix}
            (-t+1)e^{-4t} \\
            -te^{-4t} 
        \end{pmatrix}
        \end{align*}
\end{solution}

\begin{question}
    在线性空间$\R^{2\times 2}$中,对于任意的$\bs{A},\bs{B}\in \R^{2\times 2}$,
    定义$\bs{A}$与$\bs{B}$的内积为$(\bs{A},\bs{B})=\mathrm{tr}(\bs{A}^T\bs{B})$,
    $\bs{V}=\{\bs{A}|\bs{A}\in \R^{2\times 2} ,\mathrm{tr}(\bs{A})=0\}$为$\R^{2\times2}$的子集,
    其中$\mathrm{tr}(\bs{A})=a_{11}+a_{22}$为$\bs{A}=(a_{ij})_{2\times 2}$的迹。
    \begin{enumerate}[label=(\arabic{*})]
        \item 证明:$\bs{V}$是$\R^{2\times 2}$的子空间。
        \item 求$\bs{V}$的一组标准正交基,及$\bs{V}$的正交补$\bs{V}^{\perp}$。
    \end{enumerate}
\end{question}

\begin{solution}
    \begin{enumerate}[label=(\arabic{*})]
        \item 证明其满足加法封闭及数乘封闭即可。
        
        任取$\bs{A},\bs{B} \in \R^{2\times 2},k \in \R$。
        $\mathrm{tr}(\bs{A}+\bs{B})=\mathrm{tr}(\bs{A})+\mathrm{tr}(\bs{B})=0+0=0$,这说明了对加法封闭;
        $\mathrm{tr}(k\bs{A})=k\mathrm{tr}(\bs{A})=0$,这说明了对数乘封闭。即$\bs{V}$是$\R^{2\times 2}$的子空间。
        \item 显然,$\mathrm{dim}(\bs{V})=3,\mathrm{dim}(\bs{V}^{\perp})=4-3=1$。
        
        $\bs{E}_1=\begin{pmatrix}
            \frac{1}{\sqrt{2}}&0\\
            0&-\frac{1}{\sqrt{2}}
        \end{pmatrix} \quad 
        \bs{E}_2=\begin{pmatrix}
            0&1\\
            0&0
        \end{pmatrix} \quad 
        \bs{E}_3=\begin{pmatrix}
            0&0\\
            1&0
        \end{pmatrix}$构成了$\bs{V}$的一组标准正交基。

        $\bs{E}_4=\begin{pmatrix}
            \frac{1}{\sqrt{2}}&0\\
            0&\frac{1}{\sqrt{2}}
        \end{pmatrix}$构成了$\bs{V}^{\perp}$的一组标准正交基。
    \end{enumerate}
\end{solution}

\begin{question}
    设$\mc{T}$是$n$维线性空间$\bs{V}$的线性变换,$\mathrm{rank}(\mc{T})=r>0$,$\mc{T}^2=\mc{T}$,证明:
    \begin{enumerate}[label=(\arabic{*})]
        \item 存在$\bs{V}$的一组基$\bs{\alpha}_1,\bs{\alpha}_2,\ldots,\bs{\alpha}_n$,
        满足$\mc{T}(\bs{\alpha}_i)=\left\{
            \begin{array}{ll}
                \bs{\alpha}_i,1\leq i\leq r\\
                \bs{0},r \leq i \leq n
            \end{array}
            \right.$,其中$\bs{\alpha}_{r+1},\ldots,\bs{\alpha}_n$是$\mathrm{Ker}\mc{T}$的基。
        \item 写出$\mc{T}$在基$\bs{\alpha}_1,\bs{\alpha}_2,\ldots,\bs{\alpha}_n$下的矩阵,以及$\mc{T}$的最小多项式。
    \item $\bs{V}=\mathrm{Im}\mc{T}\oplus \mathrm{Ker}\mc{T}$
,其中$\mathrm{Im}\mc{T}$是$\mc{T}$的像空间,$\mathrm{Ker}\mc{T}$是$\mc{T}$的核空间。
    \end{enumerate}
\end{question}

\begin{proof}
    \begin{enumerate}[label=(\arabic{*})]
        \item
    即证存在$\bs{V}$的一组基$\bs{\alpha}_1,\bs{\alpha}_2,\ldots,\bs{\alpha}_n$,
    使得$\mc{T}(\bs{\alpha}_1,\ldots,\bs{\alpha}_n)=(\bs{\alpha}_1,\ldots,\bs{\alpha}_n)\begin{pmatrix}
        \bs{E}_r& \bs{O}\\
        \bs{O}& \bs{O}
    \end{pmatrix}$。

    由题意与Hamilton-Cayley定理可知,$\lambda=1$或$0$。
    
    取$n$维线性空间$\bs{V}$下的一组标准正交基,并记线性变换$\mc{T}$在该组基下的矩阵为$\bs{A}$。
    \begin{enumerate}[label=(\arabic{*})]
        \item 若$r=n$,则$\lambda=1$,$m_{\bs{A}}(\lambda)=\lambda-1$,说明$\bs{A}$可相似对角化。
        则存在可逆矩阵$\bs{P}$,
        使得$\bs{P}^{-1}\bs{A}\bs{P}=\bs{\Lambda}$,其中$\bs{\Lambda}=\mathrm{diag}\{1,\ldots,1\}=\bs{E}$。
        此时$\mathrm{dim}(\mathrm{Ker}\mc{T})=0$,
        即$\bs{0}$为$\mathrm{Ker}\mc{T}$的基,满足题意。
        \item 若$r=0$,则$\lambda=0$,$m_{\bs{A}}(\lambda)=\lambda$,说明$\bs{A}$可相似对角化。
        则存在可逆矩阵$\bs{P}$,
        使得$\bs{P}^{-1}\bs{A}\bs{P}=\bs{\Lambda}$,其中$\bs{\Lambda}=\mathrm{diag}\{0,\ldots,0\}=\bs{O}$。
        此时$\mathrm{dim}(\mathrm{Ker}\mc{T})=n$,即$\bs{a}_1,\ldots,\bs{a}_n$为$\mathrm{Ker}\mc{T}$的基,满足题意。
        \item 若$0<r<n$,$\lambda=1$或$0$,$m_{\bs{A}}(\lambda)=\lambda(\lambda-1)$,说明$\bs{A}$可相似对角化。
        则存在可逆矩阵$\bs{P}$,
        使得$\bs{P}^{-1}\bs{A}\bs{P}=\bs{\Lambda}$,
        其中$\bs{\Lambda}=\mathrm{diag}\{\underbrace{1,\ldots,1}_{r\text{个}1},0,\ldots,0\}=
        \begin{pmatrix}
            \bs{E}_r &\bs{O}\\
            \bs{O} & \bs{O}
        \end{pmatrix}$。
        此时$\mathrm{dim}(\mathrm{Ker}\mc{T})=n-r$,
        即$\bs{a}_{r+1},\ldots,\bs{a}_n$为$\mathrm{Ker}\mc{T}$的基,
        满足题意。
    \end{enumerate}
    \item $\mc{T}$在基$\bs{\alpha}_1,\bs{\alpha}_2,\ldots,\bs{\alpha}_n$下的矩阵为$\begin{pmatrix}
        \bs{E}_r& \bs{O}\\
        \bs{O}& \bs{O}
    \end{pmatrix}$。
    \begin{enumerate}[label=(\arabic{*})]
        \item 若$r=n$,则$\lambda=1$,$m_{\bs{A}}(\lambda)=\lambda-1$。
        \item 若$r=0$,则$\lambda=0$,$m_{\bs{A}}(\lambda)=\lambda$。
        \item 若$0<r<n$,则$\lambda=1$或$0$,$m_{\bs{A}}(\lambda)=\lambda(\lambda-1)$。
    \end{enumerate}

    \item 由于$\mathrm{Im}\mc{T}$和$\mathrm{Ker}\mc{T}$均为$\bs{V}$的子空间,则
    $\mathrm{Im}\mc{T}+ \mathrm{Ker}\mc{T} \subset \bs{V}$。
    任取$\bs{\alpha}\in \mathrm{Im}\mc{T}\cap \mathrm{Ker}\mc{T}$,则存在$\bs{\beta}\in \bs{V}$,
    使得$\mc{T}\bs{\alpha}=\bs{0},\mc{T} \bs{\beta}=\bs{\alpha} $,
    于是$\bs{\alpha}=\mc{T}\bs{\beta}=\mc{T}(\mc{T} \bs{\beta})=
    \mc{T}\bs{\alpha}=\bs{0}$,即$\bs{\alpha}=\bs{0}$,此时$\mathrm{Im}\mc{T}\cap \mathrm{Ker}\mc{T}=\{\bs{0}\}$,
    这说明$\mathrm{Im}\mc{T}$和$\mathrm{Ker}\mc{T}$构成直和。
    又因为$\mathrm{dim}(\mathrm{Im}\mc{T}+\mathrm{Ker}\mc{T})=\mathrm{dim}(\mathrm{Im}\mc{T})
    +\mathrm{dim}(\mathrm{Ker}\mc{T})=n$,所以$\mathrm{Im}\mc{T}+ \mathrm{Ker}\mc{T} = \bs{V}$。
    综上,$\mathrm{Im}\mc{T}\oplus \mathrm{Ker}\mc{T} = \bs{V}$。
\end{enumerate}
\end{proof}

\ifx\allfiles\undefined
\end{document}
\fi