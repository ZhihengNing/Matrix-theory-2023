\ifx\allfiles\undefined
\documentclass[12pt, a4paper, oneside, UTF8]{ctexbook}
\def\configPath{../config}
\def\coverPath{\configPath/cover}
\def\packagePath{\configPath/package}
\def\theormPath{\configPath/theorem}
\def\customPath{\configPath/custom}
\def\prefacePath{\configPath/preface}

% 在这里定义需要的包
\usepackage{amsmath}
\usepackage{amsthm}
\usepackage{amssymb}
\usepackage{graphicx}
\usepackage{mathrsfs}
\usepackage{enumitem}
\usepackage{geometry}
\usepackage[colorlinks, linkcolor=black]{hyperref}
\usepackage{stackengine}
\usepackage{yhmath}
\usepackage{extarrows}
\usepackage{arydshln}
% \usepackage{unicode-math}
\usepackage{tasks}
\usepackage{fancyhdr}
\usepackage[dvipsnames, svgnames]{xcolor}
\usepackage{listings}


\input{\theormPath/theorem1_zh}
\input{\customPath/custom}




\begin{document}
\else
\fi

\chapter{模拟卷三}
\begin{question} 
   设$\bs{A}=\begin{pmatrix}
    1&-1&0\\
    2&0&-1\\
    0&4&1
   \end{pmatrix}$,
   求矩阵$\bs{A}$的LR分解。
\end{question}

\begin{question}
    设线性方程组$\left\{
        \begin{array}{ll}
            x_1-2x_2=1\\
            3x_1-6x_2=1\\
            -x_1+2x_2=-6
        \end{array}
        \right.$
    ,用广义逆验证它是矛盾方程,并求它的最小二乘解的通解。
\end{question}

\begin{question}
    设$\bs{A}=\begin{pmatrix}
        0&1&2\\
        -4&3&4\\
        1&0&1
    \end{pmatrix}$,
    求可逆阵$\bs{P}$和$\bs{A}$的若当(Jordan)标准型$\bs{J}$,
    使$\bs{P}^{-1}\bs{A}\bs{P}=\bs{J}$,并求$e^{2\bs{A}t}$。
\end{question}


\begin{question}
    设$\mc{T}$为线性空间$\R^{2\times 2}$上的变换,$\mc{T}(\bs{X})=\bs{A}\bs{X}\bs{A},\bs{X} \in \R^{2\times 2}$,
    其中$\bs{A}=\begin{pmatrix}
        1&-1\\
        0&1
    \end{pmatrix}$,
    求线性变换$\mc{T}$在基$\bs{A}_1=\begin{pmatrix}
            1&1\\
            1&1
        \end{pmatrix},\bs{A}_2=\begin{pmatrix}
            0&-1\\
            1&0
        \end{pmatrix},\bs{A}_3=\begin{pmatrix}
            1&-1\\
            0&0
        \end{pmatrix},\bs{A}_4=\begin{pmatrix}
            1&0\\
            0&0
        \end{pmatrix}$下的矩阵,
        并求$\mc{T}$的特征值。
\end{question}

\begin{question}
    用矩阵函数求解常微分方程组初值问题的解
    \begin{align*}
    \left\{
        \begin{array}{ll}
            \frac{\d \bs{x}}{\d t}=\begin{pmatrix}
                -5 &1\\
                -1&-3
            \end{pmatrix}\bs{x}\\
            \bs{x}(t)|_{t=0}=(1,0)^T
        \end{array}
        \right.
    \end{align*}
\end{question}

\begin{question}
    在线性空间$\R^{2\times 2}$中,对于任意的$\bs{A},\bs{B}\in \R^{2\times 2}$,
    定义$\bs{A}$与$\bs{B}$的内积为$(\bs{A},\bs{B})=\mathrm{tr}(\bs{A}^T\bs{B})$,
    $\bs{V}=\{\bs{A}|\bs{A}\in \R^{2\times 2} ,\mathrm{tr}(\bs{A})=0\}$为$\R^{2\times2}$的子集,
    其中$\mathrm{tr}(\bs{A})=a_{11}+a_{22}$为$\bs{A}=(a_{ij})_{2\times 2}$的迹。
    \begin{enumerate}[label=(\arabic{*})]
        \item 证明:$\bs{V}$是$\R^{2\times 2}$的子空间。
        \item 求$\bs{V}$的一组标准正交基,及$\bs{V}$的正交补$\bs{V}^{\perp}$。
    \end{enumerate}
\end{question}


\begin{question}
    设$\mc{T}$是$n$维线性空间$\bs{V}$的线性变换,$\mathrm{rank}(\mc{T})=r>0$,$\mc{T}^2=\mc{T}$,证明:
    \begin{enumerate}[label=(\arabic{*})]
        \item 存在$\bs{V}$的一组基$\bs{\alpha}_1,\bs{\alpha}_2,\ldots,\bs{\alpha}_n$,
        满足$\mc{T}(\bs{\alpha}_i)=\left\{
            \begin{array}{ll}
                \bs{\alpha}_i,1\leq i\leq r\\
                \bs{0},r \leq i \leq n
            \end{array}
            \right.$,其中$\bs{\alpha}_{r+1},\ldots,\bs{\alpha}_n$是$\mathrm{Ker}\mc{T}$的基。
        \item 写出$\mc{T}$在基$\bs{\alpha}_1,\bs{\alpha}_2,\ldots,\bs{\alpha}_n$下的矩阵,以及$\mc{T}$的最小多项式。
    \item $\bs{V}=\mathrm{Im}\mc{T}\oplus \mathrm{Ker}\mc{T}$
,其中$\mathrm{Im}\mc{T}$是$\mc{T}$的像空间,$\mathrm{Ker}\mc{T}$是$\mc{T}$的核空间。
    \end{enumerate}
\end{question}

\ifx\allfiles\undefined
\end{document}
\fi